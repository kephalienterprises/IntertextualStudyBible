\documentclass[12pt, twoside, openright]{book}
\usepackage{ebgaramond}
\usepackage[scaled=0.90]{helvet} %Helvetica
\usepackage[T1]{fontenc}
\usepackage{titlesec, xcolor}
\usepackage[spanish]{babel}
\usepackage{bibleref-spanish}
\usepackage{indextools}
\usepackage{enumitem}

%index

\def\atest{{\large \textbf{Antiguo Testamento}}\newline}
\def\ntest{\rule{0pt}{10ex}{\large \textbf{Nuevo Testamento}}\newline}

\biblerefmap{Genesis}{1@\atest!01}
\biblerefmap{Exodus}{1@\atest!02}
\biblerefmap{Leviticus}{1@\atest!03}
\biblerefmap{Numbers}{1@\atest!04}
\biblerefmap{Deuteronomy}{1@\atest!05}
\biblerefmap{Joshua}{1@\atest!06}
\biblerefmap{Judges}{1@\atest!07}
\biblerefmap{Ruth}{1@\atest!08}
\biblerefmap{ISamuel}{1@\atest!09}
\biblerefmap{IISamuel}{1@\atest!10}
\biblerefmap{IKings}{1@\atest!11}
\biblerefmap{IIKings}{1@\atest!12}
\biblerefmap{IChronicles}{1@\atest!13}
\biblerefmap{IIChronicles}{1@\atest!14}
\biblerefmap{Ezra}{1@\atest!15}
\biblerefmap{Nehemiah}{1@\atest!16}
\biblerefmap{Esther}{1@\atest!17}
\biblerefmap{Job}{1@\atest!18}
\biblerefmap{Psalms}{1@\atest!19}
\biblerefmap{Proverbs}{1@\atest!20}
\biblerefmap{Ecclesiastes}{1@\atest!21}
\biblerefmap{Song of Solomon}{1@\atest!22}
\biblerefmap{Isaiah}{1@\atest!23}
\biblerefmap{Jeremiah}{1@\atest!24}
\biblerefmap{Lamentations}{1@\atest!25}
\biblerefmap{Ezekiel}{1@\atest!26}
\biblerefmap{Daniel}{1@\atest!27}
\biblerefmap{Hosea}{1@\atest!28}
\biblerefmap{Joel}{1@\atest!29}
\biblerefmap{Amos}{1@\atest!30}
\biblerefmap{Obadiah}{1@\atest!31}
\biblerefmap{Jonah}{1@\atest!32}
\biblerefmap{Micah}{1@\atest!33}
\biblerefmap{Nahum}{1@\atest!34}
\biblerefmap{Habakkuk}{1@\atest!35}
\biblerefmap{Zephaniah}{1@\atest!36}
\biblerefmap{Haggai}{1@\atest!37}
\biblerefmap{Zachariah}{1@\atest!38}
\biblerefmap{Malachi}{1@\atest!39}

\biblerefmap{Matthew}{2@\ntest!01}
\biblerefmap{Mark}{2@\ntest!02}
\biblerefmap{Luke}{2@\ntest!03}
\biblerefmap{John}{2@\ntest!04}
\biblerefmap{Acts}{2@\ntest!05}
\biblerefmap{Romans}{2@\ntest!06}
\biblerefmap{ICorinthians}{2@\ntest!07}
\biblerefmap{IICorinthians}{2@\ntest!08}
\biblerefmap{Galatians}{2@\ntest!09}
\biblerefmap{Ephesians}{2@\ntest!10}
\biblerefmap{Philippians}{2@\ntest!11}
\biblerefmap{Colossians}{2@\ntest!12}
\biblerefmap{IThessalonians}{2@\ntest!13}
\biblerefmap{IIThessalonians}{2@\ntest!14}
\biblerefmap{ITimothy}{2@\ntest!15}
\biblerefmap{IITimothy}{2@\ntest!16}
\biblerefmap{Titus}{2@\ntest!17}
\biblerefmap{Philemon}{2@\ntest!18}
\biblerefmap{Hebrews}{2@\ntest!19}
\biblerefmap{James}{2@\ntest!20}
\biblerefmap{IPeter}{2@\ntest!21}
\biblerefmap{IIPeter}{2@\ntest!22}
\biblerefmap{IJohn}{2@\ntest!23}
\biblerefmap{IIJohn}{2@\ntest!24}
\biblerefmap{IIIJohn}{2@\ntest!25}
\biblerefmap{Jude}{2@\ntest!26}
\biblerefmap{Revelation}{2@\ntest!27}

\makeindex[title=\'Indice de Escrituras,name=scr,options= -s style.ist]
%\makeindex[title=\'Indice General]
\renewcommand{\biblerefindex}{\index[scr]}



%headings
\definecolor{gray75}{gray}{0.75}
\newcommand{\hsp}{\hspace{20pt}} 
\titleformat{\chapter}[hang]{\normalfont\sffamily\LARGE\bfseries}{\thechapter\hsp\textcolor{gray75}{|}\hsp}{0pt}{\normalfont\sffamily\LARGE\bfseries}
\titleformat{\section}{\normalfont\sffamily\Large\bfseries}{\thesection}{1em}{}

%eliminate widows and orphans
\widowpenalty=10000
\clubpenalty=10000

%eliminate section numbering
\setcounter{secnumdepth}{0}

%title	
%% publisher’s logo
\providecommand{\HUGE}{\Huge}
\newlength{\drop}
\newcommand*{\titleGM}{\begingroup% Gentle Madness
\drop = 0.1\textheight
\vspace{\baselineskip}
\vfill
	\hbox{%
	\hspace*{0.2\textwidth}%
	\rule{1pt}{\textheight}
	\hspace*{0.05\textwidth}%
	\parbox[b]{0.75\textwidth}{
	\vbox{%
		\vspace{\drop}
		{\noindent\HUGE\bfseries Dos\\[0.5\baselineskip]
			Hombres}\\[2\baselineskip]
		{\Large\itshape Artículos Prácticos Sobre La Vida Cristiana}\\[4\baselineskip]
		{\Large BILL HALL}\\[2\baselineskip]\par
		{\noindent Traducido por Caleb George}\par
		\vspace{0.4\textheight}
		{\textbf{Kephali Press}}\\[\baselineskip]
		}% end of vbox
		}% end of parbox
	}% end of hbox
\vfill
\endgroup}



%***********************************************************************************************************************

\begin{document}
\pagestyle{empty}
\titleGM
\frontmatter
\pagestyle{headings}
\tableofcontents
\clearpage
\thispagestyle{empty}
\mainmatter

\chapter{CONSTRASTES/SALVACIÓN}
\pagestyle{headings}
\section{Dos Hombres No Pueden Estar De Acuerdo En La Religión}
Dos hombres son religiosos, y ambos parecen ser sinceros, pero parece que nunca están de acuerdo en cuestiones religiosas. Un vistazo breve a sus aplicaciones de la Biblia ayuda a explicar su problema.

\textbf{El primer hombre} ve la biblia como la respuesta completa y final a toda cuestión religiosa que pertenece a la salvación. Para él una declaración clara de la biblia termina toda controversia. Su enfoque simple se expone bien por una pegatina para parachoques que hemos visto: «Dios lo dijo; yo lo creo; eso lo resuelve». De hecho, él estaría de acuerdo en que la palabra de Dios «lo resuelve» si lo cree él o no. 

\textbf{El segundo hombre} confía en varias fuentes para sus creencias religiosas. El cree la biblia y muchas de sus convicciones se basan en lo que la biblia dice. Pero él también está convencido de que él ha sido «guiado» a ciertas creencias por el Señor. Unas de esas creencias él no las podría defender con la biblia, y de hecho unas de ellas parecen contradecir la biblia, pero él está seguro de que son verdaderas, porque el Señor no lo hubiera «guiado» así si no fueran verdaderas. Un hombre dijo francamente a este escritor, «Yo leo la biblia, pero más dependo del Espíritu para guiarme en lo que yo creo». Él también ha tenido sus creencias verificadas por predicadores informados, quienes podrían equivocarse en unas cosas, pero apenas se equivocarían en cualquier cuestión seria de la verdad.

\textbf{El primer hombre} lee la biblia buscando respuestas de Dios. \textbf{El segundo hombre} lee la biblia por la misma razón, pero su percepción se afecta por lo que él ha sido «guiado a creer». Él tiene dificultad con ser objetivo, y, de hecho, él podría aferrarse a lo que el ha sido «guiado a creer» antes de la sencilla enseñanza de las escrituras. Él francamente encuentra su semejanza en el hombre de \ibibleverse{Colossians}(2:18-19) quien se basa «en las visiones que ha visto» y «no asiéndose a la Cabeza». Sus practicas religiosas pueden ser diferentes de las del hombre descrito en Colosenses, pero su enfoque de autoridad en religión es igual.

Nosotros no dudamos en tomar el lado del primer hombre en su enfoque. Dios verdaderamente nos «guía» a la verdad, pero Él lo hace por medio de su palabra inspirada. Considera los pasajes siguientes: «Lámpara es a mis pies tu palabra, y luz para mi camino» (\ibibleverse{Psalms}(119:105)). «Santifícalos en la verdad; tu palabra es verdad» (\ibibleverse{John}(17:17)). «Toda Escritura es inspirada por Dios y útil para enseñar, para reprender, para corregir, para instruir en justicia, a fin de que el hombre de Dios sea perfecto, equipado para toda buena obra» (\ibibleverse{IITimothy}(3:16-17)).

Los dos hombres de nuestro articulo llegaran a un acuerdo notable cuando ambos se acerquen a las escrituras como la palabra final de Dios, pero no antes. Diferencias religiosas no son el producto de las escrituras, sino de varias actitudes hacia que constituye la autoridad final en la religión. Unidad verdadera es deseable, y se puede disfrutar por los que humildemente se someten a la palabra y voluntad de Dios. 

\section{Dos Hombres «Saben» Que Son Salvos}
Dos hombres «saben» que son salvos. 

\textbf{El primer hombre} basa su seguridad de salvación en su experiencia. Él se había asegurado de que cuando él completamente entregara su vida a Jesucristo, lo aceptara como su Salvador personal, y lo recibiera en su corazón, entonces una paz interior y un sentimiento de bienestar arrasaría su alma; que él podría reconocer ese sentimiento cuando viniera; y que su paz y sentimiento de bienestar serian una prueba inconfundible de su salvación. De acuerdo con esa enseñanza el sí «recibió a Jesús en su corazón». Él sí sintió esta paz maravillosa arrasar su alma. Y él «sabe» que es salvo. 

La seguridad \textbf{del segundo} hombre se basa en la promesa de Dios. Él había leído en la palabra de Dios, «Él que crea y sea bautizado será salvo; pero él que no crea será condenado» (\ibibleverse{Mark}(16:16)). De otros pasajes él había aprendido la necesidad de arrepentimiento y confesión de fe (\ibibleverse{Acts}(2:38); \ibiblechvs{Acts}(8:37); \ibibleverse{Romans}(10:10)). Confiando en la promesa de salvación de Dios, él obedeció del corazón esos mandamientos (\ibibleverse{Romans}(6:17)), y jamás ha dudado desde aquel entonces que Dios le perdono de todos sus pecados pasados, según Su promesa.

La seguridad de salvación \textbf{del primer hombre} se basa sobre un fundamento inseguro. La biblia enseña ni la acción que él ha tomado ni el criterio que él ha aceptado. Nosotros no cuestionamos sus sentimientos; pero si cuestionamos que tales sentimientos sean prueba de salvación, porque son el producto de su enseñanza. El cultista quien se ha enseñado ciertas practicas repugnantes en la religión y que se ha convertido la victima de su líder pervertido experimentara sentimientos parecidos. Si los sentimientos de este ultimo no son prueba suficiente de salvación, tampoco la son los sentimientos del primero.

«Pero un ángel apareció y me habló», alguien puede estar pensando; o, «Yo hablé en lenguas». La biblia enseña, sin embargo, que incluso ocurrencias inusuales e inexplicables en la vida de uno no pueden dejar de lado a la enseñanza de la palabra revelada de Dios. Muchos de los que estarán en el infierno habrán dicho, «Señor, Señor, ¿no profetizamos en tu nombre, y en tu nombre echamos fuera demonios, y en tu nombre hicimos muchos milagros?» (\ibibleverse{Matthew}(7:22)). Pero todas sus experiencias, incluso «milagros», no sustituirán por su hacer la voluntad del Padre como est\'a registrada en las escrituras (\ibibleverse{Matthew}(7:21)); ve también \ibibleverse{Galatians}(1:8); \ibibleverse{IIThessalonians}(2:9-12); \ibibleverse{Deuteronomy}(13:1-5); \ibibleverse{Colossians}(2:18-19).

\textbf{El segundo hombre} ha basado su seguridad en un fundamento sólido. Las promesas de Dios son seguras. Él no puede mentir (\ibibleverse{Hebrews}(6:18)). Lo que Él ha prometido Él es capaz de cumplir (\ibibleverse{Romans}(4:21)). La persona que obedece sus mandamientos por fe en sus promesas puede saber, porque Dios es fiel. «Y en esto sabemos que hemos llegado a conocerle: si guardamos sus mandamientos» (\ibibleverse{IJohn}(2:3)).

Preguntamos a nuestros lectores, «¿Quién realmente demuestra una fe fuerte en Dios: él que simplemente confía en las promesas de Dios y encuentra seguridad en Su palabra o él que tiene que experimentar algún sentimiento arrollador que arrase su alma?» El juicio de Dios será basado, no en lo que nosotros «sabemos», sino en Su palabra. ¡No se dejen engañar!

\section{Dos Hombres Erran Con Respecto A La Gracia}
Dos hombres erran con respecto a la gracia. \textbf{El primer hombre} predica la gracia, pero no reconoce que la gracia de Dios se vincula a la responsabilidad humana. \textbf{El segundo hombre} predica responsabilidad, pero raramente habla de la gracia de Dios. 

\textbf{El primer hombre} cree que la salvación es solo por la gracia de Dios. El afirma que cualquier acción obligatoria de la parte del hombre en obediencia a mandamientos anularía la gracia y constituiría salvación meritoria. «Es absurdo creer que la gracia de Dios podría vincularse a cualquier cosa como bautismo», es la manera con que una persona lo declaró. 

\textbf{El segundo hombre} habla bien de los requisitos del evangelio. El a menudo predica la necesidad del bautismo, asistencia fiel, contribuciones liberales, buenos morales, etc. El habla de Jesús como nuestro ejemplo perfecto y de su sumisión completa al Padre en su muerte, pero raramente de El cómo la propiciación por nuestros pecados. Pocas veces el lleva sus oyentes a sentir su necesidad constante de la misericordia y perdón de Dios y su impotencia y desesperanza absoluta aparte de la sangre limpiadora de Cristo. 

\textbf{El primer hombre} prometería salvación sin la diligencia necesaria en aprender y hacer la voluntad de Dios. \textbf{El segundo hombre} pondría tanto énfasis en aprender y hacer la voluntad de Dios que el enfocaría los ojos de sus oyentes más en ellos mismos que en el Señor. \textbf{El primer hombre} necesita aprender la verdad de \ibibleverse{Titus}(2:11-12): «Porque la gracia de Dios se ha manifestado, trayendo salvación a todos los hombres, enseñándonos, que negando la impiedad y los deseos mundanos, vivamos en este mundo sobria, justa y piadosamente». \textbf{El segundo hombre} necesita aprender y valorar la exhortación de \ibibleverse{Philippians}(3:1): «Por lo demás, hermanos míos, regocijaos en el Señor».

Recordaríamos \textbf{al primer hombre} de la naturaleza de la gracia de Dios como esta revelada por los siglos. Comenzaríamos con la gracia de Dios como fue extendida a Noe en el tiempo del diluvio. «Noé halló gracia ante los ojos del Señor» (\ibibleverse{Genesis}(6:8)). Noe, sin embargo, se dio instrucciones para obedecer. Y Noe reconoció la necesidad de obediencia: «Y así lo hizo Noé; conforme a todo lo que Dios le había mandado, así hizo» (\ibibleverse{Genesis}(6:22)). Si Noe hubiera fallado en sus responsabilidades, el jamás hubiera sido salvado del diluvio por la gracia de Dios. Recordaríamos a este hombre de la gracia de Dios como fue extendida a Josué en la captura de Jericó. «Mira, he entregado en tu mano a Jericó» (\ibibleverse{Joshua}(6:2)). Pero Dios tuvo instrucciones para Josué: marchar, tocar las trompetas, gritar. Cuando Josué y los israelitas cumplieron su responsabilidad «la muralla se vino abajo, y el pueblo subió a la ciudad» (\ibibleverse{Joshua}(6:20)). Recordaríamos a este hombre del ciego de Juan 9 cuyos ojos el Señor abrió (\ibibleverse{John}(9:14, 17, 21, 26, 30)) cuando el hizo lo que el Señor mando. Nuestro primer hombre debería poder ver que (1) la gracia de Dios no descarta instrucciones (ley); (2) la gracia de Dios no descarta obediencia; y (3) la gracia de Dios no descarta obediencia estricta. S

Recordaríamos \textbf{al segundo hombre} que buenas obras sin la gracia de Dios jamás pueden salvar. Comenzaríamos con el mensaje de Efesios. Pablo en Efesios ciertamente dio instrucciones – instrucciones prácticas, instrucciones que deben obedecerse, sobre morales, deberes de esposas, esposos, hijos, padres, siervos, amos – pero no hasta que hubo establecido firmemente la gracia de Dios como la base de salvación (capítulos 1-3\ibible{Ephesians}(1-3:)) y como la motivación de obediencia a las instrucciones de Dios (observe la palabra «pues» en 4.1). Recordaríamos a este hombre del peligro de ser como los fariseos que «confiaban en sí mismos como justos, y despreciaban a los demás» (\ibibleverse{Luke}(18:9-14)). Le recordaríamos que cuando uno peca él no tiene «nada con que pagar» y entonces debe acercarse a Dios como uno que es pobre en espíritu, llorando, humilde, con hambre y sed de justicia (\ibibleverse{Luke}(7:41-42); \bibleverse{Matthew}(5:3-6)). 

Nosotros no nos atreveríamos a decir cuál de estos maestros es el más peligroso, porque ambos erran con respecto a la gracia. Nos encontramos naturalmente retrocediendo de la enseñanza del primer hombre y temiendo mucho las consecuencias de su enseñanza, pero jamás queremos ser culpables del error del segundo. No podemos predicar gracia sin predicar responsabilidad, pero debemos evitar ser culpables de predicar responsabilidad sin predicar la gracia. 

\section{Dos Hombres Buscan Fe}
Dos hombres buscan fe. El enfoque \textbf{del primer hombre} es intentar encontrar soluciones a todos los problemas. Él ha cavado profundamente en las cuestiones difíciles relacionadas a la cuenta de creación en Genesis. Él ha leído volúmenes sobre el diluvio. Él encuentra la historia de Jonás particularmente desafiante. Él confía mucho en la archivología e historia secular para confirmación de sus soluciones. Él cree porque está satisfecho con sus propias respuestas a los problemas de fe. 

El enfoque \textbf{del segundo hombre} se centra en Jesucristo. Él también ha tenido que considerar pruebas y luchar con ciertos problemas, pero él esta completamente persuadido que Jesucristo es el Hijo de Dios y ha confesado que él cree ese hecho con todo su corazón. Creyendo en Jesús como el Hijo infalible de Dios, él no cuestiona nada de lo que Jesús creó, nada de lo que Jesús dijo, ni nada de lo que Jesús autorizó decirse. 

Él, también, se preguntaba acerca de la cuenta de creación en Genesis, pero sus preguntas cesaron cuando él leyó las palabras de Jesús: «¿No habéis leído que aquel que los creó, desde el principio los hizo varón y hembra, y añadió: “Por esta razón el hombre dejará a su padre y a su madre y se unirá a su mujer, y los dos serán una sola carne”?» (\ibibleverse{Matthew}(19:4-5)). Si el sello de aprobación de Jesús estuvo en la historia de creación, eso bastó para él. Él no tenía la solución a todos los problemas, pero él creó \textbf{porque} Jesús lo creó, y él creó en Jesús. 

Su fe en otras cuentas del Antiguo Testamento se estableció de igual modo. Él encontró el sello de aprobación de Jesús en el diluvio (\ibibleverse{Matthew}(24:37-39)) y la historia de Jonás (\ibibleverse{Matthew}(12:40)), y de hecho, en todo el registro del Antiguo Testamento en una declaración amplia registrada en (\bibleverse{Luke}(24:44)): «Esto es lo que yo os decía cuando todavía estaba con vosotros: que era necesario que se cumpliera todo lo que sobre mí está escrito en la ley de Moisés, en los profetas y en los salmos».

Como el Nuevo Testamento contiene lo que Jesús dijo combinado con lo que Él autorizó decirse, el segundo hombre no tuvo ningún problema en creer el Nuevo Testamento. Su fe en la biblia entera simplemente se basa en su fe en Cristo como el Hijo infalible de Dios. Él también encuentra el estudio de los problemas de fe intrigante y retador, pero su propia fe personal no depende de encontrar soluciones a todos los problemas. 

La fe \textbf{del primer hombre} descansa en terreno inestable, porque se funda en sabiduría humana. Si después algún descubrimiento arqueológico u otra prueba no conocida ahora compruebe que sus soluciones son falsas, el mero fundamento de su fe ya no estaría. Él tendría que buscar nuevas soluciones o perder su fe enteramente. 

Puede ser que la fe \textbf{del segundo hombre} no atraiga a los altamente sofisticados, pero se funda en la piedra – en Él que es «el mismo ayer y hoy y por los siglos» (\ibibleverse{Hebrews}(13:8)).

 «Así que la fe viene del oír, y el oír, por la palabra de Cristo» (\ibibleverse{Romans}(10:17)).

\chapter{CONTRASTES/\\ VIDA CRISTIANA}

\section{Dos Hombres Sirven Al Señor}
Dos hombres sirven al Señor. \textbf{El primero} se motiva por su amor por la iglesia local como un instituto. Él apoya con entusiasmo su programa de actividades. Él trabaja diligentemente para que otros se bauticen, porque él está interesado en esta organización de la cual el es una parte, y el quiere verla crecer. Él asiste a todas las reuniones, da liberalmente de su dinero, y realmente hace su parte en apoyar la iglesia local. 

\textbf{El otro} se motiva por su amor por el Señor. Él, también, trabaja para convertir otros, pero lo hace porque el esta preocupado por sus almas. Él, también, asiste a todas las reuniones, porque en hacerlo él se acerca al Señor a quien ama, y tiene oportunidad de glorificar su nombre. Él también da liberalmente de su dinero, porque él ama al Señor y está interesado en hacer su parte en financiar Su obra. Él ama la iglesia y se regocija cuando ella crece, pero su amor llega mucho más profundamente que el del primer hombre; de hecho, él serviría al Señor si no hubiera ningún otro cristiano en la tierra ni posibilidad de que hubiera otro. 

El entusiasmo \textbf{del primer hombre} depende en gran manera de otros. Mientras la congregación este creciendo y activa, su entusiasmo sigue fuerte; pero cuando surgen problemas, o su predicador favorito se va, o unos de sus hermanos no hacen su parte, o alguien lo critica, o la congregación simplemente en general tiene que enfrentar un periodo difícil, su entusiasmo comienza a enflaquecerse, y él se vuelve “infiel.” 

\textbf{El segundo hombre} es estable y no vacilante. Hermanos vienen y van; la congregación de la cual el es una parte tiene sus periodos de depresión; problemas surgen de vez en cuando; pero el entusiasmo de este hombre sigue constante a través de todo, porque está centrado en Él que nunca cambia, Él que ha prometido, «Nunca te dejaré ni te desampararé». 

Tengamos cuidado, entonces, como construimos (\ibibleverse{ICorinthians}(3:10)). Tal vez el titulo de este articulo debería decir «Un Hombre Sirve Al Señor», porque es muy dudoso que el primer hombre sirva al Señor en absoluto. 

\section{Dos Hombres Ven El Rigor}
Dos hombres determinan obedecer las escrituras rigorosamente. Ninguno quiere volver «ni a la derecha ni a la izquierda», incluso en el asunto más pequeño. Pero las actitudes detrás de la determinación de estos dos hombres varían enormemente. 

\textbf{El primer hombre} ve el rigor en obedecer la ley de Dios como el medio principal de salvación. Mientras él habla académicamente de la gracia de Dios, prácticamente él piensa poco en su necesidad continua de la gracia de Dios. Él determina asistir a toda asamblea, dar liberalmente, mantener control de su familia; en pocas palabras, él realmente «va a vivir su religión».  Mientras predica que un hombre no se puede salvar por obras de mérito, él busca inconscientemente hacer la mera cosa que el predica que uno no puede hacer. 

\textbf{El segundo hombre} reconoce la gracia de Dios como el medio principal de salvación. Él habla frecuentemente de misericordia divina que podría extenderse incluso a él mismo. Él es igual de riguroso en obedecer la palabra de Dios como el primer hombre. Él, también, asiste a toda asamblea, da liberalmente, etc., pero su obediencia cuidadosa a todo mandamiento es una manifestación de su amor por Dios (\ibibleverse{John}(14:15)), su fe en las promesas de Dios (\ibibleverse{James}(2:20)), y su reconocimiento de que la gracia de Dios solo se extiende a los que lo obedecen (\ibibleverse{Matthew}(7:21)). Sabiendo cuan corto estará de cumplir la perfección de Dios, sin embargo, él determina volver constantemente a Dios por perdón y misericordia. Su única jactancia será la de la cruz. 

Una de dos cosas pasara \textbf{al primer hombre}. O él se convencerá de que él realmente esta «viviendo su religión» – que él realmente está logrando estar a la altura de ese estándar que él se ha puesto – en cuyo caso él, en su justicia propia, despreciara a los débiles esforzándose que no pueden estar a la altura de su supuesto nivel de justicia («Refirió también esta parábola a unos que confiaban en sí mismos como justos, y despreciaban a los demás» \ibibleverse{Luke}(18:9); o, él volverá desanimado reconociendo honestamente sus fallas, y levantara las manos en desespera. «Yo intenté», él dirá, «pero simplemente no pude vivirlo». De cualquier manera, el hombre esta condenado, porque su pensar es incorrecto.

\textbf{El segundo hombre} encontrara gozo verdadero en el Señor y paz que sobrepasa todo entendimiento. Él vivirá en esperanza constante, una esperanza no construida sobre su propia perfección, sino sobre la seguridad del perdón de Dios. Él será paciente con los débiles e inmaduros, porque él será muy consciente de su propia indignidad ante Dios. Él no buscará ponerse alguna fachada hipócrita, porque él nunca pretenderá ser más que un pecador salvado por gracia. 

Nuestros dos hombres parecerán muy similares en el exterior en su rigor hacia las escrituras (ambos probablemente se llamarán «legalistas»), pero habrá una diferencia notable en el interior, y estamos bien seguros de que habrá una diferencia considerable en la dirección final y el resultado de sus vidas. 

\section{Dos Hombres Reaccionan A Enseñanza Sobre Morales}
Dos hombres escuchan lecciones sobre la vida practica cristiana pero sus actitudes difieren grandemente. 

\textbf{El primer hombre} ve toda esa enseñanza como reglas arbitrarias de la iglesia. Advertencias con respecto a baile, natación mixta, inmodestia general, divorcio por toda causa, la bebida, etc., son todas vistas como estándares de la «Iglesia de Cristo», tradiciones apoyadas por los viejitos de la iglesia que son desentendidos con respecto a la forma moderna de pensar sobre morales. 

\textbf{El segundo hombre} se lleva a entender que toda esa enseñanza brota de respeto genuino por la biblia; que advertencias sobre las maldades mencionadas anteriormente se basan en tales escrituras como \ibibleverse{Matthew}(5:27-28); \ibibleverse{Galatians}(5:19-21); \ibibleverse{ITimothy}(2:9); \ibibleverse{Matthew}(19:9); y \ibibleverse{Romans}(13:12-14); que ellas, entonces, no son reglas arbitrarias de la iglesia, sino son ciertamente una verdadera imagen de la voluntad de Dios para su pueblo. 

\textbf{El primer hombre} odia esta enseñanza. ¡Por supuesto que sí! ¿No tiene él tanta percepción moral como otro? Para que dejaría él que algún otro hombre decidiera lo que sea bueno o malo para él? El hará lo que le guste. Nadie va a atar sus pensamientos sobre él. 

\textbf{El segundo hombre}, reconociendo que los estándares bajo consideración son de Dios y no de los hombres, felizmente cumple. Jesucristo es su Señor y Rey. Él vivirá cualquier vida que su Señor quiere que viva. Él hará cualquier sacrificio que su Señor quiere que haga. Su conformidad crece de un deseo de complacer a Dios, no al hombre. 

\textbf{El primer hombre} puede tomar la forma de un adolescente rebelando contra la autoridad de sus padres; o la forma de un hombre «criado en la iglesia», cuya lealtad a la iglesia esta comenzando a enflaquecer, o la forma del nuevo convertido que esta teniendo dificultad en definir modestia, decencia, y lascivia en términos prácticos. La falla puede radicar dentro de la persona misma. Él puede desear separarse de toda restricción, así negando ver objetivamente principios bíblicos detrás de la enseñanza que el está rechazando. O la falla en unos casos puede radicar en los que enseñan. Ellos pueden ser culpables de «establecer» sus puntos a lo largo de estas líneas por golpear el pulpito y estampar los pies, en lugar de razonamiento solido de las escrituras; de buscar inconscientemente lealtad a la iglesia o lealtad al predicador en lugar de lealtad al Señor. De cualquier manera, estamos preocupados por el alma de nuestro primer hombre, porque está equivocado en su pensar.

Que ninguno malinterprete. Nos oponemos fuertemente a toda maldad mencionada anteriormente. Pero la verdad es – ningún hombre está obligado a inclinarse a cualquier cosa que \textbf{nosotros} enseñamos porque \textbf{nosotros} lo enseñamos; pero, por otro lado, él si esta obligado a vivir por todo principio que está verdaderamente establecido sobre la palabra de Dios. Es el deber de todo maestro, entonces, advertir de estas maldades, pero en la basis de la autoridad de Dios. Es el deber de todo oyente considerarlas en la luz de las escrituras. Una mayor consciencia de Dios – de parte de igual maestro como oyente – es la necesidad. 

\section{Tres Hombres Rebelan Contra La Hipocresía}
Tres hombres rebelan contra la hipocresía, pero varían enormemente en sus reacciones. \textbf{El primer hombre} recurre al abandono moral total. Él arroja toda restricción mientras se entrega al cumplimiento de todo deseo carnal. «El yo» se convierte en su dios. Él se endurece a las lagrimas de su familia mientras él sale a hacer lo que él quiere hacer. Su «justificación» por su conducta vergonzosa: «Por lo menos no soy hipócrita!».

\textbf{El segundo} hombre va a otro extremo. Él esta harto de la debilidad e hipocresía que él ve en todas las iglesias, y él no va a ser como tales personas. Él se convertirá en un cristiano y desde el principio «él lo va a vivir». Él será un ejemplo de lo que un cristiano realmente debe ser. Para él, la cura de la hipocresía es la perfección. 

\textbf{El tercer hombre} quiere evitar la hipocresía en su vida, pero a la vez, él tiene un sentido profundo de su propia imperfección. Así que él no presenta ningún aire de infalibilidad, sino busca ser genuino. Su genuinidad pronto se nota por otros. Él no afirma ser perfecto, pero él lucha por la perfección. Mientras él adora a Dios, él no afirma ser perfecto como adorador, pero cuando comienzan a cantar él entrega su corazón a lo que esta haciendo; cuando se dirige la oración, él escucha y hace la oración su oración; durante la cena el medita en el sufrimiento de su Señor; y por todo el sermón él participa en un estudio de la palabra de Dios; si su mente vaga, él la vuelve a enfocar; y cuando el periodo de adoración termina él le pide a Dios perdonar sus fallas y aceptar su adoración a pesar de su imperfección. 

Cuando él va a su trabajo, él no afirma la perfección entre sus compañeros de trabajo, pero ellos saben que él intentará dar ocho horas de trabajo por ocho horas de pago; que él es confiable; que él es puro en su forma de hablar y vida; y que, si él se abruma por presiones a su alrededor a pecar, él humildemente buscará el perdón de los que han sido dañados. 

Él es el mismo en su casa. Su familia lo respeta porque él es genuino y no afirma tener fuerza y bondad más allá de la realidad. Su familia ve sus fallas, pero su única cualidad redentora que le permite mantener su respeto es su habilidad de decir, «Lo siento». En toda área de su vida, él anda humildemente ante su Dios y su prójimo. 

\textbf{Nuestro tercer hombre} ha encontrado la verdadera cura de la hipocresía. \textbf{El primer hombre}, si no se arrepiente, será algún día un desgraciado miserable, su vida completamente desgarrado y destrozado. \textbf{El segundo hombre} va rumbo a la desilusión. Sus metas no son realistas; su visión es totalmente incorrecta. Pero el hombre que «camina humildemente con su Dios» y es completamente libre de engaño es un hombre verdaderamente bendecido. Él es en vida y actitud lo que Dios quiere que él sea, y él vive en la esperanza de los cielos. 

«Bienaventurados los pobres en espíritu, pues de ellos es el reino de los cielos» (\ibibleverse{Matthew}(5:3)).

\chapter{CONTRASTES/IGLESIA}

\section{Dos Hombres Asisten Al Servicio De Adoración}
Dos hombres asisten al servicio de adoración. \textbf{El primer hombre} asiste enteramente por un sentido de deber. Él entiende la enseñanza de \ibibleverse{Hebrews}(10:25): «No dejando de congregarnos», y él está determinado obedecer fielmente esa enseñanza. Él no permitirá que nada dentro de su poder sea un obstáculo a su asistir a los periodos de adoración de la iglesia. 

\textbf{El segundo hombre} reconoce su deber en este asunto también, pero su motivación principal en asistir es su amor por el Señor y su gozo en mezclar su voz y corazón con otros cristianos en loor y adoración al Señor. Él se deleita en adoración y en la fuerza espiritual que él deriva por medio de adoración. 

\textbf{El primer hombre} esta mentalmente pasivo en todo el servicio de adoración. Si por casualidad las palabras del canto le llaman la atención, él observa y las aprecia; si no, él simplemente canta, pensando poco en lo que canta. Si el sermón es interesante, él escucha; si no, él simplemente se relaja, y espera que el tiempo no se arrastre demasiado. Él sí medita brevemente con respecto al sufrimiento y la muerte de Cristo mientras participa de la cena, porque de alguna forma la importancia de la fiesta memorial se ha impresionado en su mente. 

\textbf{El segundo hombre} viene mentalmente preparado para adorar. Él presta mucha atención a las palabras de cada canto y hace el sentimiento de los cantos su propio sentimiento. De hecho, él a veces estudia las palabras de cantos frecuentemente usados para que esté seguro de que él entienda su significado. Profundidad de significado es de mayor importancia para él que una melodía pegadiza o un ritmo cadencioso. Él escucha a cada frase de la oración que se dirige, y si él puede aprobar las peticiones de la oración, él se une con él que dirige con su «Amen». Él discierne el cuerpo del Señor mientras parte el pan, y él escucha cuidadosamente al sermón, voluntariamente prestando atención, atesorando la palabra en su corazón, para no pecar contra Dios (\ibibleverse{Psalms}(119:11)). Si su mente vaga de vez en cuando, él la vuelve a la adoración. Él adora con una consciencia de Dios como el objeto de su adoración, Él hacía quien estas expresiones de adoración se dirigen. 

\textbf{El primer hombre} reduce su servicio a un mero código de ritos externos, mientras \textbf{el segundo hombre} obedece «del corazón», combinando el externo con el interno. Es más probable que \textbf{el primer hombre} se satisface con su servicio al Señor, porque él ha aceptado un estándar más fácil, pero es \textbf{el segundo hombre} que disfruta de la aprobación de Dios. «Dios es espíritu, y los que le adoran deben adorarle en espíritu y en verdad» (\ibibleverse{John}(4:24)). 

Preguntamos al lector: «¿En cuál de estos dos hombres ves un reflejo de tu mismo?». ¡La necesidad es obvia! Debemos descartar nuestra pereza e indiferencia, revivificar nuestros espíritus, y volver a adorarle a Dios aceptablemente. Hay una diferencia considerable entre mera asistencia a un servicio de adoración y adoración que es verdaderamente aceptable. 

\section{Dos Hombres Intentan Adorar}
Adoración bajo las mejores de condiciones a veces puede ser difícil. Distracciones, errores humanos, y a veces situaciones graciosas pueden ocurrir para quitarle a uno la atención al Señor. Actitudes, sin embargo, pueden demostrar ser un factor importante en adoración aceptable (o no aceptable). Por ejemplo\ldots

Dos hombres sinceramente intentan adorar. Pero \textbf{el primer hombre} está frustrado por todo. Sus frustraciones comienzan con los primeros anuncios cuando el hombre a carga toma diez minutos para decir lo que cualquier hombre normal podría decir en tres. Él apenas termina cuando el líder de cantos aumenta su frustración, escogiendo un canto él es seguro contiene una frase no bíblica. El hombre que preside en la mesa no ayuda cuando él usa el término «barra» en vez de «pan» y después el hombre que se pide «dar gracias por el pan» da gracias por todo sino el pan. El predicador hace una gran contribución por totalmente aplicar mal un pasaje de escritura («A lo mejor él no pasó suficiente tiempo en ese pasaje», el hombre piensa). Cuando el periodo de adoración por fin se despide, él intenta compartir sus frustraciones con los a su alrededor, pero parece que a nadie le importa. 

\textbf{El segundo hombre} observa muchos de los errores que el primer hombre observa. De hecho, sin fanfarria él simplemente no canta la frase cuestionable en el canto y él le agradece en silencio a Dios por el pan cuando él reconoce la falla del líder en hacerlo. Pero mientras observa los errores, él se enfoca en los buenos sentimientos de los cantos que se usan y en la muerte de su Salvador durante la Cena del Señor. Él hace la oración que se dirige su propia y aprecia los buenos pensamientos presentados en la lección. Él ha venido a adorarle a Dios. Él hace concesiones por fragilidad humana de parte de los líderes en adoración, aprecia sus esfuerzos sinceros, y niega dejar que sus errores le alejen de su propósito.  

\textbf{El primer hombre} debe ser compadecido. Su capacidad de «adorar» depende de la capacidad de los lideres en el periodo de adoración, y cualquier persona medio-observante sabe lo inepto el liderazgo puede ser a veces. Él viene a adorar, pero pasa la hora criticando. Él les echa la culpa a otros por lo que realmente es su propio problema. Como resultado, su problema con adoración se convierte en un problema con sus hermanos también; pero uno no puede tener un problema en su adoración y con sus hermanos sin tener un problema en su relación con Dios. 

\textbf{El segundo hombre}, por medio de mantener una actitud positiva hacia sus hermanos, incluso cuando hacen errores, es capaz de adorar aceptablemente y se acerca a Dios por su adoración. 

No estamos consintiendo a periodos de adoración que son conducidos descuidadamente. Líderes en adoración deberían buscar evitar errores y hacer su trabajo efectivamente. Pero adoración aceptable depende mucho más del corazón y la actitud del adorador que de las habilidades de los líderes. Nuestro primer hombre puede señalar a otros, culpándolos, pero su necesidad verdadera es un cambio total de actitud dentro de sí. 

\section{Dos Hombres Enfrentan Sus Limitaciones}
Dos hombres son algo inepto en áreas de liderazgo público. Ninguno de los dos tiene mucho talento en dirigir cantos, predicar, enseñar clases bíblicas, o cumplir otros roles que son tan esenciales a periodos de adoración efectiva. Ambos han intentado, pero sus incapacidades en tales esferas son aparentes, para ellos mismos y para otros. Pero mientras comparten esta limitación, sus actitudes difieren dramáticamente. 

\textbf{El primer hombre} se retira a su caparazón, manifestando todos los síntomas de un complejo de inferioridad. Él siente que él no pertenece, que otros no lo aprecian. Él no hace nada para el beneficio de la causa del Señor sino asistir. Él raramente visita a los enfermos o habla a un visitante o invita a un recién llegado a su casa. «No puedo» se convierte en la frase dominante de su vocabulario. Él se queja porque «solo unos pocos están dirigiendo las cosas».

\textbf{El segundo hombre}, reconociendo sus incapacidades obvias en roles de liderazgo, mira alrededor por otras áreas en que él puede ser útil. Él se ofrece a mantener cortada la grama alrededor del edificio y a abrir el edificio temprano en cada servicio. Él está allí para extender un saludo cálido a los primeros en llegar. Esto es simplemente típico de él. Él constantemente está observando una necesidad y trabajando en su propia manera callada para cuidar esa necesidad. Ningún otro hombre en la iglesia es más activo en la obra que él. 

\textbf{El primer hombre} lucha para ser fiel. Sus sentimientos se lastiman tan fácilmente. Cada lección que trata de mayor diligencia en el servicio del Señor se predica directamente con él en la mente, él está seguro. A si mismo no le gusta y su actitud es una barrera a buenas relaciones con otros. 

\textbf{El segundo hombre} se aprecia por todos los que le conocen. Su influencia es grande. Él es perfectamente capacitado para la obra de un diacono. Su muerte dejará un hueco en la iglesia que ningún solo hombre podrá llenar. 

La diferencia entre estos dos hombres se puede ver claramente en la exhortación de \ibibleverse{Ecclesiastes}(9:10): «Todo lo que tu mano halle para hacer, hazlo según tus fuerzas». Lo que el segundo hombre tiene y lo que le falta al primer hombre es \textbf{visión} para ver lo que se necesita hacer (su mano encuentra algo que hacer) e \textbf{iniciativa} para hacerlo con su fuerza. Estas dos calidades capacitan a uno a ser feliz, ocupado, útil, agradable e influyente; la falta de ellas deja al otro miserable, limitado, sensitivo, sofocado por autocompasión. 

A nuestros muchos lectores que son limitados en roles de «liderazgo» preguntaríamos: «¿Cuál de estos dos hombres presentan una verdadera imagen de ti?». ¡Abre tus ojos! ¡Hay necesidades por todos lados! ¡Ve a trabajar! «Hay espacio en el reino… por las cosas pequeñas que tú puedes hacer». No todos pueden ser el mariscal del campo; no todos pueden ser el jefe; pero todos pueden contribuir. Que cada persona encuentre su propio rol, trabaje diligentemente en ese rol, y regocije en la contribución que él puede hacer para el bienestar de la obra del Señor. 

\section{Dos Hombres Son Miembros De La Misma Iglesia}
Dos hombres son miembros de la misma iglesia local. Esa iglesia esta participando en una serie de predicaciones y es la noche del jueves. \textbf{El primer hombre} está presente. \textbf{El segundo hombre}, había salido esa mañana en un viaje de placer, habiendo asegurado el predicador visitante que el escucharía la cinta. 

\textbf{El primer hombre} adora a Dios ese jueves por la noche. Él está presente ante el trono de Dios. Él canta a Dios. Él alaba a Dios. Él se une con otros en oración. Él expresa agradecimiento a Dios. Él está en comunión con otros cristianos por esa hora, exhortándolos y alentándolos a ellos y siendo exhortado y alentado por ellos. Él adora. Él participa. Él está involucrado. Él glorifica a Dios. 

\textbf{El segundo hombre} escucha una cinta. 

\textbf{El primer hombre} demuestra que sus prioridades son correctas, que él está buscando «primero el reino de Dios y su justicia» (\ibibleverse{Matthew}(6:33)), que él ve el reino de los cielos como un tesoro escondido y como una perla de gran precio, algo que valorar sobre todo placer mundano o tesoro (Mateo 13.44-46). Él no ofrecerá al Señor «lo que no le cuesta nada» (\ibibleverse{IISamuel}(24:24)). Él está presente. Él está poniendo las primeras cosas en el primer lugar. 

\textbf{El segundo hombre} escucha una cinta. 

Las acciones \textbf{del primer hombre} alientan a otros. Él alienta al predicador, sentado con su biblia abierta, escuchando atentamente (solo un predicador podría entender lo alentador que un buen oyente puede ser), y después haciendo algún comentario perceptivo después de la lección. Él alienta al líder de los cantos, participando con animo en el cantar. Él alienta al inconverso, cantando el canto de invitación con fervor, esperando sinceramente por una respuesta. Él alienta a los visitantes, hablándoles, platicando por unos pocos minutes, haciendo que se sientan bienvenidos e invitándolos a volver. Él alienta a otros miembros de la congregación. Con simplemente asistir a todo servicio sin fallar él habla volúmenes de lo que fidelidad verdadera involucra. Él no ha hecho nada para ser visto. De hecho, mientras él vuelve a la casa él tiene poca consciencia de lo alentador que él ha sido para muchos a causa de su presencia. 

\textbf{El segundo hombre} escucha una cinta. 

\textbf{El primer hombre} es un hombre mejor por haber asistido. Su fe está más fuerte. Su determinación de vivir por el Señor está más fuerte. Él encuentra gozo en saber que él ha hecho la decisión correcta en como pasar esa noche. Él está más cerca a Dios y más cerca a los cielos. Él está mejor equipado para las luchas y estreses del día siguiente. 

El viaje de placer \textbf{del segundo hombre} no es tan placentero como él había anticipado. Él sabe que la serie de predicaciones sigue en la iglesia y que él debería estar allí. Él ha intentado acallar su consciencia por prometer escuchar la cinta, pero al fondo él sabe que Dios no está complacido con sus acciones. Él sí escucha la cinta, pero ¡que contraste entre sus acciones y las del primer hombre! Que contraste entre los efectos producidos por el periodo de adoración en el segundo hombre mientras escucha su cinta y los que se habían producido en el primero. 

No es nuestro propósito desacreditar el uso apropiado de cintas. Son maravillosas. Permiten a los que no pueden salir de sus casas a escuchar sermones que nunca podrían escuchar de otra manera. Permiten a predicadores a extender la influencia y enseñanza del evangelio más allá de los limites de una asamblea. Permiten a viajeros en sus carros a pasar provechosamente tiempo que de lo contrario hubiera sido malgastado. Ellas llegan a cristianos en lugares aislados que de lo contrario tienen poca oportunidad de escuchar buena enseñanza bíblica. Ellas incluso permiten a predicadores a criticar sus propias lecciones (Este escritor no solo ha criticado, sino también botado unas).

Adoración aceptable, sin embargo, se enfoque en Dios, no en un predicador. Adoración aceptable involucra participación: «Aclamad con júbilo al Señor» (\ibibleverse{Psalms}(100:1)). Hay un aspecto social de adoración aceptable: «enseñándoos y amonestándoos unos a otros…» (\ibibleverse{Colossians}(3:16)). Adoración aceptable involucra alabanza y el dar gracias y petición y expresiones de dependencia. Esto no se puede hacer por «audio». Nuestro segundo hombre escucha una cinta, y recibe alguna información del sermón, pero lo que él hace no puede substituir por adoración verdadera, ni puede justificar su ausencia de esa reunión el jueves por la noche. 

\section{Dos Iglesias Quieren Crecer}
Dos iglesias quieren crecer, pero sus actitudes hacia crecimiento difieren en gran manera.

\textbf{La primera iglesia} ve el crecimiento como su propósito principal. Metas se presentan ante la membresía: «Queremos doblar nuestra membresía dentro de los próximos tres años», por ejemplo. Éxito (o fracaso) se juzga casi enteramente en la basis del crecimiento numérico de esa congregación. 

\textbf{La segunda iglesia} ve el salvar almas como su propósito principal y cualquier crecimiento de la membresía es simplemente el resultado natural de ese propósito principal. Miembros de la segunda iglesia se infunden con el valor de almas inmortales en vez de un sentido de orgullo congregacional. 

Miembros de \textbf{la primera iglesia} vuelven ansiosos de llevar la gente al agua. El bautismo es el punto en que la gente se agrega a la lista de membresía; como consecuencia, va a requerir tantos bautismos para mantener el ritmo de su meta de doblar su membresía. No solo tienen que llevarlos al agua, también tienen que llevarlos allí dentro del periodo de tiempo que se ha fijado arbitrariamente por sus líderes. 

Miembros de \textbf{la segunda iglesia} están mucho más ansiosos de llevar la gente al arrepentimiento. Su preocupación es por adiciones al cuerpo del Señor más que adiciones a una lista de membresía. Su enfoque es llevar pecadores a una consciencia de su pecado y de las consecuencias de quedarse en pecado. Si ellos pueden hacer esto en un solo estudio, ¡fantástico! Pero si tiempo considerable se requiere para desarraigar conceptos falsos y plantar la semilla verdadera del evangelio, ellos pacientemente aceptan esto. Su único sentido de urgencia brota de la incertidumbre de la vida y su duración. Pero ellos saben que atajos no son la respuesta; que bautismo sin arrepentimiento es sin valor; y que una vez que la gente se lleva a arrepentimiento verdadero, habiendo sido enseñado correctamente, bautismo para la remisión de pecados seguirá. Entonces ellos esperan con paciencia hasta que el evangelio lleve a cabo su efecto deseado en los corazones de los que ellos enseñan. 

Miembros de \textbf{la primera iglesia} serán tentados a usar tácticas cuestionables en su trato con la gente. Los métodos y planteamientos viejos ya no parecen ser efectivos. Planteamientos nuevos y más positivos se deben encontrar. Entonces los miembros de la primera iglesia hacen su llamamiento al orgullo de la gente. Ellos los persuaden de su autoestima; construyen su autoimagen; les dicen lo valioso que serían para la congregación. «Te necesitamos», ellos dicen a sus candidatos. Ellos también pueden ensalzar las virtudes de la congregación, persuadiendo sus candidatos del valor de ser una parte de un grupo de personas tan vibrante y en crecimiento. Entonces, la gente «se convierten en miembros», y se conforman a las reglas que se les presentan para aprobación dentro del grupo, pero puede ser que haya habido poca tristeza sobre pecado; de hecho, ellos incluso pueden seguir creyendo que eran cristianos antes de «convertirse en miembros». 

Los miembros de \textbf{la segunda iglesia} reconocen que el evangelio nunca apela al orgullo de la gente. Ellos llevan a la gente a ver su bancarrota espiritual; que ellos no tienen «nada con que pagar»; que su valor verdadero no se encuentra en el yo, sino en Cristo; que ellos deben humillarse y mirar a Cristo por su exaltación; que ellos son pecadores en necesidad desesperada de salvación; que su única esperanza se encontrará en Cristo. Ellos los llevarían a decir, en las palabras de Sra. C.H. Morris: 
\begin{flushleft}
\begin{verse}
Cerca, más cerca, yo nada traigo,\\
Nada que le pueda ofrecer a Jesús mi Rey,\\
Solo mi corazón pecaminoso, ahora contrito;\\
Concédeme el lavamiento Tu sangre imparte.
\end{verse}
\end{flushleft}
\textbf{La primera iglesia} puede comenzar a aceptar enseñanza inferior. Sus ancianos quisieran mantener solidez doctrinal, pero hay la presión de producir, de mantener la tasa de crecimiento puesta ante la congregación. Cuando solidez doctrinal se convierte en un obstáculo a ese propósito, los ancianos pueden sucumbir a las presiones y relajar su enseñanza. \textbf{La segunda iglesia} no enfrenta tal presión, porque en su preocupación por el bienestar espiritual de la gente, hay deseo de verdad en todo tema vital a la salvación. 

El énfasis de \textbf{la primera iglesia} es organizacional e institucional; el énfasis de \textbf{la segunda} es espiritual y celestial. 

Elogiamos \textbf{la segunda iglesia} a nuestros lectores. Problemas serios pueden resultar cuando iglesias ven crecimiento como su propósito principal. Si se van a establecer metas – y metas pueden servir un buen propósito – que se enfoquen en el numero para enseñar en vez del numero para bautizar. Si nuevos enfoques se necesitan, que se conciban solo si sean compatibles con la sabiduría de Dios. En nuestros esfuerzos para alcanzar a otros, que todos determinen a saber nada «excepto a Jesucristo, y este crucificado\ibible{ICorinthians}(2:2)». Cuando iglesias así se vuelven serias acerca de salvar almas, Dios dará el aumento y el crecimiento se cuidará solo. 

\section{Dos Maneras De Mantener Fieles A Los Miembros}
Hay dos maneras de mantener «fieles» a los miembros de la iglesia. \textbf{La primera manera} es asegurarse de que todos estén participando. Tengan un proyecto para cada miembro, y asegúrense de que él se sienta importante en su rol. Alábenle por el buen trabajo que él está haciendo. Hagan que él se sienta necesario; háganle sentir que el bienestar de la congregación entera descansa firmemente sobre sus hombros. Eso le mantendrá «fiel». 

Hay dos problemas con este método. En el primer lugar, alienta la creación de proyectos que no están remotamente relacionados con la obra de la iglesia local. Un miembro juega en el equipo de pelota de la iglesia; otro dirige la tropa de los Boy Scout; otro es un miembro activo de la «Sociedad Dorcas»; otro planifica el programa para el almuerzo de varones. Sí, todos están ocupados, pero con actividades que no son autorizadas en el Nuevo Testamento. 

En el segundo lugar, miembros a menudo se dan roles para los cuales no están capacitados. Una dama se escoge para enseñar una clase, no porque esté capacitada, sino porque ella tiene que estar involucrada. Un hombre se nombra como diacono para ayudarlo a ser «fiel». Otro hombre se nombra a servir la Cena del Señor por un mes para alentarlo a estar presente cada domingo ese mes. Este método así coloca «el carrito antes del caballo», porque nadie nunca debería ser designado para ningún trabajo en el servicio del Señor que no esté fiel y capacitado ya para el trabajo que hay que hacer (\ibibleverse{IITimothy}(2:2)).

\textbf{La segunda manera} de mantener fieles a la gente es desarrollar dentro de ellos un amor genuino por el Señor. Cuando las personas aman al Señor, serán fieles, y no requerirá algún tipo de proyecto «especial» para mantenerlas fieles. Ellos también participarán: en adoración, en estudio, en oración, en vivir piadosamente, en compartir el evangelio con un amigo, en ayudar a los necesitados. Yo he conocido a literalmente cientos de cristianos que jamás en sus vidas han sido nombrados a cualquier trabajo especial, pero cuyo simple amor por el Señor los mantiene fiel. No hay una fidelidad superficial de parte de estos; la suya es una fidelidad que es verdadera. 

Responsabilidades especiales son aceptables para los que estén capacitados, pero el hombre que requiere algún deber especial para ser fiel nunca ha aprendido lo que es la fidelidad verdadera. 

\section{Dos Reyes Buscan Grandeza}
Jeremías habla de dos reyes que buscaron la grandeza (\ibibleverse{Jeremiah}(22:13-16)). \textbf{El primer rey}, Joacim, tuvo un enfoque interesante: amplifica tus casas, remodela tus edificios, píntalos de rojo, rodéate de lujos, y demuestra que eres un gran rey. Su concepto es obvio: \textbf{La grandeza se halla en lo externo}. La pregunta desafiante de Jeremías a Joacim fue: «¿Acaso te harás rey porque compites en cedro?» (\ibiblechvs{Jeremiah}(22:15)). 

El padre de Joacim – \textbf{nuestro segundo rey} – había demostrado un concepto diferente de la grandeza. Josías no había dejado de «comer y beber», pero él había demostrado mucha más preocupación por hacer justicia y rectitud, juzgando la causa del afligido y necesitado, y llegando a conocer a Dios (\ibiblechvs{Jeremiah}(22:15-16)). Su concepto había sido: \textbf{Lo externo tiene su lugar, pero la grandeza verdadera se halla en justicia, compasión, rectitud, servicio humilde a Dios y a los hombres}. 

El concepto de Joacim fácilmente puede hallar paso en la iglesia. Un predicador, deseando hacer un nombre para sí, cultiva cuidadosamente las conexiones perfectas, desarrolla el vestimento y la personalidad perfectos, y comienza a hacer su camino hacia la «grandeza». En buscar una solución a periodos de adoración aburridos y sin vida, hombres cambiaran el orden del servicio, bajar las luces, alentar canto «espontaneo», incluso cambiar himnarios – buscando la excelencia por manipular lo externo. Una congregación amplifica el edificio, renueva su auditorio, inicia nuevos programas, y desarrolla fraseología más moderna para dar a sus programas un aspecto de sofisticación mientras se extiende para alcanzar el estatus de «primera clase» en los ojos de los hombres. Utilizando el estilo de Jeremías, preguntamos: ¿Te conviertes en predicador por moda, estilo, y maniobras políticas? ¿Te conviertes en una gran iglesia porque estas competiendo en ladrillo y mortero? ¿Haces que la gente sea espiritual por bajar las luces y cambiar el orden de los servicios?

No estamos sugiriendo que lo externo se debe ignorar. Tienen su lugar. Pulcritud de apariencia y amabilidad de manera pueden ser de algún valor en el trabajo de un predicador. Edificios limpios y bien mantenidos normalmente reflejan diligencia en otros aspectos del trabajo de una iglesia. Y el Señor ciertamente nos ha enseñado acerca de nuestros periodos de adoración, «Que todo se haga decentemente y con orden» (\ibibleverse{ICorinthians}(14:40)). 

Pero grandeza verdadera de parte de un predicador o una iglesia se halla, no en lo externo, sino en autenticidad, humildad, conocimiento, espiritualidad, amor de la verdad, preocupación por los perdidos, fe, valor, esperanza, obediencia, confianza, servicio a Dios y al hombre. Estas son las cualidades que hacen la grandeza en los ojos de Dios. La verdadera solución a periodos de adoración aburridos y sin vida se halla en adoradores que aman, que están agradecidos, que adoran. Cambia la gente, y los periodos de adoración se cuidarán solas.

La grandeza sí se debe buscar, pero que no confundamos la admiración temporal de hombres inconstantes e inestables con la grandeza verdadera. 

\section{Dos Ancianos Supervisan}
Dos hombres se nombran ancianos, pero varían grandemente en como supervisan el rebaño. \textbf{El primer hombre} es un mero portavoz para una menoridad pequeña de la congregación. Él tiene poco contacto con cualquiera fuera de su familia y sus amigos más cercanos. Cuando él ha escuchado la opinión de ellos sobre alguna cuestión, piensa haber escuchado a todos; su aprobación fuerte de cualquier obra se interpreta como un mandato de toda la congregación. 

\textbf{El segundo hombre} se preocupa por el grupo entero. Él cultiva amistades con todos y valora el pensamiento de los menos vocales igual al de sus amigos más cercanos. De hecho, él a menudo busca el consejo y pensamiento de los que no son tan rápidos de hablar. Él extiende hospitalidad a todos los que son una parte de la iglesia, pero especialmente a los que pueden sentirse desatendidos o excluidos de cosas. Él ama a cada miembro y encuentra alguna manera de comunicar su amor y aprecio por cada uno. 

\textbf{El primer hombre} llama a las mismas pocas personas para toda tarea que realizar. Cuando él visita a los enfermos o los que no pueden salir de su casa, él siempre esta acompañado de estos mismos. Él está inconsciente del desarrollo espiritual de los en la «margen»; de hecho, apenas los conoce. Si él tiene que ir a uno de ellos con un problema, él va como un desconocido en vez de un amigo o un hermano. 

\textbf{El segundo hombre} trabaja para el desarrollo espiritual de todo miembro. El está consciente de jóvenes que tienen potencia y está ayudándolos a desarrollar esa potencia para el uso del Señor. Él está alentando constantemente a hombres a dirigir su primera oración o hacer su primera charla. El alienta a mujeres que han dudado en el pasado en involucrarse en enseñar, extender hospitalidad, o preparar un plato para una familia desconsolada. El sabe que buen liderazgo desarrolla liderazgo en otros, y está buscando constantemente por los que pueden servir como ancianos en el futuro y los alienta. La mejor palabra para describir este hombre es «consciencia». El conoce a la gente y ellos le conocen a él. Los problemas de ellos son los problemas de él; las tristezas de ellos, tristezas de él; gozos de ellos, gozos de él. El está consciente de su potencia, sus fuerzas, sus debilidades, sus necesidades. 

\textbf{El primer hombre} en realidad puede ser un obstáculo al bienestar de la iglesia y una gran frustración a los propósitos y metas del segundo hombre. Ancianos tienen la responsabilidad de mezclar con el grupo entero y de actuar imparcialmente. Ellos deben considerar las \textbf{necesidades} de todos en vez de los \textbf{deseos} egoístas de unos pocos. Ellos deben «velar por las almas» de todos (\ibibleverse{Hebrews}(13:17)).

Es obvio que es \textbf{el segundo hombre} en nuestro articulo que complace a Dios como anciano. De hecho, puede ser que nuestro articulo seria mejor titulado, «Un Anciano Supervisa», porque nuestro primer hombre hace más «pasar por alto» que «supervisar»; más descansar en el aprisco que pastorear el rebaño. «Tened cuidado de vosotros y de toda la grey\ldots» (\ibibleverse{Acts}(20:28)k).

\section{Dos Iglesias Tienen Ancianos}
Dos iglesias tienen ancianos, pero difieren enormemente en sus actitudes hacia los ancianos. \textbf{La primera iglesia} ve los ancianos como grandes blancos para la crítica. Todo movimiento se cuestiona, y existe mucho descontento y agitación. De hecho, los ancianos tienen que pasar tanto tiempo tratando con la agitación que tienen poco tiempo para planificar e implementar trabajo que llevará a la iglesia al cumplimiento de su propósito dado por Dios. 

\textbf{La segunda iglesia} reconoce la carga pesada llevada por los ancianos, y están agradecidos que hay hombres que desean hacer el trabajo. Ellos hacen concesiones por sus imperfecciones y buscan apoyarlos en toda manera posible. Si una decisión se hace que algún miembro cree imprudente, el va a los ancianos con su problema y no causa agitación dentro de la congregación. Los miembros están «listos para hacer toda buena obra»\ibible{IITimothy}(3:17), dispuestos a cooperar en cualquier actividad que concuerde con la autoridad de Dios. 

\textbf{La primera iglesia} piensa de los ancianos como meros tomadores de decisiones. Les molesta el esfuerzo de los ancianos de amonestar. Ellos ven cualquier intento de hablar con alguien acerca del pecado en su vida como una invasión de su privacidad. Ellos realmente quisieran que los ancianos solo tomarían las decisiones que tienen que tomar y que dejarían a todos en paz. 

\textbf{La segunda iglesia} reconoce que, en su rol como pastores, los ancianos deben tomar ciertas decisiones que pertenecen al bienestar de la iglesia, pero los ven principalmente como «veladores de almas» y alimentación espiritual. Ellos respetan a los ancianos y responden a sus amonestaciones. Ellos saben que los empujones son para su propio bien y que los ancianos tienen su bienestar espiritual y eternal en la mente. 

Los miembros de \textbf{la primera iglesia} raramente expresan aprecio por sus ancianos y raramente oran por ellos. Ellos ven las fallas de sus ancianos, pero están cegados a sus buenos atributos. No hay esa buena relación entre los ancianos y la congregación que Dios desee en la iglesia. Algunas de las fallas pueden radicar en los mismos ancianos, pero la congregación realmente no les ha dado la oportunidad de dirigir y ejercer la supervisión que debería ser suya. 

\textbf{La segunda iglesia} es un ejemplo de paz y armonía. Ellos aprecian a sus ancianos, oran por ellos, y «los tienen en muy alta estima con amor, por causa de su trabajo» (\ibibleverse{IThessalonians}(5:12-13)). Ellos a menudo les expresan su amor y apoyo, así que los ancianos trabajan entre ellos con confianza, cumpliendo sus responsabilidades pesadas «con alegría y no quejándose» (\ibibleverse{Hebrews}(13:17)). 

\textbf{La primera iglesia} se pregunta por que no pueden tener ancianos fuertes como los de la segunda iglesia, nunca reconociendo que la reacción de la misma congregación determina en gran manera la efectividad con que los ancianos supervisen, dirigen, y gobiernan la iglesia. La clave del cambio en la primera iglesia yace dentro de solo unos pocos, que, con un cambio en sus actitudes, podrían guiar al grupo entero a actitudes mejores. ¿Puede ser posible que \textbf{tú} estás entre los pocos?

\chapter{CONTRASTES/\\PREDICADORES}

\section{Tres Hombres Presentan La Palabra}
¿Cuán positivo y seguro debería ser uno en su presentación de la palabra? Hemos observado tres tipos de hombres en relación con esta pregunta.

\textbf{El primero} es el tipo dudoso que raramente dice algo con confianza. Toda declaración se prologa con «posiblemente», «puede ser», «unos de los comentarios dicen», o «yo pienso». Incluso \ibibleverse{Revelation}(21:8): «Pero los cobardes, incrédulos, abominables, asesinos, inmorales, hechiceros, idólatras y todos los mentirosos tendrán su herencia en el lago que arde con fuego y azufre…» se cita en tonos suaves (¡y eso requiere habilidad verdadera!) que a nadie convencerán.

\textbf{El segundo} expresa a todo en una manera positiva y asegurada, especialmente esas proclamaciones de las cuales él sabe absolutamente nada. El puede estar equivocado de la manera más positiva. Él es el verdadero dogmatista, viviendo por la filosofía, «Grita aquí; el punto es débil»; o posiblemente él se haya engañado, no reconociendo sus propias limitaciones. 

\textbf{El tercero} estudia muy cuidadosamente la palabra del Señor, y en consecuencia habla con confianza de esas conclusiones que él puede defender usando las escrituras. Cuando él habla, da la impresión al oyente que «este hombre realmente cree lo que está diciendo». Él no tiene miedo de ser cuestionado acerca de cualquier convicción, porque él está «siempre preparado para presentar defensa ante todo el que [le] demande razón de la esperanza que hay en [él]» (\ibibleverse{IPeter}(3:15)). A la vez, él no vacila en decir, «No lo sé», cuando se le pregunta algo de que él no está seguro. 

De estos tres tipos, el segundo es el menos aceptable y el más peligroso. Nos maravillamos al descaro de tales hombres, pero nos maravillamos aún más a la credulidad de congregaciones que dejarán que ellos dominen sus clases, púlpitos, y reuniones de negocios. 

Al otro lado, la persona que deja que su repulsión al segundo tipo le empuje a aceptar el tipo flojo, «nunca estés seguro de nada», hace un error triste. La verdad tiene su propio sonido peculiar, y ese sonido es el sonido de seguridad. Uno lo escucha indudablemente en la enseñanza de los apóstoles: «Yo sé en quien he creído…» (\ibibleverse{IITimothy}(1:12)); «No seguimos fábulas ingeniosamente inventadas…» (\ibibleverse{IIPeter}(1:16)); «nosotros también creemos, por lo cual también hablamos; sabiendo…» (\ibibleverse{IICorinthians}(4:13-14)). Y uno lo escuchará en enseñanza fiel hoy en día. 

La verdad es el factor decisivo. Cuando uno puede apoyar su enseñanza con una aplicación correcta de las escrituras, él puede – no, él debería – hablar con valor, claridad y certeza. Tal enseñanza no es dogmatismo; es contender «ardientemente por la fe»\ibible{Jude}(1:3). No se debería criticar; debería ser apreciado y alentado. Que nadie entonces sea intimidado, «Porque no nos ha dado Dios espíritu de cobardía, sino de poder, de amor y de dominio propio. Por tanto, no te avergüences del testimonio de nuestro Señor…» (\ibibleverse{IITimothy}(1:7-8)).

\section{Dos Hombres Enseñan La Palabra De Dios}
Personas que son útiles en el servicio del Señor tienen equilibrio en su pensamiento. Ellos mantienen moderación de la mente. Ellos saben lo que creen y por qué lo creen, y no se dan a fluctuaciones radicales de pensamiento. 

Dos hombres son maestros de la palabra de Dios. Ambos son hombres concienzudos que aman a la verdad. \textbf{El primer hombre}, sin embargo, fácilmente se influye por cualquier cosa que lee o escucha. Él frecuentemente cambia posiciones en asuntos doctrinales. Queriendo ser independiente en su pensamiento, él es rápido de rechazar pensamiento tradicional «de la Iglesia de Cristo». Él adopta en su pensamiento cualquier cosa que en la superficie le suena razonable. El ama encontrar alguna «joya» nueva y emocionante que distinguirá su enseñanza de la enseñanza vieja y manida que la gente ha escuchado por años. Sobre todo, el quiere que su enseñanza provoque pensamiento, que sea estimulante, diferente y emocionante, nueva y fresca en su enfoque. 

\textbf{El segundo hombre} quiere que su enseñanza sea estimulante y que provoque pensamiento también, pero él reconoce que uno no tiene que rechazar lo probado para ser estimulante e independiente. El no se sorprende hallar que sus convicciones son parecidos a las de otros que han labrado para separar error de verdad, tradicionalismo de la pura palabra de Dios, sectarismo de la unidad hallada en Cristo. Después de todo, el comparte las mismas metas y estudia el mismo libro. El no ve ninguna virtud en ser diferente de ellos. El ha llegado a conclusiones solidas basadas en su estudio de las escrituras y no se mueve fácilmente de esas conclusiones. El encuentra estimulante la verdad porque es la verdad. 

\textbf{El primer hombre} tiende a ser «llevado de aquí para allá por todo viento de doctrina» (\ibibleverse{Ephesians}(4:14)). Sus oyentes nunca saben qué esperar de él. De una cosa pueden estar seguros: cualquier manía en que él esté en aquel entonces saldrá en sus lecciones, porque todo nuevo concepto parece convertirse en una obsesión. 

\textbf{El segundo hombre} tiene su corazón «fortalecido con la gracia» (\ibibleverse{Hebrews}(13:9)); él está «cimentado y constante» (\ibibleverse{Colossians}(1:23)). El cuestiona posiciones que él ha tenido, y se fuerza por su honestidad e integridad personales a cambiar posiciones a veces, pero él actúa muy lento y cuidadosamente en hacerlo. Él guarda muchas preguntas que surgen en su mente para sí, porque el reconoce que ellas no son vitales para su propia salvación ni para la salvación de otros. 

\textbf{El primer hombre} perturba a otros innecesariamente con su enseñanza, dejándolos con más preguntas que respuestas. Unas de sus conclusiones son peligrosas. Y aunque después él reconoce su peligro y las descarta, él ya ha plantado, en su apuro, semillas de error en los corazones de sus oyentes. Su influencia se daña porque los hermanos le tienen miedo – miedo justificable. Su utilidad en el reino se afecta grandemente. 

Sugeriríamos a \textbf{nuestro primer} hombre que antes de crear problemas entre el pueblo de Dios y dañar su propia influencia, él puede preguntarse las siguientes preguntas: 
\begin{enumerate}
\item ¿Estoy seguro de las conclusiones a que he llegado? ¿Es posible que yo haya pasado por alto alguna escritura o argumento pertinente que negaría mis conclusiones?
\item ¿Estoy seguro de que mis conclusiones no se hayan mancilladas por prejuicio, desilusión, amargura, celos, consideraciones emocionales, o algún otro factor que puede afectar negativamente al pensamiento de uno?
\item Incluso si yo estoy seguro de mis conclusiones, ¿es vital el punto que estoy enfatizando, de suficiente importancia para justificar problemas?
\item ¿He dejado que estas conclusiones se conviertan en una obsesión? ¿Me encuentro hablando de ellas frecuentemente – en clases bíblicas? ¿En sermones? ¿En discusiones privadas? ¿Encuentro que mi lectura de las escrituras algo «colorado» por estos nuevos conceptos?
\end{enumerate}
No estamos alentando transigencia, ni estamos sugiriendo que uno tenga que ganar la aprobación de la hermandad de sus conclusiones antes de enseñarlas. Estamos diciendo, sin embargo, que la precaución dicta que uno vaya lento en adoptar nuevos conceptos y que tenga aún más cuidado en enseñarlos. «Vuestra bondad sea conocida de todos los hombres» (\ibibleverse{Philippians}(4:5)).

\section{Dos Hombres Predican El Evangelio}
La predicación fiel del evangelio demanda cierta cantidad de enseñanza negativa. Las maldades de la inmoralidad, error doctrinal, y división se deben mostrar. Pero hay una diferencia entre predicación negativa y el negativismo. Por ejemplo…

Dos hombres predican el evangelio. \textbf{El primer hombre}, sin embargo, se da al negativismo. Prácticamente todos sus sermones son o una exposición de alguna doctrina falsa o una denunciación de la congregación por su debilidad. Miembros asisten a periodos de adoración pensando «¿de qué va a quejarse hoy?». Mientras puede ser que su intención no sea esto, su mensaje básico a la congregación es, «¡No lo son! ¡No lo pueden! ¡No lo harán!».

\textbf{El segundo hombre} reconoce la necesidad de enseñanza negativa también, pero su enfoque principal es positivo. Él mira adelante a las metas que él cree que la iglesia puede alcanzar, y planifica sus lecciones con esas metas en la mente. Él busca hacer su parte en llevar todo miembro a una relación más estrecha con Dios, a una mayor fe en la oración, a un mayor amor los unos por los otros, a un mayor deseo por los cielos. El también quiere llevar a cada miembro a un odio por el pecado y falsa enseñanza, y su enseñanza negativa se presenta con esto en la mente. Cuando el pecado levanta su cabeza fea en la congregación, él sabe reprender, incluso con dureza. Pero su mensaje básico a la congregación es «¡Lo son! ¡Lo pueden! ¡Yo sé que lo harán!».

\textbf{El primer hombre} simplemente no puede entender por qué tantos critican su predicación. «No pueden soportar la sana doctrina», él piensa. Y él argumenta bien su caso: «No fue el Señor negativo en Su denunciación de los escribas y los fariseos (\ibibleverse{Matthew}(23:))?», él pregunta. «Y qué de Sus cartas a los efesios (\ibibleverse{Revelation}(2:1-7)) y a los laodicenses (\ibibleverse{Revelation}(3:14-22))?». Habiendo razonado así, él esta seguro que su predicación no es el problema; que el problema radica en otro lado. 

\textbf{El segundo hombre} reconoce que Jesús frecuentemente habló negativamente (a veces con una finalidad comparable a lo que nosotros expresaríamos con respecto a los falsos maestros de nuestro día – \ibibleverse{Matthew}(15:12-14); \ibiblechvs{Matthew}(21:28-46)), pero que su enseñanza en general era una combinación sensitiva de lo negativo con instrucción, esperanza, preocupación, y consuelo. Él, también, escucha las denunciaciones, pero también escucha las apelas del Señor a los denunciados, «O Jerusalén, Jerusalén…» (\ibibleverse{Matthew}(23:37)). «Al vencedor…» (\ibibleverse{Revelation}(2:7)), y «He aquí, yo estoy a la puerta y llamo…» (\ibibleverse{Revelation}(3:20)). Buscando seguir el patrón del Señor en su propia enseñanza, el se esfuerza por alcanzar el equilibrio apropiado entre lo negativo y lo positivo. El extiende esperanza a otros. El constantemente los señala al Salvador misericordioso que perdona a Sus seguidores verdaderos.

\textbf{El primer hombre} desalienta a otros, mientras \textbf{el segundo hombre} toca las vidas de muchos para bien. Ambos hombres son sinceros, pero el primero simplemente derriba la maldad, mientras el segundo derriba la maldad \textbf{y} construye lo bueno. 

\section{Tres Hombres Predican El Evangelio }
Tres hombres predican el evangelio. \textbf{Tom} es de los que despotrican. Él se excita tanto en su predicación. Sudor chorrea mientras él habla con emoción del Señor y Su evangelio. Su personalidad es igual fuera del púlpito. Él es muy extrovertido y fácilmente hace amigos. Tom es muy querido, y está trabajando muy efectivamente por el Señor. 

\textbf{Dick} es muy diferente de Tom. Él es muy conversacional en su predicación. Sus sermones se desarrollan más lentamente que los de Tom, pero hay una calidez en su presentación, y uno casi siente que Dick esté hablando directamente a él. Dick, también, es efectivo, igual dentro y fuera del púlpito, como un siervo del Señor. 

\textbf{Harry} es diferente de los otros dos. Él es del tipo erudito de predicación. El tema en cuestión se destaca arriba de estilo o presentación. Es más difícil escucharle a él que a los otros dos, pero el oyente atento sale del servicio bien recompensado. Su pensamiento ha sido desafiado. Su «cubeta» se ha llenado. Créelo o no, Harry también es efectivo y constantemente está en demanda entre las iglesias. 

Este escritor debe admitir que Dick es su favorito tipo de predicador; eso es, hasta que escucha a Tom; y después Tom es su tipo favorito hasta que escucha a Harry. Simplemente parece que su estilo favorito o predicador favorito depende de a quién está escuchando en el momento. Pero él debe admitir, también, que hay unos predicadores que le cuesta escucharlos. Simplemente no le atraen mucho. Y, aun así, el mismo predicador que no le atrae a este escritor puede ser el favorito de alguien más. Los hombres difieren tanto en estilo, en personalidad, en presentación, en temas, en voz, e incluso en capacidad. Pero todo predicador fiel es el predicador favorito de alguien, porque, así como predicadores varían en su predicación, oyentes varían en lo que les gusta en predicadores. 

Los corintios aparentemente no podían apreciar diferentes tipos de predicadores. Sectarismo resultó. «Yo soy de Pablo», ellos dijeron; o «Yo se de Apolo», o «Yo soy de Cefas», o «Yo soy de Cristo» (\ibibleverse{ICorinthians}(1:12)). ¿Te lo puedes imaginar? Aquí están cuatro grandes caracteres, uno de los cuales es el Hijo de Dios; pero tres se tienen que rechazar porque los corintios solo tienen espacio en sus corazones para uno. Cuánto necesitaban amplificar sus corazones para estar realmente agradecidos por todo hombre bueno y toda buena influencia que había tocado sus vidas. «Así que nadie se jacte en los hombres, porque todo es vuestro: ya sea Pablo, o Apolos, o Cefas, o el mundo, o la vida, o la muerte, o lo presente, o lo por venir, todo es vuestro, y vosotros de Cristo, y Cristo de Dios» (\ibibleverse{ICorinthians}(3:21-23)).

Predicadores deben encontrar el rol en que pueden hacer su trabajo más efectivo. Ancianos deben encontrar hombres que mejor pueden satisfacer las necesidades de la congregación. Pero que todos seamos agradecidos por todo hombre bueno que fielmente proclama el evangelio de Cristo. 

\section{Dos Hombres Ven A Su Vecino Perdido}
Dos hombres ven a su vecino perdido, pero lo ven desde diferentes perspectivas. \textbf{El primer hombre} lo ve como un activo para la iglesia. Por años, la iglesia con que él adora ha luchado por falta de recursos humanos y liderazgo, y aquí esta el mismo hombre que podría llenar el hueco. «Si solo pudiéramos ganarle a nuestro lado», el primer hombre piensa, «qué ayuda él nos podría ser».

\textbf{El segundo hombre} lo ve principalmente como alguien en necesidad de salvación. Lo ve como alguien que está perdido, sin Cristo ahora, y condenado al infierno eternamente. Ciertamente, el hombre podría ser un activo para la iglesia si fuera sinceramente convertido, pero su necesidad por Cristo y salvación es infinitamente mayor que la necesidad que la iglesia tiene de él. «Si podemos traerlo a Cristo», el segundo hombre piensa, «se puede salvar, ir a los cielos, y traer gloria al Señor».

\textbf{El primer hombre} se acerca al vecino, diciéndole cuanto la iglesia lo necesita y qué activo él podría ser. Su llamamiento se construye sobre adulación y orgullo. 

\textbf{El segundo hombre} se le acerca con el evangelio. El intenta llevarlo a una consciencia de su pecaminosidad, desesperanza, e impotencia. Le señala a Cristo como la única respuesta a su mayor necesidad. Él sabe que hasta que este hombre se sacude de su opinión alta de sí mismo, él no puede venir a Cristo; que hasta que él venga con pobreza de espíritu, duelo, mansedumbre, hambre y sed de justicia, etc., él no puede tener salvación ni ser adecuado para el uso del Amo. Su llamamiento entonces se construye sobre persuasión y convicción urgente. 

\textbf{El primer hombre} haría que su vecino bajara por el pasillo con una sonrisa en la cara para hacerles un favor a la iglesia. El segundo hombre haría que él viniera con lagrimas amargas buscando lo que él necesita tan desesperadamente. El concepto de la iglesia del primer hombre es principalmente el de una organización buscando ayuda por sus necesidades básicas. El concepto del segundo hombre es el de personas que se han inclinado en sumisión a Cristo y han encontrado en Él la respuesta a sus necesidades. El llamamiento del primer hombre es carnal; el llamamiento del segundo hombre es espiritual. El primer hombre puede tener éxito en conseguir que su vecino esté «en la iglesia» y «ganarle a nuestro lado», pero el segundo hombre tendrá éxito en salvar un alma de la muerte. Por desgracia, el primer hombre puede «ganar» más con su enfoque que el segundo hombre, porque el llamamiento carnal parece mucho más fuerte que el espiritual en nuestro día, pero, en verdad, él debilita a la iglesia con todos los que él «convierte».

«Las armas de nuestra contienda no son carnales, sino poderosas en Dios para la destrucción de fortalezas» (\ibibleverse{IICorinthians}(10:4)). Estas armas poderosas se deben usar hábilmente, pero no hay lugar para sustitutos. Llamamientos carnales se deben descartar. Adulación y orgullo representan el espíritu muy opuesto al del reino del Señor. Ciertamente, nuestro segundo hombre usa el único enfoque que complace a Dios. 

\section{Dos Hombres Se Convirtieron En Predicadores}
Dos hombres se convirtieron en predicadores. \textbf{El primer hombre} abordó el trabajo solo desde la perspectiva profesional. Él había observado varias cosas de la vida de un predicador que le atrajeron. Le gustó la idea de pararse en frente de audiencias cada semana, y ser por treinta minutos «el centro de atención»; le gustaron los cumplidos y las palabras amables de aliento; él había observado el aumento de salario de los predicadores; y en particular él esperaba predicar en series y ser bien conocido en la hermandad; él había notado que predicadores eran admirados, y a menudo se entretenían en los hogares de los hermanos. Estas eran las incentivas principales que le guiaron a convertirse en predicador. 

\textbf{El segundo hombre} había pensado poco en convertirse en predicador. Él simplemente había trabajado duro para ser un cristiano, para ser el tipo de persona con quien Dios estaría complacido. Pero los ancianos, impresionados con su dedicación y buena vida, le habían pedido hablar un domingo por la noche, así él había comenzado. Pronto otras iglesias se habían enterado de sus habilidades, y ellas, también, le habían invitado a hablar. Al final un conflicto se había desarrollado entre sus deberes urgentes seculares y el tiempo pasado en la obra del Señor, y era necesario tomar una decisión: o dejar su trabajo secular o rechazar oportunidades de trabajar por el Señor; Su amor por el Señor no permitiría este último, así que, después de mucho pensamiento, él dejó su trabajo para predicar el evangelio «por tiempo completo». Su única incentiva fue su amor por almas y deseos de complacer al Señor. 

\textbf{El primer hombre} pronto se vuelve algo desilusionado. Él encontró que los hermanos no cedieron a todo capricho del predicador como él había pensado que hacían, y estuvo particularmente enfurecido cuando uno de los hermanos cuestionó algo que él había dicho en una clase bíblica. Además, los artículos que él había enviado a «las publicaciones de la hermandad» ni se imprimieron, y después de cinco años de predicar él aún no se había pedido predicar en una serie. Unos de sus amigos por entonces estaban predicando por las iglesias más grandes y estaban recibiendo publicidad considerable, y esto le causó más desanimo. Eventualmente, una oportunidad por empleo secular se le presentó y la tomó. Él siguió asistiendo a los servicios de la iglesia, pero amargura y resentimiento se quedaron. Él era la víctima trágica de profesionalismo en el púlpito. 

\textbf{El segundo hombre}, «después de poner la mano en el arado», nunca volvió atrás. El predicó dondequiera que las oportunidades se presentaron – en lugares grandes, en lugares pequeños; en casas, en tiendas; a los ricos, a los pobres; a los educados, a los incultos; cuando había apoyo económico, cuando no había apoyo económico. Y cuando él predicaba, agradecía a Dios por usarlo a pesar de su indignidad. Él nunca alcanzó la fama, pero él nunca buscó la fama. Cuando él murió, dejó atrás a miles de almas que se habían movido por su influencia, y él fue a su «hogar», a ese galardón eterno por lo cual él había dedicado su vida. 

«Hermanos míos, no os hagáis maestros muchos de vosotros, sabiendo que recibiremos un juicio más severo» (\ibibleverse{James}(3:1)). 

\section{Dos Predicadores Ven Su Trabajo}
Dos predicadores varían grandemente en sus actitudes hacía su trabajo. \textbf{El primer predicador} siente que él debería estar en el púlpito cada vez que la iglesia asamblea para «periodos de adoración» y que debería enseñar una clase durante todo periodo de estudio bíblico, mientras \textbf{el segundo predicador} alienta a otros a hablar y frecuentemente se sienta en clases bíblicas enseñadas por otros. Sus acciones diferentes brotaron de dos filosofías diferentes con respecto a como las iglesias se desarrollan. 

\textbf{El primer predicador} cree que el mayor desarrollo resulta cuando el «mejor» predique y enseñe. Él ha estudiado duro y ciertamente es el hombre más informado en la congragación; él se paga por predicar; él, entonces, debe ser el mejor cualificado para el trabajo; por consiguiente, él debería hacer la predicación y enseñanza. Él guarda su «posición» en el púlpito y el aula muy cuidadosamente. Se rinde cualquier de los dos con mucha renuencia. «Después de todo, ¿para qué forzar a la iglesia a escuchar hombres de habilidad inferior cuando yo podría estar haciendo la instrucción?», él se pregunta. 

\textbf{El segundo hombre} cree que el mayor desarrollo del grupo entero resulta cuando cada individuo se da la oportunidad de desarrollarse. Es su meta llevar la congregación a menos dependencia del «predicador de tiempo completo» en vez de más dependencia de él; a desarrollar habilidades dentro de otros en vez de ahogar su desarrollo. Él reconoce, también, que él no es el único que haya tenido pensamientos que la congregación necesita escuchar; que un enfoque diferente puede ser instructivo y refrescante, y especialmente cuando viene de los labios de uno que puede relacionarse con la congregación mejor que el «predicador de tiempo completo». No es que él sea perezoso y no quiera enseñar. Él ha dedicado su vida intentando convertirse en un maestro mejor, y, francamente, le es más fácil enseñar que no enseñar, predicar que no predicar; pero él se sienta, a veces angustiosamente, mientras otros se dan una oportunidad de desarrollar e instruir. Él lo hace porque está convencido que la fuerza general de la iglesia será servida por traer otros al programa de enseñar. 

Al final puede ser una cuestión de juicio, pero estamos de acuerdo con la filosofía del segundo predicador. No convertiríamos los periodos de adoración en meras sesiones de entrenamiento (clases de entrenamiento se deberían proveer para principiantes), pero mientras hombres se desarrollan en sus habilidades de edificar la iglesia, se deberían permitir hablar y compartir sus pensamientos con la congregación. Y debe ser la meta de predicadores y ancianos capacitar tales hombres para edificar. Además, la iglesia se debería llevar a una aceptación de tales hombres en el púlpito y el aula. Iglesias no se hacen fuertes por un solo hombre haciendo toda la predicación y la enseñanza. 

Buen liderazgo desarrolla liderazgo en otros; predicación efectiva desarrolla otros que son capaces de predicar (\ibibleverse{IITimothy}(2:2)). Y estamos convencidos de que es el segundo predicador que cumplirá estos propósitos, no el primero. 

\section{Dos Predicadores Hablan Del Dinero}
Dos predicadores tienen puntos de vista opuestos con respecto al dinero. \textbf{El primero} cree que cuando él va a platicar con una congregación acerca de trabajar con ellos, el debería pedir el más dinero posible de esa congregación en ese mismo momento, porque seguramente él nunca recibirá más de ellos después de mudarse allí. \textbf{El segundo} cree que los hermanos tienen una bastante buena idea de lo que un hombre necesita para vivir y sostener a su familia, así que él adopta un tipo de enfoque mas suave. Él es franco con respecto a sus necesidades, pero no exigente. 

\textbf{El primer hombre} tiene toda la razón en su evaluación de congregaciones. Él \textbf{no} puede sacar nada mas de ellas después de mudarse. Ellos ya han observado su hambre de dinero (no nos atrevamos a llamarlo «avaricioso»); ellos ya se han drenado de todo lo que él puede sacar de ellos; y ellos no están por darle más. El dinero se convierte en un tema sensitivo. El predicador resiente la norma «bolsa apretada» de la iglesia. La iglesia resiente el deseo del predicador por más dinero. Dondequiera que vaya, él enfrenta el mismo problema. El escribe y habla constantemente del egoísmo de las iglesias. Las iglesias lo miran más y más como un «profesional», aunque él probablemente será el ultimo en saberlo. 

\textbf{El segundo hombre} recibe menos apoyo que el primero en el momento de mudarse, pero él usualmente encuentra que iglesias aprecian su enfoque suave y consideran sus necesidades económicas. Ocasionalmente se le ofrece un aumento, lo cual él puede o no aceptar, dependiendo de sus propias necesidades y la capacidad de la congregación de proveer el aumento. Oh, a veces él se encuentra trabajando por una iglesia que no es tan considerada como debería ser, pero él ha determinado que, si la iglesia no le apoya de acuerdo con su pensamiento, él simplemente bajará su pensamiento de acuerdo con el apoyo. Un poco mas cuidado en llamadas de larga distancia, cenando en restaurantes, viajando fuera de la ciudad, y comprando ropa (el predicador no tiene que ser el pionero de una moda) hace maravillas por su presupuesto. Pese a todo, él sale bastante bien económicamente, y se le ahorran los conflictos de dinero que el primer predicador tiene con toda congregación con que él trabaja. Incluso mientras se envejece y enfrenta periodos de menos actividad, él disfruta un sentido peculiar de seguridad económica en su fe en la providencia de Dios combinada con el amor y la estima que han resultado de su devoción desinteresada a la causa y serán demostrados incluso en maneras monetarias. 

Suponemos que es innecesario declarar cual enfoque se defiende por el escritor. Hay cientos de otros predicadores quienes comparte este enfoque. No tenemos compasión por las iglesias tacañas y poco pensativas. Pero tampoco tenemos compasión por predicadores que siempre están llorando acerca de dinero. Después de todo, fue a un predicador que se escribieron las palabras consiguientes: «Porque la raíz de todos los males es el amor al dinero, por el cual, codiciándolo algunos, se extraviaron de la fe y se torturaron con muchos dolores. Pero tú, oh hombre de Dios, huye de estas cosas\ldots» (\ibibleverse{ITimothy}(6:10-11)). \footnote{Por favor, ver página 149 para leer un artículo que le da equilibrio a este.} 

\section{Dos Hombres No Están De Acuerdo Con El Predicador}
Dos hombres no están de acuerdo con el predicador. Ambos se han enseñado que no deben simplemente «tragar» todo lo que diga el predicador; que deben pensar por si mismos. Se deben comendar, entonces, por su evaluación cuidadosa de lo que se enseña. 

Las palabras claves \textbf{del primer hombre}, sin embargo, son, «Me parece». Toda enseñanza se juzga según su propio pensamiento, con respecto a si tiene sentido para él o no. 

Las palabras claves \textbf{del segundo hombre} son, «¿Qué dice Dios acerca de ello?». El desee verdad y sabe que la verdad solo se puede encontrar en la palabra de Dios (\ibibleverse{John}(17:17)). Si él no está de acuerdo con el predicador, es porque él está convencido de que el predicador ha aplicado mal a un pasaje de escritura o ha fallado de considerar una escritura que podría afectar su conclusión. Él viene con una biblia abierta y una mente abierta, preparado a defender su posición o ceder si él ve que no es defendible. 

\textbf{El primer hombre} exalta el yo. Él confía demasiado en su propio pensamiento. Él puede hacerlo inconscientemente, pero en realidad él hace su propio intelecto y experiencias su dios. Su pensamiento se refleja en las palabras de Naamán, «He aquí, yo pensé», palabras que hubieran llevado Naamán a la tumba un leproso si no hubiera sido por la admonición de sus siervos (\ibibleverse{IIKings}(5:1-14)). 

\textbf{El segundo hombre} exalta a Dios. Su confianza está en lo que Dios dice en las escrituras. El reconoce que su propio intelecto y experiencias desvanecen hasta la nada cuando se colocan en el resplandor de la luz de la verdad. Un «así dice el Señor» termina toda controversia con él. Su pensamiento se refleja en el de los bereanos que «eran más nobles que los de Tesalónica… escudriñando diariamente las Escrituras, para ver si estas cosas eran así» (\ibibleverse{Acts}(17:11)). 

A menos que \textbf{el primer hombre} cambia su actitud él no tiene remedio. Él es susceptible a toda manera de falsas ideas. Él no puede llegar a conocer a Dios y su verdad por medie de su propia sabiduría (\ibibleverse{ICorinthians}(1:21)). Él debe descartar su propia sabiduría, intelecto, y experiencia; él debe volverse pobre en espíritu, manso ante Dios, llorando, con hambre y sed de justicia; él debe inclinarse en sumisión al Señor y a Su palabra. Él debe decir con Pablo, «¡Oh, profundidad de las riquezas y de la sabiduría y del conocimiento de Dios! ¡Cuán insondables son sus juicios e inescrutables sus caminos! Pues, ¿quién ha conocido la mente del Señor?, ¿o quién llegó a ser su consejero?» (\ibibleverse{Romans}(11:33-34)).

\textbf{El segundo hombre} ciertamente es un hombre bendecido y afortunado, porque él aprenderá la verdad que lo hará libre (\ibibleverse{John}(8:32)). Desgraciadamente, él es un raro hombre en el siglo veinte. Pero él sí existe – e incluso puede existir dentro del hombre que en este momento está leyendo este artículo. ¡Qué meta para cada uno de nosotros! Después de todo, es una cosa tener un desacuerdo con un predicador, ¡pero es bastante diferente tener un desacuerdo con Dios Todopoderoso!

\chapter{CONTRASTES/HOGAR}

\section{Dos Mujeres Estaban Casadas Con Incrédulos}
Dos mujeres estaban casadas con incrédulos. 

\textbf{La primera mujer} reconoció que ella debió amar al Señor, incluso sobre su marido (\ibibleverse{Luke}(14:26)), y determinó en su corazón que ella seria fiel al Señor. Ella nunca perdió un servicio de la iglesia solo para estar con su marido. Ella nunca traicionó sus principios para participar en alguna actividad cuestionable con él. A la vez ella nunca dio a su marido razón para cuestionar su amor; de hecho, él podía ver que su fidelidad al Señor la guio a estar en sujeción a él, amarlo, satisfacer sus necesidades físicas, ser la mejor esposa y madre que ella pudo ser. Ella siempre era leal a su marido hasta que conflictos surgieron entre su lealtad a él y su lealtad a Cristo. Ella oraba constantemente por él y le expresó su preocupación por su alma, pero ella determinó no molestarlo constantemente acerca de obedecer el evangelio. 

El concepto de \textbf{la segunda mujer} era completamente diferente. Ella a menudo había escuchado esas escrituras de amar al Señor sobre todo lo demás, pero ella no quería ofender a su marido. «Después de todo», ella pensó, «si yo voy a cada servicio y constantemente estoy dejando a mi marido solo en casa, él pronto odiará a la iglesia, y nunca se convertirá en cristiano». Ella lo protegió cuidadosamente del predicador o cualquiera que pudo querer hablar con él, y a veces incluso iba con él a lugares donde cristianos nunca deberían ir. Ella pensaba que si ella hiciera unas cosas que él quería hacer, entonces posiblemente él haría unas cosas que ella quería hacer, como ir a los servicios. 

El marido de \textbf{la primera mujer} observó el verdadero valor de servir al Señor. El podía ver el gozo, la paz, la fuerza que era suya por medio de Cristo. El apreciaba las buenas personas con quienes ella adoraba. Él comenzó a leer unos de los panfletos y materiales bíblicos que se trajeron a la casa de los servicios y eventualmente comenzó a asistir a unos de los servicios con su esposa. Él resistió por un tiempo, pero gradualmente el poder del evangelio derribó esa resistencia hasta que un domingo por la noche, mientras la congregación cantaba «¡Oh! ¿Por Qué No Esta Noche?», él pasó al pasillo, preparado para obedecer al evangelio. 

El marido de \textbf{la segunda mujer} nunca se impresionó mucho con la iglesia. Él sabía que ella importaba muy poco a su esposa. Y nunca se le ocurrió que tal vez ella quisiera que él fuera cristiano. La misma esposa se volvió más y más débil, y aunque ella nunca dejó de asistir completamente, ella encontró poco gozo en el Señor.

No implicaríamos que cada mujer fiel tendrá éxito en convertir a su marido. Pero las siguientes instrucciones del Señor deberían ser consideradas, porque ellas contienen la enseñanza del Señor a mujeres cuyos maridos no son cristianos: «Asimismo vosotras, mujeres, estad sujetas a vuestros maridos, de modo que, si algunos de ellos son desobedientes a la palabra, puedan ser ganados sin palabra alguna por la conducta de sus mujeres al observar vuestra conducta casta y respetuosa» (\ibibleverse{IPeter}(3:1-2)).

\section{Tres Visiones de Modestia}
Tres mujeres enfrentan el problema de la modestia. Las tres reconocen la enseñanza de \ibibleverse{ITimothy}(2:9): «Asimismo, que las mujeres se vistan con ropa decorosa, con pudor y modestia, no con peinado ostentoso, no con oro, o perlas, o vestidos costosos…» pero sus actitudes hacia la modestia difieren considerablemente. 

\textbf{La primera mujer} toma la perspectiva de \textbf{«¿Dónde trazas la línea?»} Si alguien pudiera trazar la línea para ella (¿en la rodilla? ¿tobillo? ¿pantorrilla?) y comprobar por la biblia que esa fuera la línea separando la modestia de la inmodestia, ella la guardaría (según ella). Pero, mientras tanto, hasta que alguien encuentre la línea en la biblia, ella viste lo que ella quiere. Si alguien le habla de su inmodestia, ella se justifica con una pregunta, «¿Dónde trazas la línea?», seguida por una observación, «Sabes que la abuelita llevaba su vestido hasta los tobillos». 

\textbf{La segunda mujer} toma la perspectiva \textbf{«sigue el código de vestimenta»}. Ella ha escuchado sermones sobre modestia, y ha establecido para si un código de vestimenta por lo cual ella vive religiosamente: no short, no sujetador halter, no traje de baño, falda debajo de la rodilla, cuello alto, etc. Ella es una buena mujer y se debe elogiar por su diligencia, pero nunca se le ha ocurrido que una mujer puede vestirse usando su código y aun así ser mundana, provocativa e inmodesta en apariencia. Ella quedaría asombrada al aprender que personas razonables la consideran inmodesta a veces. 

\textbf{La tercera mujer} se preocupa por vestirse bien, pero se preocupa más por el \textbf{carácter} de lo cual vestimenta es un reflejo. Reconociendo la enseñanza bíblica de pureza y castidad, ella se ha vuelto verdaderamente pura y casta, no solo en conducta, sino también en corazón y disposición. Ella es completamente pura, «por dentro y fuera», y su ropa refleja esa pureza. Decencia de vestimenta no es algo mecánico, «sigue el código de vestimenta», para ella. Es una consecuencia natural de su modestia por dentro. Mientras otras hermanas en Cristo andan a tientas con \ibibleverse{ITimothy}(2:9) y se preguntan por qué el Señor sería tan restrictivo con su vestimenta, ella ve esa enseñanza como algo perfectamente natural, un suplemento obvio a la enseñanza bíblica de pureza y castidad de corazón y de vida. 

Su vestimenta refleja su carácter en otras áreas. Por ejemplo, ella busca ser una dama en corazón y el comportamiento, y esta actitud se refleja en su vestimenta. Ella recuerda las advertencias en las escrituras de orgullo, y se ha vuelto verdaderamente «pobre en espíritu». Esto también se refleja en su vestimenta. Su vestimenta es un comentario de su carácter. Un vistazo revela que aquí hay una mujer pura, una dama, genuina.

Y no es esto lo que enseña \ibibleverse{ITimothy}(2:9)? El versículo dice que debemos vestirnos con decoro, pudor, y sobriedad. Cuando desarrollamos estas cualidades en nuestros corazones, entonces y solo entonces, desaparecerá nuestros problemas de vestimenta.

\section{Dos Padres Crían A Sus Hijos}
¿Cuál es peor? ¿Un padre que totalmente domina las vidas y los pensamientos de sus hijos o un padre que no impone ninguna restricción sobre sus hijos, permitiéndolos a hacer lo que quieran? Puede ser que no haya una respuesta legitima a la pregunta. Los dos están cometiendo un error triste.

\textbf{El primer padre} no enseña la responsabilidad a sus hijos. Como él mismo toma todas las decisiones para ellos, sus hijos nunca aprenden a tomar decisiones por sí mismos. Como él les dice cada movimiento que deben hacer, él les roba de iniciativa individual. Como él los levanta y se asegura de que siempre lleguen a tiempo por cada evento, ellos nunca aprenden el valor de puntualidad por sí mismos. Como él contesta toda pregunta para ellos en vez de guiarlos a encontrar respuestas para si mismos, él les quita la habilidad de pensar y llegar a conclusiones lógicas. Los hijos del primer padre, en sus últimos años de adolescencia y sus años veinte, se encontrarán enfrentando un mundo por lo cual están muy mal preparados. En su inmadurez estarán vulnerables a la dominación de otros, y en unos casos serán los que quisieran destruirlos. 

Pero \textbf{el segundo padre} comete un error triste, también. Él fuerza a sus hijos a tomar decisiones que ellos no tienen suficiente madurez para tomar; él los permite caer en situaciones que no son capaces de manejar. «El niño consentido avergüenza a su madre» (\ibibleverse{Proverbs}(29:15)).

Hijos necesitan padres que les enseñarán lo bueno y lo malo. Necesitan aprender temprano en la vida lo que significa «no» y tener cada «no» reforzado con disciplina apropiada. Hijos se deben entrenar en el camino en que deben andar (\ibibleverse{Proverbs}(22:6)). Deben ser criados «en la disciplina e instrucción del Señor» (\ibibleverse{Ephesians}(6:4)).

Los frutos de permisividad se pueden ver en Adonías, el hijo de David, de quien se dijo, «Su padre nunca lo había contrariado preguntándole: ¿Por qué has hecho esto?» (\ibibleverse{IKings}(1:6)). Adonías siempre había hecho lo que él quería sin restricciones de su padre. No debería ser sorprendente, entonces, encontrarlo usurpando el trono, haciendo lo que él quería hacer, aunque Dios había escogido a Salomón como el sucesor de David.

Los frutos de permisividad también se pueden ver en los hijos de Eli. «Sus hijos trajeron sobre sí una maldición, y él no los reprendió» (\ibibleverse{ISamuel}(3:13)). Si Elí no restringía a sus hijos en este punto de sus vidas, es probable que él nunca los había restringido. Sus vidas vulgares e inmorales fueron el resultado de la permisividad de su padre.

Padres, entonces, necesitan encontrar equilibrio entre demasiada dominación y demasiada permisividad. Haríamos las siguientes sugerencias:
\begin{enumerate}
\item \textbf{Ponga limites para tus hijos entre lo que se les permite hacer y lo que se les prohíbe hacer.} Asegúrate de que los hijos sepan dónde están los límites. Mantén esos limites consistentemente. 
\item \textbf{Permite que tus hijos muevan libremente dentro de los límites.} Si te enteras de que ellos hayan salido de los límites, responde con disciplina firma y apropiada. «Corrige a tu hijo y te dará descanso, y dará alegría a tu alma» (\ibibleverse{Proverbs}(29:17)).
\item \textbf{Cuando los hijos crecen, enséñalos que los limites que se han fijado realmente no son los limites de los padres, sino de Dios.} Háblales a menudo de Dios cuando tus hijos son jóvenes. 
«Y estas palabras que yo te mando hoy, estarán sobre tu corazón; y diligentemente las enseñarás a tus hijos» (\ibibleverse{Deuteronomy}(6:6-7)). Construye en ellos el deseo de agradarle a Dios, reforzado por un temor de salir de los límites.
\item \textbf{Comienza a «soltar las riendas» mientras los hijos se maduren y se capaciten de tomar decisiones.} Responsabilízalos por su mala conducta. No les quites muy rápido las consecuencias de decisiones malas. Después de todo, nosotros aprendimos de nuestros errores, y ellos deben ser permitidos a hacerlo, también. Pon confianza en ellos. Expresa confianza en su deseo de hacer el bien. Deja que sepan que tienes altas expectativas de ellos.
\item \textbf{Nunca jamás deles permiso a hacer lo malo.} Sus hijos probablemente harán lo malo en ocasiones, pero ellos deben siempre saber que Papá y Mamá se lastimarían profundamente si se enteraran. Nunca defiendas a tus hijos cuando están equivocados. Sí defiéndelos cuando están correctos. Nunca deles una razón para cuestionar tu amor por ellos.
\item \textbf{Ora que Dios de alguna forma anule tus errores, porque tú no cumplirás tu deber perfectamente.} Entonces agradécele por Su gracia y regocija en el crecimiento y desarrollo espiritual de tus hijos.
\end{enumerate}

\section{Dos Adolescentes Luchan Por La Madurez}
Dos adolescentes luchan por la madurez. \textbf{El primero} se ha criado desde la niñez a reconocer que hay un lado oscuro; que hay fracasos además de éxitos; que hay sacrificios y decepciones para experimentar además de gozos. Mientras sus padres han hecho lo que pudieron para suavizar los golpes de la vida, ellos han permitido que su hijo aprenda gradualmente que los golpes están allí y que se instruya por las experiencias para poder aguantarlos cuando vienen.

\textbf{El segundo adolescente} ha sido cuidadosamente protegido de las realidades oscuras de la vida. Él nunca ha visitado una funeraria, y nunca ha visitado a una persona con una enfermedad terminal. Sus padres le han enseñado poco acerca de sacrificar por el Señor. Cuando los conflictos han surgido entre sus propias actividades en que tiene interés y la obra del Señor, ellos han hecho la decisión por él, y buscando sobre todo la felicidad de su hijo, ellos le han permitido abandonar la obra del Señor por su propio placer. La decisión ni una vez se le ha explicado ni se le ha permitido tomar su propia decisión. Se le ha concedido prácticamente toda petición material por padres que no pueden aguantar ver a su hijo estar decepcionado. 

\textbf{El primer adolescente} tiene una ventaja considerable en el proceso de madurarse. Decisiones involucrando prioridades le son fáciles, porque él las ha estado tomando con la ayuda de la guía de sus padres desde la niñez. Las tentaciones siempre son difíciles y los deseos de la carne son fuertes, pero, después de todo, él nunca ha podido tener todo lo que él quería. Él es mucho mas capaz de aguantar los golpes de la vida, porque él ha tenido algún entrenamiento con eso. Enfermedad y muerte son una realidad para él, y él no se siente incomodo en su presencia. En pocas palabras, él ama el éxito y gozo, pero puede vivir con fracaso y tristeza. 

\textbf{El segundo adolescente} tiene que luchar mucho. Él se saca de repente de su ambiente albergado e irreal, y se hunde en un mundo cruel para lo cual no está preparado. Él se pone enojado y amargado cuando otros no le consienten. Él está devastado por el trato injusto que recibe en el trabajo. Él se siente muy incomodo cuando se habla de la muerte y retrocede de la realidad de ella. Él no se puede contentar con su salario anémico, pero cuando sus padres intervienen para ayudarle a financiar su estilo de vida acostumbrado, su dignidad recibe un golpe severo. Él se siente perdido. Intenta «encontrarse a sí mismo». Y, peor aún, su alma está en peligro, porque él nunca aprendió a organizar sus prioridades o tomar decisiones que son esenciales para el bienestar espiritual.

Este escritor preguntó una vez a un estimado amigo como uno puede construir fuerza de carácter en sus hijos. La única respuesta del amigo fue, «No lo sé, pero no es por darles todo lo que quieran». ¿Será posible que en nuestro gran «amor» por nuestros hijos nos hayamos convertido en sus peores enemigos?

\section{Dos Chicas Quieren Salir Con Jerry}
Jerry es un joven inteligente con un gran futuro. Él es un cristiano, verdaderamente dedicado a complacer el Señor y a prepararse para los cielos. Dos chicas reconocen estas grandes cualidades y quieren salir con Jerry, pero sus enfoques son considerablemente diferentes. 

\textbf{La primera chica} intenta atraer a Jerry por medios mundanos. Ella confía en su belleza física y en su ropa a la moda. Cuando él está cerca, ella enciende el encanto y es bastante atrevida en su presencia. Ella tiene su propio carro y busca utilizar ese artículo querido para la mayor ventaja posible. Ella es miembro de la iglesia y asiste regularmente, pero valores materiales obviamente pesan mas que valores espirituales en su vida. Ella tiene unas buenas cualidades, pero está enfatizando tanto su apariencia exterior que es difícil para uno penetrar el barniz para ver su verdadero carácter. 

\textbf{La segunda chica} no hace ningún esfuerzo obvio para atraer a Jerry, porque su «espíritu tierno y sereno» nunca la permitiría ser atrevida o coqueta. En su esfuerzo de complacer a Dios, ella busca desarrollar cualidades espirituales en su vida, y a ella le gustaría pensar que estas cualidades espirituales la harían atractiva a un joven espiritual. Porque ella es como Cristo, ella es cálida, amable, simpática, preocupada por los demás, el tipo de persona que uno puede sentirse cerca a ella, y ella manifiesta estas cualidades en la presencia de Jerry igual a lo que hace con todos sus conocidos. Ella no posee la belleza física de la primera chica, pero ella es ordenada y sana en apariencia y posee una hermosura por dentro, la cual «es preciosa delante de Dios» (\ibibleverse{IPeter}(3:4)). 

¿Con que chica es más probable que Jerry salga? No estamos seguros. Hemos visto buenos jóvenes que, halagados por la atención de chicas coquetas, han hecho malas elecciones. Y Jerry podría cometer ese error. Pero, conociendo a Jerry, estamos bastante seguros de que él distinguirá la chica que hará una buena cita de la que hará una buena esposa, y escogerá esta última. 

Dos cuestiones vitales se quedan para preguntarse. Chicas, ¿cuál de las dos chicas te representa correctamente a ti? Chicos, ¿cuál escogerías para una cita? Jóvenes que aman al Señor escogen parejas que les ayudan a ir a los cielos.

\chapter{AUTORIDAD}

\section{El Reloj De Un Hombre – La Vida De Un Hombre}
Yo miré mientras el hombre abría su regalo, un reloj hermoso de marca Seiko – digital, calendario, alarma, todo – todo, eso es, sino lo tranquilizantes él necesitaba después de intentar inútilmente a configurar la cosa, pobre hombre – todo el día estuvo tocando este botón y ese botón hasta que se quejó de su dedo dolorido. El día siguiente él se fue para el trabajo con su nuevo reloj Seiko a ver si alguien en el trabajo supiera configurarlo. Por supuesto en la caja, yaciendo totalmente ignorado e inobservado estuvo el juego de instrucciones que en simples pasos dijo como configurar el reloj. Pero el hombre prefirió intentarlo para sí mismo.

Lo que un hombre hace con su reloj no es de gran consecuencia. Pero muchas personas tristemente hacen con sus vidas lo que ese hombre hizo con su reloj. Ellos buscan aquí y buscan allá; experimentan con una religión, después otra; preguntan a padres, profesores, predicadores, amigos; leen toda la literatura disponible; haciendo todas de estas cosas en su búsqueda inútil de comunión con Dios y la vida eterna en los cielos. Mientras tanto, yaciendo en la mesa al par, ignorada e inobservada, está la biblia, las instrucciones de Dios para vida eterna y felicidad. 

«Tu palabra», dijo David, «lámpara es a mis pies… y luz para mi camino» (\ibibleverse{Psalms}(119:105)). La palabra de Dios provee instrucciones tocantes a salvación, morales, relaciones humanas, vida familiar, adoración, espiritualidad, comunión. Tiene el poder de transformar desgraciados miserables en siervos felices y útiles del Señor. Puede llevar la persona dispuesta a fe y esperanza y amor. La palabra de Dios es verdad (\ibibleverse{John}(17:17)). Capacita al hombre de Dios «para toda buena obra» (\ibibleverse{IITimothy}(3:16-17)). Cuan trágico es verla ignorada por las mismas personas que están buscando por vida e inmortalidad en tantas otras maneras. 

Pero, volviendo a mi amigo. Puede ser que él encuentre completamente por sí mismo como configurar ese reloj.  Él es bastante listo con tales cosas. Pero Dios \textbf{en su sabiduría} proveyó un camino de salvación que ningún hombre puede descubrir por su propia sabiduría e ingenio. Esto es el significado de \ibibleverse{ICorinthians}(1:21): «Porque ya que en la sabiduría de Dios el mundo no conoció a Dios por medio de su propia sabiduría, agradó a Dios, mediante la necedad de la predicación, salvar a los que creen». Si uno vendrá a Cristo, debe negarse a si mismo, quitar de consideración lo que le parece razonable a él, y aceptar humildemente la palabra de Dios y seguir sus instrucciones, incluso cuando el mundo entero lo ve como «necedad». Por esta razón no se llaman «muchos sabios conforme a la carne, ni muchos poderosos, ni muchos nobles» (\ibibleverse{ICorinthians}(1:26)), mientras la gente común Lo escuchan con gusto (\ibibleverse{Mark}(12:37)). La gente común simplemente no es tan dispuesta a intentar descifrarlo por sí mismos. 

Debemos quitar la conjetura de servir el Señor, y aprender de su libro de instrucciones lo que debemos hacer. «No depende del hombre su camino, ni de quien anda el dirigir sus pasos» (\ibibleverse{Jeremiah}(10:23). La eternidad está en juego. 

\section{Las Escrituras Son Nuestra Guía}
Dos perspectivas diferentes existen tocantes a como uno llega a un conocimiento de la voluntad de Dios. La primera perspectiva es que uno llega a este conocimiento por el leer y entender cuidadosamente las escrituras; que Jesús prometió a Sus \textbf{apóstoles} que ellos serían guiados a toda la \textbf{verdad} por el Espíritu (\ibibleverse{John}(14:26); \ibiblechvs{John}(16:13)); que ellos, junto con otros hombres inspirados, escribieron esa \textbf{verdad} en las escrituras; que cuando nosotros \textbf{leemos} lo que ellos escribieron, podemos «\textbf{comprender} [su] discernimiento del misterio de Cristo» (\ibibleverse{Ephesians}(3:3-4)); que las escrituras, como resultado, son una guía todo-suficiente desde la tierra hasta los cielos. 

La segunda perspectiva es que cada hijo de Dios se guía en alguna manera directa por el Espíritu en comprender la voluntad de Dios. La gente a menuda se escucha decir «Dios me está guiando a este entendimiento», o «en este camino», y en decir esto quieren decir que Él los está dirigiendo a través de alguna guía directa. Mientras ellos no ignoran de todo a las escrituras, ellos sienten que se estén guiados en alguna manera adicional a una comprensión de la voluntad de Dios, aplicando \ibibleverse{John}(14:26) y \ibibleverse{John}(16:13) a todo «creyente».

Este escritor confiesa tener la primera perspectiva y preguntaría a los que tienen la segunda perspectiva las siguientes preguntas:
\begin{enumerate}
\item Si, en verdad, todos los creyentes son guiados directamente a una comprensión de la voluntad de Dios, por qué era necesario para los primeros conversos seguir «continuamente a la enseñanza de los apóstoles» (\ibibleverse{Acts}(2:42))? ¿No hubieran tenido ellos la misma comprensión de la voluntad de Dios que tenían los apóstoles? 
\item ¿Cómo explicamos las diferencias en doctrina y practica que existen entre los que proclaman ser guiados en su comprensión directamente por el Señor? Diferencias abundan entre los que proclaman guía directa, mientras las escrituras enseñan solo «una fe» (\ibibleverse{Ephesians}(4:4-6)). ¿El Señor realmente está guiando a toda esta gente a ideas contrarias? ¿Es Él el autor de confusión (\ibibleverse{ICorinthians}(14:33))?
\item Si me pudieras comunicar con exactitud – o oralmente o escrita – esta comprensión a qué has sido guiado, ¿podría yo poner tanta confianza en ella como la pongo en las escrituras de Mateo, Marcos, Lucas, Juan, Pablo, etc.? ¿Podría yo seguir continuamente en tu enseñanza como los primeros cristianos hicieron en la enseñanza de los apóstoles? Si es así, ¿cómo sabría yo si debo seguir continuamente en tu enseñanza en lugar de la enseñanza de alguna persona cuya comprensión contradice la tuya? Con todos estos conflictos, ¿no tendríamos que volver a la biblia para saber lo correcto? ¿Y no haría eso, en realidad, llevarnos a la primera perspectiva declarada en este artículo, la cual ya acepto?
\end{enumerate}
La verdad es – las escrituras son la verdad divina de Dios (\ibibleverse{John}(17:17)). Uno puede leerlas y entenderlas (\ibibleverse{Ephesians}(3:3-4)). Ellas son todo-suficientes como guía de la tierra a los cielos (\ibibleverse{IITimothy}(3:16-17)). Ellas proveerán la basis de nuestro juicio en el día final (\ibibleverse{John}(12:48)). Léelas cuidadosamente y obedécelas en amor. 

\section{¿No Cambiarías?}
«Yo no cambiaría el sentimiento que tengo aquí en mi corazón por todas las biblias que pudieras amontonar en este edificio». Estas son las palabras un hombre usó una vez para describir su propia confianza en su salvación, una confianza basada en el sentimiento maravilloso que él tenia en su corazón. 

Este escritor, mientras tanto, acepta el punto de vista opuesto. Él se apoya sobre la biblia como su \textbf{única} garantía de salvación. De hecho, él no cambiaría la garantía él encuentra en su biblia por el sentimiento mas maravilloso que jamás se haya tenido por cualquier persona – ni por el testimonio \textbf{combinado} de sentimientos, predicadores, padres, e incluso los ángeles de los cielos.

Sabes, la biblia contiene las promesas de Dios, y nada es mas seguro que las promesas de Dios. Dios es Él que no puede mentir (Hebreos 6.18). Además, cualquier cosa que Él promete Él es capaz de cumplirlo (\ibibleverse{Romans}(4:21)). Igual la palabra de Dios como el poder de Dios entonces son absolutamente confiables. Cuando Dios promete por medio de su Hijo, «Él que crea y sea bautizado será salvo» (\ibibleverse{Mark}(16:16)), esa promesa no puede fallar.

\textbf{¿Sentimientos?} Sentimientos, al otro lado, pueden engañar. Saulo de Tarso es un ejemplo perfecto de uno que se engañó de tal manera, porque él sentía que le servía a Dios mientras él perseguía a cristianos (\ibibleverse{John}(16:2); \ibibleverse{Acts}(26:9-11)). Debe haber habido un tiempo cuando él tenia un sentimiento maravilloso en su corazón, sabiendo que él estaba avanzando en lo que él creía ser lo correcto mucho más rápidamente que sus contemporáneos (\ibibleverse{Galatians}(1:14)).

La gente perdida de \ibibleverse{Matthew}(7:22) sirve como otro ejemplo de sentimientos engañosos, porque ellos sentían que todo estaba bien porque habían profetizado en el nombre del Señor, habían echado fuera a demonios, y habían hecho muchas obras maravillosas. Sus sentimientos, \textbf{aunque apoyados por señales}, sin embargo, se comprobaron ser sentimientos falsos.

\textbf{¿Predicadores?} El testimonio de predicadores también es una garantía inadecuada de salvación, porque hay verdaderos maestros y falsos maestros. Jesús advirtió, «Cuidaos de los falsos profetas, que vienen a vosotros con vestidos de ovejas, pero por dentro son lobos rapaces» (\ibibleverse{Matthew}(7:15)). Uno no puede distinguir entre lo verdadero y lo falso en la basis de oraciones piadosas, una bonita personalidad, una disposición bondadosa, o la habilidad de citar escrituras de memoria, porque cualquiera de estos \textbf{puede} ser el vestido de ovejas de que habló Jesús. Uno puede distinguir entre ellos solo por la enseñanza de la biblia; pero esto nos vuelve a la verdadera basis de confianza de que se habló anteriormente en el artículo. 

\textbf{¿Ángeles?} La garantía de la biblia es aun más grande que la palabra de ángeles. «Pero si aun nosotros, o un ángel del cielo, os anunciara otro evangelio contrario al que os hemos anunciado, sea anatema» (\ibibleverse{Galatians}(1:8)).

La biblia es la palabra de Dios. Sus mandamientos se deben obedecer; sus promesas se deben creer; sus garantías se deben aceptar. Es el estándar. Todos los demás estándares fallarán. 

\section{Principios de Restauración en Las Epístolas de Juan}
Cambios inevitablemente ocurren con el paso del tiempo. Nuevos maestros se levantan, introduciendo doctrinas que son falsas, pero atractivas. Cada nueva generación tiende a ser mas sofisticada que la anterior, rechazando las practicas “anticuadas” de sus antepasados y adoptando nuevas ideas y prácticas. 

Esto no es un nuevo fenómeno. Cambios ocurrieron en el primer siglo como lo hacen en el siglo veinte. Como hay problemas entre la hermandad hoy en día, así había problemas entre la hermandad que afectaron los cristianos del primer siglo: la cuestión de la circuncisión, y hacía el final del primer siglo, gnosticismo, con sus varias doctrinas y morales corruptores. Para cuando escribió Juan sus epístolas, muchos cambios habían ocurrido desde Pentecostés en el pensamiento, morales, y actitudes de la gente. Mientras Juan aborda los cambios que habían ocurrido, él presenta tres principios que deberían guiarnos en manejar los cambios de nuestro día.
\begin{enumerate}
\item \textbf{Cuando cambios ocurren, debemos volver «al principio», no a lo que se ha aceptado tradicionalmente.} «Volver al principio» es la mera esencia de restauración, y es al principio que Juan guía sus lectores. Leemos de su pluma, «Amados, no os escribo un mandamiento nuevo, sino un mandamiento antiguo, que habéis tenido desde el principio» (\ibibleverse{IJohn}(2:7)); otra vez, «Porque este es el mensaje que habéis oído desde el principio» (\ibibleverse{IJohn}(3:11); ve también \ibibleverse{IIJohn}(1:5-6)). Juan además asegura a sus lectores que «si lo que oísteis desde el principio permanece en vosotros, vosotros también permaneceréis en el Hijo y en el Padre» (\ibibleverse{IJohn}(2:24)). Aceptación de lo que ha sido desde el principio es, entonces, esencial a aceptación con Dios.
\item \textbf{Cuando cambios ocurren, debemos volver a la fuente de verdad, no a altamente educados «clérigos» ni a «saberlo todo» dictadores en la iglesia.} De hecho, las escrituras de Juan borrarían toda distinción «clero-laico». Él no escribe a unos pocos eruditos entrenados en un seminario que, a su vez, deben interpretar sus escrituras para los laicos no entrenados. El se dirige a «hijitos» [RVR1960], «padres», y «jóvenes» (\ibibleverse{IJohn}(1:12-14)). MacKnight, en sus comentas sobre estos versículos, usa los términos «nuevos conversos», «viejos cristianos», y «cristianos vigorosos». Todos deben leer la carta de Juan, entenderla, y seguir su enseñanza. La gente nunca volverá a la verdad mientras ellos dejan que unos pocos hombres educados hagan todo su estudio y pensar por ellos. 

Juan enseña a sus lectores a «probad los espíritus para ver si son de Dios» (\ibibleverse{IJohn}(4:1)), en vez de seguir ciegamente a su enseñanza. La señora escogida debe determinar si un maestro «permanece en la enseñanza de Cristo» o no antes de extenderle hospitalidad y comunión (\ibibleverse{IIJohn}(1:9-11)). Gayo no puede ceder a los dictados del dominante Diótrefes (\ibibleverse{IIIJohn}(1:9-11)). El mensaje de Juan en todos estos pasajes es que todo cristiano debe leer, pensar, llegar a conclusiones sólidas, y estar firme, incluso cuando su firmeza lo lleva a un conflicto con las élites de la iglesia. Esta es la única manera de encontrar la verdad cuando cambios ocurren. 
\item  \textbf{Cuando cambios ocurren, debemos volver a los apóstoles, no a filosofías pretenciosas que pueden estar ganando popularidad a nuestro alrededor.} Las filosofías de los gnósticos eran impresionantes. Los que las apoyaron adoptaron un aire de superioridad. Pero Juan dice de los apóstoles: «Nosotros somos de Dios; el que conoce a Dios, nos oye; el que no es de Dios, no nos oye. En esto conocemos el espíritu de la verdad y el espíritu del error» (\ibibleverse{IJohn}(4:6)). «Vuelvan a los apóstoles», Juan está diciendo. Esta es la manera por la cual los espíritus se deben probar, y los falsos profetas se separan de los verdaderos. En elogiar a Demetrio, Juan dice, además, «tú sabes que nuestro testimonio es verdadero» (\ibibleverse{IIIJohn}(1:12)). «Volvemos a los apóstoles» hoy en día cuando vamos a las escrituras del Nuevo Testamento. Esta es nuestra única manera de saber «el espíritu de verdad y el espíritu de error» y de tomar una posición basada en testimonio que «es verdadero».
\end{enumerate}
Podemos diferir en nuestra estimación del «movimiento de restauración» del siglo 19 y de sus líderes, pero no podemos desviarnos del principio de restauración. Solo hay una solución para la división religiosa y doctrinas corruptoras de nuestro día, y esa solución es la misma que había en el día de Juan: volver al principio – volver a la fuente de verdad – volver a los apóstoles. ¡Aquí tomamos nuestra posición! ¡Aquí resistimos a todos los enemigos de verdad y derecho! ¡Aquí sabemos que de verdad somos de Dios!

\section{Es Original}
“Es original,” tú dices. ¿En serio? Puede ser que no sea mas original que un artículo que escribí hace poco por \textit{Perspectives}. Yo pensé que fuera original, pero después me topé con un artículo parecido por Ralph Williams que se había escrito mucho antes de mi artículo, lo cual sin duda había influenciado mi pensamiento y se había convertido en una parte de mí, tanto que yo estaba seguro de que esos pensamientos fueran míos. ¡Mis pensamientos originales!

O puede ser no mas original que unos de esos pensamientos originales (?) que Sewell y yo ideamos para sermones, solo para aprender después que nuestro papá ha presentado esos pensamientos en sermones desde que éramos niños. De hecho, me estoy preguntando ahora quien ha escrito un artículo sobre el tema “Es Original.”

Somos bendecidos con una herencia rica. Hemos escuchado a grandes hombres predicar grandes sermones, sacando casi todo pensamiento imaginable del mejor libro que jamás se haya escrito. La mayoría de los sermones se nos han olvidado. Tampoco somos capaces de relacionar pensamientos específicos a los primeros hombres que escuchamos predicarlos. Pero esos pensamientos se convirtieron en una parte de nosotros, y esos sermones, poco a poco, cambiaron nuestras vidas y nuestros pensamientos para llevarnos a nuestras convicciones presentes y a nuestro estado ante Dios. Estamos endeudados a muchos, pero especialmente a Dios, en quien toda la verdad tiene su origen (\ibibleverse{John}(17:17)). 

Pero una advertencia se necesita ahora. Porque, mientras estamos agradecidos por las verdades que se han enseñado a través de los años, nunca debemos aceptar cualquier enseñanza como verdad solo porque hermanos “siempre” lo han creído y enseñado. La humildad demandaría que seamos lentos de rechazar tal enseñanza, pero la verdad se determina solo por una examinación de la enseñanza bíblica. “Tu palabra es verdad” (\ibibleverse{John}(17:17)). “Si alguien habla, que hable conforme a las palabras de Dios” (\ibibleverse{IPeter}(4:11)). La vieja y aceptada enseñanza se debe examinar, no solo por cada generación, sino por todo individuo dentro de esa generación, con el mismo cuidado con que cualquier nueva enseñanza se examina. 

Independencia de pensamiento no demanda un rechazo de vieja y aceptada enseñanza, pero sí demanda una examinación cuidadosa de esa enseñanza en la luz de las escrituras. 

\section{El Poder De Lo Sencillo}
Jesucristo tuvo un aprecio profundo por las cosas sencillas. Su \textbf{enseñanza} era profunda, pero siempre sencilla. El alcanzó los corazones de sus oyentes, no con jerga filosófica y pretenciosa, sino con ilustraciones y yendo directo «al grano» en su enseñanza. El pudo ver en un granjero sembrando su semilla, o un lirio enseñando su belleza, o un pastor dejando su rebaño a buscar una oveja perdida, o un padre amoroso dando la bienvenido a un hijo díscolo, una lección que podría enseñar alguna verdad espiritual. 

Sus \textbf{apóstoles} se escogieron de la clase humilde. El pudo apreciar a las personas, no por lo que ellos poseían, sino por lo que eran; y, en unos casos, no por lo que eran, sino por lo que podrían ser. El reconoció verdadera cualidad, y verdadera cualidad a menudo se encuentra en lo sencillo y humilde. 

La \textbf{adoración} Él ordenó era sencilla en su naturaleza. «Y el primer día de la semana, cuando estábamos reunidos para partir el pan, Pablo les hablaba…» (\ibibleverse{Acts}(20:7)). Incluso los más pobres podían adorar, porque todo lo que se requirió de una naturaleza material era un poco de pan y fruto de la vid. Los de poco talento podían adorar, porque Dios escuchaba en vista del corazón en vez de la belleza de la voz.

Él autorizó una \textbf{organización} simple para su iglesia, con cada congregación nombrando sus propios obispos y diáconos (\ibibleverse{Philippians}(1:1)). No había asociaciones, conferencias o sínodos denominacionales. No había organizaciones o sociedades entre iglesias. Sin embargo, a través de la organización sencilla dada a la iglesia por el Señor, el mundo del primer siglo se evangelizó plenamente y los necesitados entre ellos fueron provistos. 

El Señor sabía que el éxito en Su obra no se realizaría por complejidad de organización, sino por dedicación, fe, y un compromiso de parte de sus seguidores. Cometemos un error terrible cuando intentamos sustituir el primero por el ultimo. 

¿Por qué esta sencillez? «Para que nadie se jacte delante de Dios» (\ibibleverse{ICorinthians}(1:29)). Los sistemas complejos que los hombres diseñan tienden a traerles gloria a ellos mismos en lugar de a Dios. 

Para volver a la sencillez que nuestro Señor ordenó puede que no sea muy impresionante a los de mente mundana, pero el mismo Jesús no es muy impresionante a los de mente mundana. Además, nuestro propósito no es impresionar a los de mente mundana sino complacer a Dios e inclinarnos en sumisión a Su voluntad. Que descartemos nuestros super proyectos y sistemas complejos. Que aprendamos a apreciar enseñanza sencilla y maneras sencillas. Sobre todo, que aprendamos a apreciar enseñanza \textbf{bíblica} y maneras \textbf{bíblicas}.

Nos gusta la siguiente cita de Ed Harrel: «Que necios somos para pensar que Dios estará impresionado con nuestras voces cuando cantamos; después de todo, ¡Él escucha a los ángeles cantar! Que necio pensar que Él estará impresionado con nuestras catedrales; acuérdate, ¡Él hizo el Gran Cañón!». Lo que Él está buscando es un corazón puro, amoroso, y obediente a Su voluntad. Y eso es sencillo.

\chapter{DIOS}

\section{Dios-Consciencia}
Un hombre verdaderamente piadoso es uno que vive con una consciencia constante de la presencia divina de Dios. El es Dios-consciente. Cuando él se despierta en la mañana, allí está Dios. Mientras se viste para el trabajo, allí está Dios. Mientras él entra a desayunar con su familia, mientras él trabaja por el día, mientras él maneja a casa, mientras él pasa las horas de la tarde, mientras él se acuesta en su cama al final del día, allí está Dios. 

Enoc era un hombre que era Dios-consciente, porque él «anduvo con Dios» (\ibibleverse{Genesis}(5:24)). Él disfrutó la compañía constante de Dios. A dondequiera que Enoc fue, Dios fue allí con él, y Enoc siempre estaba consciente de que Él estaba allí. Él no pudo huir de la presencia de Dios (\ibibleverse{Psalms}(139:7)), ni buscó hacerlo. Él era un hombre piadoso.

Cuan afortunado es aquel hombre que haya desarrollado dentro de sí esta Dios-consciencia. Es fácil orar para él, porque Dios es para él un compañero cercano, siempre cerca, cuyos «oídos [están] atentos a sus oraciones» (\ibibleverse{IPeter}(3:12)). Hablar con Dios es tan natural para él como hablar con cualquier compañero.

Él no tiene miedo, porque él simplemente pone su mano en la de Dios en tiempos difíciles. «Dios es nuestro refugio y fortaleza, nuestro pronto auxilio en las tribulaciones. Por tanto, no temeremos\ldots» (\ibibleverse{Psalms}(46:1-2)). Incluso cuando pase «por el valle de sombra de muerte», él puede «no temer mal alguno», porque Dios está con él. 

El poder de tentación grandemente se disminuye, porque él nunca olvida que «todas las cosas están al descubierto y desnudas ante los ojos de aquel a quien tenemos que dar cuenta» (\ibibleverse{Hebrews}(4:13)). Su deseo de complacer a su Dios siempre-presente es mayor que el poder de tentación. 

Él es agradecido, reconociendo que Dios, con quien anda, es la fuente de «toda buena dádiva y todo don perfecto» (\ibibleverse{James}(1:17)).

Él ama a Dios. Él habla a Dios. Él anda con Dios. Él siempre está consciente de la presencia de Dios. Él nunca está sin Dios. Aun así, esta relación nunca degenera a una relación de «amiguitos», porque él reverencia a Dios; él reconoce su grandeza; él reconoce con gratitud su propia indignidad de tal relación con Dios Todopoderoso. 

Esta es la mera esencia de piedad. Alguien, hace años, observando que en la lengua inglesa la palabra por piedad, godliness, y la frase «ser como Dios», God-like-ness, son muy parecidas, supuso que significan lo mismo. Esa suposición falsa se pasó a otros, y ahora tiene una posición fuerte en el pensamiento de un gran número de personas anglohablantes. W.E. Vine dice que piedad «denota esa devoción que, caracterizada por una actitud orientada hacia Dios, hace lo que le complace a Él». Una persona piadosa, entonces, es uno que tiene una actitud orientada hacia Dios, y cuya consciencia constante de Dios lo guía a serle obediente. 

Mientras visitábamos un hospital, hace poco, observamos este letrero, «¿Has dicho “Gracias, Dios” hoy?». Una persona piadosa probablemente lo hubiera hecho. ¿Has dicho \textbf{tú}, «gracias, Dios» hoy?

\section{Mi Oración}
Señor, ayúdame a nunca ser insensible de corazón. Ayúdame a ser sensitivo a las penas y los dolores de otros. 

Déjame sentir una emoción genuina en ocasiones genuinamente emocionantes: cuando dos cristianos jóvenes se juntan en matrimonio; cuando algún joven predica su primer sermón del evangelio, o cuando algún nuevo converso dirige su primera oración; cuando un recién nacido llora, o cuando algún anciano exitosamente pasa un hito mas en su vida; cuando un pecador responde para ser bautizado, o cuando un hermano vuelve.

Señor, yo no quiero cerrar de mi propia vida los pesares de otros. Hazme llorar cuando deben haber lagrimas: cuando un doctor acaba de dar un reporte espantoso; cuando padres se entristecen sobre su hijo rebelde; cuando lagrimas fluyen de los ojos de un niño sin madre, o de una madre que ha perdido sus hijos; cuando un hermano o hermana débil ha vuelto a caer en pecado; cuando un hombre acaba de ser abandonado por la esposa de su juventud; cuando parece que toda esperanza se haya ido dentro de los que habían esperado.

Ayúdame a estar mas emocionado sobre los éxitos de otros: cuando alguien logra algo que yo nunca he podido lograr; cuando algún hombre digno se nombra anciano de la iglesia; cuando el hijo de alguien se gradúa con honores; cuando alguna pareja se muda a esa casa por la cual esperaron mucho tiempo; cuando la habilidad de predicar de algún joven obviamente ha superado la mía. Sobre todo, Señor, no me dejes ser celoso. 

Señor, ayúdame a ser como Ti: compasivo, bondadoso, misericordioso, amable, tardo para la ira, abundante en misericordia, simpatizando con las debilidades de otros, capaz de llorar, listo para regocijar, emocional, y amoroso. Déjame negarme a mi mismo, y considerar al otro como más importante que a mi mismo. 

Y, Señor, nunca me dejes ser endurecido al pecado. Ayúdame a odiar el pecado, a llorar lagrimas cegadoras sobre mis propios pecados, a mantener una consciencia limpia de ofensas. Que yo este repulsado y apenado sobre los pecados de otros. 

Señor, ayúdame a nunca ser insensible de corazón, porque si así fuera, yo ya no podría servirte efectivamente a Ti ni a mis prójimos, y daría por vencida toda esperanza de verdadera alegría, igual en esta vida como en la venidera. 

\section{Ante Dios – Ante Los Hombres}
«Cuidad de no practicar vuestra justicia delante de los hombres para ser vistos por ellos; de otra manera no tendréis recompensa de vuestro Padre que está en los cielos» (\ibibleverse{Matthew}(6:1)). Justicia verdadera es principalmente «Dios-consciencia» en lugar de «hombre-consciencia». Dios se complace mientras cantamos sus alabanzas, o enseñamos su verdad, o dirigimos en oración, o ayudamos a los necesitados, o contribuimos para apoyar Su obra, si nuestro propósito es ganar Su aprobación y traer gloria a Su nombre. Ay de esa persona que canta por el propósito de exponer su bella voz. Ay de esa persona que busca la adulación de los hombres mientras dirige en oración. Ay de ese predicador que habla para satisfacer el «comezón de oídos» de sus oyentes. Ay de esa persona que contribuye para ser visto de los hombres. Cuando él gana sus adulaciones, él «ya ha recibido su recompensa»; nada le espera de su Padre en los cielos. 

De acuerdo con esta enseñanza, la Biblia revela como Dios juzgó a dos parejas diferentes, una disfrutando de la aprobación de Dios, la otra sufriendo su desaprobación. 

Dios desaprobó de Ananías y Safira (\ibibleverse{Acts}(5:1-11)). Su desaprobación no fue a causa de la cantidad de la ofrenda de ellos. Ellos habían traído una ofrenda muy generosa. Ellos incluso habían vendido una posesión para poder dar, y mientras no sabemos la porción que trajeron, ellos obviamente pensaron que sería suficiente para impresionar a los apóstoles. Pero aquí radica la clave de su verdadero problema: Ellos eran más conscientes de la reacción \textbf{de los hombres} a su benevolencia que \textbf{de Dios}. Si hubieran sido conscientes de Dios en lo que hicieron, y si hubieran buscado Su aprobación, no hubieran mentido. Pero ellos practicaron su justicia «\textbf{delante de los hombres} para ser vistos por ellos», y en su preocupación por impresionar a los hombres, mintieron con respecto a la cantidad que trajeron.

En contraste a Ananías y Safira, hubo Zacarías y Elizabet, de quienes se dijo, «Ambos eran justos \textbf{delante de Dios}» (\ibibleverse{Luke}(1:6)). Aunque muchos, sin duda, observaron su justicia, y como resultado glorificaron al Padre en los cielos (\ibibleverse{Matthew}(5:16)), Zacarías y Elizabet obviamente no eran tan preocupados por la aprobación de los hombres como lo eran para la aprobación de Dios. Fue la aprobación de Dios que ellos buscaron; fue Su aprobación que ganaron.

Dios bendijo a Zacarías y Elizabet, escogiéndolos a ellos para ser los padres de Juan, el precursor de Cristo. Él castigó a Ananías y Safira con muerte inmediata, y, de acuerdo con la declaración de Jesús en \ibibleverse{Matthew}(6:1), ellos «no tienen recompensa de vuestro Padre que está en los cielos».

La hipocresía es aborrecida por el Señor. Que una persona exteriormente parezca religiosa y sincera mientras interiormente solo desee la adulación y aprobación de los hombres es ser culpable de hipocresía repugnante. Que siempre busquemos practicar nuestra justicia delante de Dios para ser visto de Él. La gloriosa y eterna «recompensa del Padre» no se puede comparar con la adulación inconstante y efémera de los hombres.

\section{No La Nuestra – Sino Hechura Suya}
«Porque somos hechura suya, creados en Cristo Jesús para hacer buenas obras, las cuales Dios preparó de antemano para que anduviéramos en ellas» (\ibibleverse{Ephesians}(2:10)).

Cuando uno se bautiza él se convierte en una nueva creación, pero él no es la creación de cualquier hombre. Él es la hechura de Dios.

Él no es la hechura de la persona que lo convirtió – no principalmente, de todos modos. Los hombres pueden ensenar, influir, persuadir, y bautizar; pero solo Dios puede limpiar, perdonar, levantar a una persona a sentarse con Cristo en los lugares celestiales, y darle vida. Él es la creación de Dios – la hechura de Dios. Igual de seguro que ningún hombre puede crear un «Adán», así de seguro ningún hombre puede crear una nueva criatura en Cristo.

Tampoco es que uno se crea a sí mismo. En Cristo, uno no se levanta por su propio esfuerzo. La cristiandad no es una religión «bricolaje» – no en el sentido mas pleno. Uno no efectúa su propia salvación por su propio mérito. Al contrario, en obediencia al evangelio y en fidelidad como cristiano, él se presenta como barro en la mano de Dios, para llegar a ser la obra del Alfarero divino, quien le moldea, forma, y perfecciona para que sea elaborado en la imagen de Su Hijo.

Él es la hechura de Dios porque su salvación es «por gracia… por medio de la fe… don de Dios» (\ibibleverse{Ephesians}(2:8-9)). Si la salvación de uno fuera de obras meritorias, él no sería la hechura de Dios. Esto es la idea principal del pasaje. 
La hechura de Dios existe como un monumento a la grandeza de un alfarero; como una pintura hermosa es un monumento a un artista, así un cristiano maduro y perfeccionado es un monumento al poder maravilloso de Dios. Tal persona es un producto de la gracia de Dios y existe para «la alabanza de Su gloria» (\ibibleverse{Ephesians}(1:6, 12, 14); \ibiblechvs{Ephesians}(3:14-19)). Que Dios podría tomar un Pedro, un Juan, un Saulo de Tarso, un Aquila, una Priscila, un Juan Marcos, y moldearle a él o a ella en la vasija preciosa que cada uno se hizo es una manifestación de Su grandeza. Que Él podría hacer lo mismo con personas conocidas y observadas por este autor igualmente manifiesta Su grandeza. Que Él puede y sí hará lo mismo para mí si yo solo me someto a Su cuidado en humilde obediencia, confianza, y oración es la maravilla mas grande de todo. «Estoy tan alegre de que Jesús me ama… Jesús me ama incluso a mí» (P.P. Bliss).

La hechura de Dios se debe tratar con cuidado. Tal persona es especial, preciosa, e inestimable para Dios. Como uno tiene cuidado en tratar a una reliquia de familia o una rara pieza de cerámica elaborada por las manos de un maestro, así él debe tener cuidado en su trato de esa creación, que es la obra de Dios. «No destruyas la obra de Dios por causa de la comida», Pablo advirtió a los romanos (\ibibleverse{Romans}(14:20)). Esa persona hacía cuya consciencia tierna estás mostrando poca consideración o cuya alma estás poniendo en peligro es la obra de Dios. Ama a esa persona. Aprécialo. Seas tierno con él. Reconoce su valor. ¡Maneje con cuidado!

La hechura de Dios nunca debe estar contento hasta que se haya llevado a la perfección. «Termine, entonces, Su nueva creación», Charles Wesley escribió en su himno famoso, «Amor Divino». De acuerdo con esto, Pablo pudo expresar su confianza en los cristianos filipenses, «que el que comenzó en vosotros la buena obra, la perfeccionará hasta el día de Cristo Jesús» (\ibibleverse{Philippians}(1:6-7)). Solo los que siguen en las manos del Alfarero hasta ser llevados a la finalización y perfección se convierten en vasijas de honor. Todos los demás se estropean y son aptos solo para destrucción (\ibibleverse{IITimothy}(2:19-21)).

Que ningún hombre, entonces, se jacte de sí mismo. Justicia propia no tiene lugar en el corazón de un cristiano. Si alguno se jacta, «que se gloríe en el Señor» (\ibibleverse{ICorinthians}(1:31)) y «en la cruz de nuestro Señor Jesucristo» (\ibibleverse{Galatians}(6:14)). Somos hechura Suya.

\section{No La Nuestra – Sino Fuerza Suya}
«Por lo demás, fortaleceos en el Señor y en el poder de su fuerza» (\ibibleverse{Ephesians}(6:10)).

En nuestra batalla contra Satanás, no podemos alcanzar la victoria en la basis de nuestra propia fuerza. Debemos confiar en la fuerza y poder del Señor.
\begin{flushleft}
\begin{verse}
Nuestra fuerza nada puede\\
Pronto somos a perder\\
El Justo por nosotros lucha\\
Él que Dios ha elegido\\
¿Preguntas quien es Él?\\
Él es Jesucristo\\
Señor de ejércitos\\
No existe otro Dios\\
Él gana la batalla. (Martin Lutero)
\end{verse}
\end{flushleft}

La razón por la que no podemos ganar en la basis de nuestra propia fuerza es a causa de la naturaleza del enemigo. «Nuestra lucha no es contra sangre y carne» (\ibibleverse{Ephesians}(6:12)). Si fuera nuestra propia fuerza contra la fuerza de otro hombre, puede ser que venciéramos solos. El enemigo, sin embargo, es un enemigo espiritual y un enemigo espiritual se puede vencer solo a través del poder de la fuerza de Dios. 

Como Confiar en la Fuerza del Señor
\begin{enumerate}
\item \textbf{Por armarnos con la armadura} que Él provee: salvación como nuestro yelmo, justicia como nuestro peto, verdad como nuestro cinturón, la preparación del evangelio de paz como nuestro calzado, fe como nuestro escudo, y la palabra de Dios como nuestra espada. Hay los que confían en filosofía humana, pensamiento positivo, meditación transcendental, monacato, privación de sí mismo, etc., por su armadura. Todas tales armas carnales fallarán. 
\item \textbf{Por oración constante al Señor por ayuda.} «Con toda oración y súplica orad en todo tiempo en el Espíritu…» (\ibibleverse{Ephesians}(6:18)). El Señor nos ayuda en tantas maneras. Él nos protege de tentación (\ibibleverse{Matthew}(6:13)). Él suaviza la tentación cuando viene (\ibibleverse{ICorinthians}(10:13)). Él trae buenas influencias a nuestras vidas (\ibibleverse{Romans}(1:12)). Él perdona cuando estamos vencidos (\ibibleverse{IJohn}(1:7, 9)). Pero Él no quiere que le demos por hecho. Debemos pedirle al Señor por ayuda y agradecerle mientras Él nos lleva con seguridad por medio de cada batalla. 
\item \textbf{Por plena esperanza de victoria.} Que ninguno entre en esta batalla con una actitud derrotista. La victoria es segura para todos los que confiarán en la fuerza del Señor. Las palabras «para que podáis» o «que podéis» aparecen tres veces en el pasaje de \ibible{Ephesians}(6:11, 13, 16)Efesios 6 (vv. 11, 13, 16). El Señor nos enseña el enemigo, y es como si nosotros, impresionados con la obvia fuerza del enemigo, comenzáramos a desesperarnos; entonces el Señor dice «Lo puedes vencer; aquí está tu armadura, vístete con cada pieza; quédate cerca de Mí; si te derrumban, Yo te levantaré; solo sigue peleando; persevera; mira; ¡la victoria es tuya!».
\end{enumerate}
\textbf{Estén, Pues, Firmes}

Debemos, sin embargo, estar firmes, si quisiéramos ganar la batalla. ¡Estén firmes! ¡Esten firmes! ¡Esten firmes! La frase aparece tres veces en el texto (\ibibleverse{Ephesians}(6:11, 13, 14)). La mejor armadura en el mundo es do poco valor si hay un cobarde adentro. No es inusual encontrar un hombre que parece estar bien equipado para la batalla. Él conoce las escrituras, ha estudiado el significado de las escrituras, ha memorizado extensivamente, se mira todo un gran soldado – pero cuando se presenta el primer desafío, él comienza a sudar, duda, vacila, y transige hasta que un observador realmente no sabe en qué lado él está. Él no está firme. Tal persona enfrenta un derroto seguro. La victoria es segura, pero solo para los que estarán firmes. 

La batalla ruge. La lucha es grande. El enemigo es formidable. Hay mucho en juego – la eternidad, de hecho. Gracias al Señor – la esperanza de victoria no depende solo de la nuestra, sino de Su fuerza y poder. 

\section{¿Tenemos Que Ver La Respuesta?}
«Cómo puedo estar seguro qué Dios está contestando a mis oraciones?», alguien pregunta. «Cómo puedo precisar positivamente este evento o ese evento como su respuesta final y completa?». Estamos escuchando tales preguntas bastante frecuentemente estos días.

Uno que siente la necesidad de precisar la respuesta a su oración, y cuestiona si su oración fuera contestada a menos que lo puede hacer, puede tener un problema con su fe. Él puede querer andar por vista en vez de por fe. Oración aceptable es un resultado de confianza completa en Dios, una confianza que no tiene que ver e identificar.

Un empleador tiene un empleado en quien él tiene completa confidencia y confianza. Un problema bastante complejo se levanta. Tiempo, pensamiento e investigación considerable serán requeridos si el problema se va a llevar a una solución razonable. Pero conociendo la competencia de su empleado y teniendo completa confianza en él como un hombre responsable, de lealtad e integridad incuestionable, el empleador llama al empleado, le presenta con el problema, le entrega el asunto entero, y ni pregunta acerca del resultado. Tales empleados confiables son raros, obviamente, pero sí existen. 

Nuestro punto es este. Si un empleador puede poner esta clase de confianza en un empleado confiable, pero falible, ¿no será que nosotros podemos poner una confianza igual – no, ¡infinitamente mayor! – en nuestro Padre celestial e infalible? ¿No podemos aprender a poner nuestros problemas y cargas sobre Él, y saber que Él los llevará a la mejor solución posible según Su voluntad? ¿Y no podemos llegar a creer que Él lo puede hacer sin nuestra capacidad de ver, precisar e interpretar el resultado nosotros mismos? ¿No será posible que el pleno resultado no se sentirá hasta muy allá en el futuro, mucho después de que nosotros nos hayamos ido de este mundo lleno de pesares? «echando toda vuestra ansiedad sobre Él, porque Él tiene cuidado de vosotros» (\ibibleverse{IPeter}(5:7); ve también \ibibleverse{IJohn}(5:14-15)).

Las palabras «mejor solución posible» son palabras claves en este punto, sin embargo. Debemos tener fe, no solo que Dios actuará, sino que Él actuará para llevar a cabo la mejor solución posible, incluso si esa solución sea contraria a lo que nosotros desearíamos o buscaríamos. Debemos estar dispuestos a entregar nuestros problemas a Dios para hacer con ellos lo que Él quiera, y después decir en sumisión total a Su voluntad, «El Señor es; que haga lo que bien le parezca.» (\ibibleverse{ISamuel}(3:18)). Jesús manifestó esta clase de confianza y sumisión (\ibibleverse{Matthew}(26:39-44)). Así también los grandes caracteres del Antiguo Testamento (\ibibleverse{Hebrews}(11:)). Así también los apóstoles (\ibibleverse{IICorinthians}(4:13-14)). Y así debemos ser nosotros.

Maravillosamente bendecida es la persona que tiene esta clase de fe. Mientras otros se inquietan y se preocupan, él deja que sus ansiedades y peticiones «sean dadas a conocer… delante de Dios» (\ibibleverse{Philippians}(4:6)). Mientras otros se llenan de confusión y miedo, él posee «la paz de Dios, que sobrepasa todo entendimiento…». «Pero que pida con fe, sin dudar…» (\ibibleverse{James}(1:6)).

\chapter{LA VIDA CRISTIANA}

\section{Diariamente}
Un hombre al que se le pidió que describiera sus memorias de sus días de la universidad hace veinticinco años replicó, «Unos pocos momentos grandes; muchas metidas de pata; pero, por lo general, memorias placenteras.»
¿No describirían estas palabras a nuestras memorias de la vida en general? Mientras recordamos nuestros años, siempre hay los momentos «grandes», los «alturas» emocionales, en que nos gusta meditar. También hay las «metidas de pata», las ocasiones vergonzosas que simplemente siguen persiguiéndonos. Pero, a través de todo, las memorias placenteras prevalecen para que nos sintamos generalmente bien acerca de la vida. 

Pero, en realidad, el éxito o el fracaso en la vida no se determina por los «momentos grandes» ni por las «meteduras de pata». No estaremos salvados en la basis de unos pocos logros espirituales ni perdidos eternamente en la basis de unos pocos errores repugnantes (suponiendo que haya arrepentimiento). La vida consiste en acciones y decisiones cotidianas, y son estas que traen o el éxito o el fracaso final, felicidad eterna o condenación eterna. «Si alguno quiere venir en pos de mí, niéguese a sí mismo, tome su cruz cada día y sígame» (\ibibleverse{Luke}(9:23)).

Es una cosa expresar preocupación por nuestros niños mientras hablamos de su futura espiritualidad y fidelidad. Es otra cosa proveer diariamente un ambiente espiritual en el hogar, un buen ejemplo de piedad y fidelidad, disciplina consistente y amorosa, y un amor por Dios y respeto por el prójimo que son tan esenciales para el entrenamiento de nuestros hijos. Son las impresiones pequeñas que se hacen día a día que resultan ser tan decisivas. 
Es una cosa soñar con ser nombrado anciano en la iglesia algún día. Es otra cosa esforzarse diariamente a aprender las escrituras, desarrollar la habilidad de liderazgo, crecer espiritualmente, y vivir para ganar la confianza de una congregación entendida. Uno no se clasifica para el ancianato en una sola brinca grande. Viene por desarrollo cotidiano. 

Es una cosa hablar bien de prioridades. Es otra cosa ponerlo a Dios en primer lugar todos los días. El diablo sabe tantas maneras de poner nuestra resolución a la prueba en estos ámbitos. Nuestras intenciones son buenas, pero por medio de su sutileza nos tiene vendiendo nuestras almas por una olla de potaje o treinta piezas de plata. 

Es una cosa pensar que \textbf{moriríamos} por el Señor si nuestra fe así se probara. Es otra cosa vivir verdaderamente por Él de una forma cotidiana. Los egos se pueden alimentar de los «momentos grandes», pero verdadera espiritualidad se desarrolla por oración, estudio y meditación diaria.

Nuestra lección es esta. Establece tus metas para el futuro, y que sean altas. Pero reconoce que son los momentos pequeños y cotidianos, los momentos que a menudo se nos olvidan, acumulándose a través de los años, que realmente forman nuestro destino. El éxito de mañana depende de las elecciones y decisiones que se toman hoy. Tómalas con cuidado.

\section{Dominio Propio}
¿Qué es el valor de dominio propio? Dominio propio es lo que nos capacita a morder nuestras lenguas cuando estamos tentados a bruscamente poner a alguien en su lugar de una vez por todas… o cuando sabemos un sabroso chismorreo que sería muy entretenedor para el grupo o nos convertiría en «el alma de la fiesta» … o cuando una ocasión casi demanda que traicionemos una confianza que no se debe traicionar bajo cualquier circunstancia. 

Dominio propio es lo que nos capacita a controlar nuestras pasiones cuando otra persona nos provoca al enojo… que guarda los puños cerrados dentro de los bolsillos cuando el agitador es solo la mitad de nuestro tamaño… que mantiene a los labios sellados cuando otro nos está gritando y maldiciendo. Dominio propio es lo que nos capacita a ser como nuestro Señor «quien cuando le ultrajaban, no respondía ultrajando; cuando padecía, no amenazaba, sino que se encomendaba a aquel que juzga con justicia» (\ibibleverse{IPeter}(2:23)).  

Dominio propio es lo que nos capacita a mantener pureza de corazón y rechazar pensamientos malvados antes de que echen raíces… que nos capacita a pensar lo mejor de las acciones de otra persona cuando chismes no probados fácilmente podrían destruir nuestra confianza en él… que nos ayuda a mantener una disposición feliz cuando todo a nuestro alrededor se ha vuelto agrio. Dominio propio es lo que nos capacita a poner «todo pensamiento en cautiverio a la obediencia de Cristo» (\ibibleverse{IICorinthians}(10:5)). Dominio propio es lo que nos capacita a amar a los antipáticos y a odiar lo que ama el mundo. 

Dominio propio es lo que nos capacita a gobernar nuestros apetitos… a decir «no» cuando nuestras codicias nos llevarían al pecado o cuando lo que es dañino para la salud se nos presenta. Dominio propio es lo que capacita al fumador a botar sus cigarrillos y al alcohólico a botar su alcohol y jamás volver a ello. Dominio propio es lo que nos capacita a gobernar en vez de ser esclavizados. 

La biblia no glorifica al indiferente e impasible. No es nuestra meta ser indiferente. Para ser como Pablo es necesario que nuestros espíritus puedan ser conmovidos dentro de nosotros cuando estamos rodeado de maldad (\ibibleverse{Acts}(17:16)). Para ser como nuestro Señor debemos sentir enojo a veces cuando estamos rodeado de santurronería hipócrita (\ibibleverse{Mark}(3:5)); incluso debemos reaccionar con arrebatos de bondad en ocasiones, como cuando el Señor limpió el templo (\ibibleverse{John}(2:13-17)). Pero todos tales arrebatos se deben controlar con dominio propio, para que en nuestro enojo no pequemos (\ibibleverse{Ephesians}(4:26)).

Dios no se divierte viendo nuestras acciones descontroladas. Nuestros berrinches y bruscas, desenfrenadas palabras ponen el alma en peligro y pueden prohibirnos de recibir la entrada amplia al reino eterno del Señor (\ibibleverse{IIPeter}(1:5-11). No debemos minimizar el peligro. No debemos darnos por vencidos ante esta maldad. 

¿Qué es el valor de dominio propio? Es una de las cualidades que nos capacitan para ir a los cielos. El que lo tiene es verdaderamente rico.

\section{Arrebatos de Bondad}
«Mas el fruto del Espíritu es amor, gozo, paz, paciencia, benignidad, bondad…» (\ibibleverse{Galatians}(5:22)). En su definición de «bondad», Trench sugiere que la palabra incluye las «cualidades mas severas por las cuales hacer lo bien a otros no es siempre por medios suaves» (\textit{Diccionario Expositivo de Palabras del Nuevo Testamento}, W.E. Vine, p. 165). Dos ejemplos en la vida de Cristo se dan para ilustrar este tipo severo de bondad: Su purificación del templo (\ibibleverse{Matthew}(21:12-13)), y su denunciación de los escribas y fariseos (\ibibleverse{Matthew}(23:13-33)). Supongo que podríamos llamar tales acciones «arrebatos de bondad».

Trench puede estar o correcto o incorrecto en su definición, pero estamos bastante seguros de que unos arrebatos de bondad se necesitan en nuestro día. Un movimiento rápido hacia el mando de apago en nuestras televisiones, acompañado de unas bien-escogidas palabras de desaprobación, puede ayudar a nuestros hijos a entender que nosotros odiamos la suciedad, la profanidad, lo lascivo que tan a menudo se presenta. ¿O será posible que hayamos estado expuestos por tanto tiempo a esta maldad, que realmente nos hayamos acostumbrado a ella, ya no estemos repulsados por ella?

Nos preguntamos cuantas iglesias se hubieran salvado de las innovaciones de hoy por medio de unos arrebatos de bondad. Reacción espontánea de hermanos que a través de los años han sido respetados por su conocimiento, sinceridad, equilibrio de mente, amor, y genuinidad sin duda ha dado pausa en muchas ocasiones a los que querían introducir programas por los cuales no hubo autoridad bíblica.

La lista es casi sin fin: falsa doctrina, borracheras, prejuicio, inmodestia, abuso de drogas, etc. Todos tales justifican arrebatos de bondad ocasionales.

Uno debería estar seguro, sin embargo, que su arrebato sea uno de \textbf{bondad}. Nuestros hijos pueden ver justo por el medio de nuestra hipocresía. Nuestros hermanos pueden, también. Además, la hipocresía es deshonesta y pecaminosa.

Verdadera bondad, entonces, comienza con el corazón. Allí crece hasta que rebose, y a veces su rebosar será en la forma de un arrebato, un arrebato que es genuino y espontaneo, produciendo grandes beneficios. Uno que realmente aborrece lo que es malvado va a demostrar su aborrecimiento, \textbf{naturalmente} y \textbf{efectivamente}. Esta es la «bondad» que es el «fruto del Espíritu».

\section{No Licencia Para Pecar}
La misericordia de Dios no se debe ver como una licencia para pecar. «Nosotros, que hemos muerto al pecado, ¿cómo viviremos aún en él?» (\ibibleverse{Romans}(6:2)). 

Accidentes ocurren en ocasión en la casa Hall, con más frecuencia en la mesa del comedor. Un vaso de agua se derriba – por supuesto que está lleno. Una expresión de pena viene sobre la cara del torpe individuo. Pero no resulta ningún alboroto. La persona se disculpa; el desorden se limpia; y todo vuelve a lo normal. Pero si uno de nuestros hijos preguntara, «Papá, ¿puedo derribar un vaso de agua?», la respuesta sería «No». La paciencia de Madre no constituye permiso para botar agua. Tampoco permitiremos descuido en la mesa. Palabras de advertencia se expresan a menudo, que nuestros hijos \textbf{no boten el agua}; pero si ellos, a pesar de su precaución, y a causa de la fragilidad humana, derriban el agua, paciencia y perdón se les extenderá. 

¿No es este realmente el mensaje del Espíritu Santo en \ibibleverse{IJohn}(2:1)? «Hijitos míos, os escribo estas cosas para que no pequéis. Y si alguno peca, Abogado tenemos para con el Padre, a Jesucristo el justo».

Dios no concede permiso para cometer ni un solo pecado. Si de alguna forma un hombre pudiera comunicarse directamente con Dios, y hacer su petición, «Señor, he sido fiel en mi asistencia por un largo tiempo; no he faltado en un servicio en veinte años; entonces, Señor, ¿No estaría bien si yo no parto el pan con Tu pueblo solo esta vez para que yo pueda hacer lo que yo quiera; no me darás permiso solo esta vez?» ¿Realmente pensamos que el Señor le concedería permiso? O supongamos que su petición fuera contar una sola mentira, o involucrarse en un solo trato deshonesto de negocios, o cubrir una sola vez el hecho de que él es cristiano. ¿Concedería el Señor permiso para pecar? «Os escribo estas cosas \textbf{para que no pequéis}».

Tampoco permitirá el Señor el descuido. Él demanda que diligentemente y objetivamente estudiemos Su palabra y voluntad. «\textbf{Os escribo} estas cosas para que no pequéis». ¿Podemos creer que la persona que sin tener cuidado no lee ni estudia lo que se le escribe con el expreso propósito de guardarlo del pecado, o la persona tan prejuiciada que se ciega a lo que está escrito puede esperar la misericordia de Dios a causa de su ignorancia? ¡No lo creemos nosotros!

¿Hacia quienes, entonces, se extiende la gracia de Dios? Se extiende a los que son Sus hijos; los que buscan diligentemente por la luz de Su palabra; los que andan continuamente en esa luz; los que determinan jamás «dar lugar al diablo»; los que reconocen que, a pesar de sus esfuerzos, sin embargo, ellos sí pecan, y entonces vuelven humildemente y continuamente a Dios por perdón, confesando sus pecados. Estos son los que tienen «Abogado para con el Padre, a Jesucristo el justo» (\ibibleverse{IJohn}(1:5-2:1)). Encontramos gozo en compartir esperanza y consuelo a otros, pero mas allá de esta enseñanza, no nos atrevemos a ir.

\section{Olvidando y Extendiéndonos}
Un hombre nunca llega a un punto en su vida cuando él pueda relajarse en su lucha contra el malvado. Nadie estuvo mas consciente de ese hecho que el apóstol Pablo. Mientras estuvo en su tercera gira de predicación, el maduro y experimentado apóstol escribió a los santos en Corinto: «Golpeo mi cuerpo y lo hago mi esclavo, no sea que habiendo predicado a otros, yo mismo sea descalificado» (\ibibleverse{ICorinthians}(9:27)). Después, mientras estaba encarcelado en Roma, escribiendo a sus queridos en Filipos, él reconoció que él aun no lo había alcanzado, y tampoco era perfecto; que una lucha considerable aun le esperaba (\ibibleverse{Philippians}(3:12)). Crucial en esa lucha era la habilidad de Pablo de olvidar unas cosas.
\begin{enumerate}
\item \textbf{Pablo tuvo que olvidar lo que pudo haber sido.} Este escritor escuchó una vez a un entretenedor joven hacer la declaración, «Voy a renunciarlo todo; una persona simplemente no puede ser un cristiano y tener éxito en el campo de entretenimiento». Todos los que escucharon la declaración aplaudieron la fe del joven, y se maravillaron de su fuerza mientras él volvió a casa para cumplir su resolución. Pero el joven nunca pudo olvidar lo que pudo haber sido. Sueños de la fama y la fortuna siguieron metiéndose en su mente. Por fin la atracción de «lo que pudo haber sido» lo venció, y él regresó a esa vida que zandarearía, tentaría, y eventualmente destruiría su alma. 

Como joven, Pablo había mostrado gran potencial para alcanzar la fama mundana. Pocos hombres habían demostrado más potencial. Entrenado a los pies de Gamaliel, él había «aventajado en el judaísmo a muchos de [sus] compatriotas contemporáneos» (\ibibleverse{Galatians}(1:14)). Él había renunciado su derecho sobre la fortuna y la fama, sin embargo, para ser un cristiano. «Pero todo lo que para mí era ganancia, lo he estimado como pérdida por amor de Cristo» (\ibibleverse{Philippians}(3:7)). Qué fácil le hubiera sido para Pablo, encarcelado en Roma, con muchos de sus compañeros abandonándolo, comenzar a entristecerse sobre «lo que pudo haber sido». Si lo hubiera hecho, su efectividad como un siervo del Señor hubiera acabado y él probablemente hubiera perdido su alma. 
\item \textbf{Pablo tuvo que olvidar los pecados del pasado.} Es la verdad que Pablo nunca olvidó que él era un pecador salvado por gracia. Él frecuentemente habló de sus días como perseguidor y una vez se refirió a sí mismo como el primero de los pecadores (\ibibleverse{ITimothy}(1:15)). Pero sus pecados se le habían perdonado, lavados por la sangre de Cristo (\ibibleverse{Acts}(22:16)). Desde el día en que él se bautizó él dejó de rumiar sobre sus pecados y se convirtió en un cristiano gozoso, confiando en la promesa de la palabra de Dios de que sus pecados fueron todos perdonados. Satanás lleva una ventaja sobre un hombre que no puede olvidar los pecados del pasado (\ibibleverse{IICorinthians}(2:6-11)). La habilidad de Pablo de olvidar, entonces, era vital para su fidelidad continua al Señor.
\item \textbf{Pablo tuvo que olvidar los fracasos del pasado.} Fracasos son desalentadores, y la persona que se entristece sobre sus fracasos logrará muy poco en la vida. Pablo conoció el fracaso. Hubo pocos resultados después de sus esfuerzos en Atenas\ibible{Acts}(17:32-34). Los gálatas habían seguido un evangelio pervertido\ibible{Galatians}(1:6-7). Muchos que él convirtió se habían descarriado. Judaizantes habían socavado su enseñanza y reputación en muchas ciudades donde él había trabajado. Hombres inferiores se hubieran vuelto desesperados, lamentando su maltrato y convirtiéndose en víctimas de la autocompasión, una de las herramientas mas efectivas del diablo. Pablo tuvo que olvidar.
\item \textbf{Pablo tuvo que olvidar los éxitos del pasado.} Hubo muchos. Iglesias fuertes en Éfeso, Filipo, Tesalónica, y muchas otras ciudades se quedaron como pruebas concretas de la efectividad del trabajo de Pablo. Grandes hombres, como Tito, Timoteo y Epafrodito se habían convertido por él. Él fácilmente pudo haber dicho, «Mira lo que yo he hecho; ahora es tiempo de entregar la obra a hombres más jóvenes». Si lo hubiera hecho, él hubiera cesado de correr antes de haber «terminado la carrera», y ciertamente se hubiera perdido. 
\end{enumerate}
Pablo tuvo que olvidar, y Pablo pudo olvidar. Escribiendo a los filipenses, él dijo, «Hermanos, yo mismo no considero haberlo ya alcanzado; pero una cosa hago: olvidando lo que queda atrás y extendiéndome a lo que está delante, prosigo hacia la meta para obtener el premio del supremo llamamiento de Dios en Cristo Jesús» (\ibibleverse{Philippians}(3:13-14)). \textbf{¡Olvidando y extendiéndose!}  Aquí está la clave de servicio fiel al Señor. Pablo sirve como un ejemplo maravilloso. Él pudo olvidar la cosa que tuvo que olvidar, y extenderse a mayor servicio en la viña del Señor, a continua pureza de vida, a logros cada vez mas grandes para el Señor, al premio del supremo llamamiento de Dios en Cristo Jesús.

Un galardón maravilloso espera los que pueden olvidar y extenderse. Pablo pudo decir mientras se acercaba a la muerte, «En el futuro me está reservada la corona de justicia que el Señor, el Juez justo, me entregará en aquel día» (\ibibleverse{IITimothy}(4:8)). Y a todos nosotros el Señor dice, «Sé fiel hasta la muerte, y yo te daré la corona de la vida» (\ibibleverse{Revelation}(2:10)). Debemos perseverar. No podemos volver atrás. 

\section{A Mis Amigos Jóvenes Adultos}
Ya terminaste de estudiar y tienes tu propio trabajo y apartamento. ¡Felicidades! Yo recuerdo bien la alegría de esa recién-encontrada independencia y la satisfacción que viene de «hacer su propio camino». Pero tu nueva posición en la vida lleva consiga nuevas responsabilidades. 

¿Estás contribuyendo como debes? Es un gran brinco desde los \$5 que dabas del dinero que Mamá y Papá te enviaban a los \$50, \$60 o \$70 que deberías estar dando ahora. Y yo sé que puedes sentir que no te ajuste dar tanto; que esos pagos de tu auto, tu renta, tu ropa, tus muebles, etc., básicamente consumen todos tus ingresos. Pero el Señor debe estar en el primer lugar. Esto es el significado de \ibibleverse{Matthew}(6:33): «Pero buscad primero su reino y su justicia, y todas estas cosas os serán añadidas». Considera también el llamamiento (y promesa) de Dios a los judíos en \ibibleverse{Malachi}(3:8-10). Puede ser que tengas que contentarse con un apartamento o carro menos atractivo, o vivir sin ese estéreo o ese traje que tanto quieres. Pero no falles de dar como debes. Sobre todo, está seguro de no gastar mas en diversión y recreo que das en el servicio del Señor.

¿Estás aceptando sus responsabilidades hacía los enfermos y afligidos? Sí, yo sé que eso parecía ser la responsabilidad de Mamá y Papá y otras personas mayores hasta ahora; y yo sé, también, que estás ocupado con tu trabajo y que disfrutas hacer cosas «divertidas» con otros de tu misma edad, pero Dios da a todos estas responsabilidades (\ibibleverse{Matthew}(25:31-46)). Seguramente puedes tomar unas pocas horas cada semana para hacer una o dos visitas y enviar tarjetas a los enfermos. Y no se te olvide de los mayores. Ellos se alientan de una forma especial cuando jóvenes realmente disfrutan de estar con ellos. Tal vez haya espacio en tu corazón por unos pocos «abuelos» de más. Es posible, también, que su amistad pueda llenar un hueco en tu propia vida. Escucho a muchos jóvenes hablar de «encontrarse a sí mismos». Yo espero que estés mas preocupado con «perderse a sí mismo» a favor de Dios y otros.

Guarda tus morales diligentemente. Mamá y Papá ya no están allí para decir «No». Ahora debes evaluar la enseñanza que recibiste y determinar si la vas a seguir o si vas a asumir una posición de transigencia con el mundo. ¡Atrévete a ser un Daniel! ¡Resiste las presiones! Consérvate tan limpio y hermoso en lo interior como quisieras estar en lo exterior. El diablo no busca necesariamente una rebelión total en este punto, porque él sabe que transigencia eventualmente te llevará a su alcance.

Yo quiero que sepas que estoy de tu lado. Muchos de ustedes son especiales para mí, y me son muy alentadores. Su presencia en series de predicaciones me emociona. Sé sincero contigo mismo – ahora y siempre, y, sobre todo, nunca jamás decepciona a Dios. La felicidad en la vida o la miseria en la vida depende tanto de las decisiones tomadas en la juventud. Tu propia madurez personal será el factor determinante. 

\section{La Filosofía de la Puerta Marcada}
Todos hemos escuchado de universidades «cristianas», campamentos «cristianos», librerías «cristianas», etc. Pero has oído de puertas «cristianas»? Una puerta cristiana es una que se ha tallado para sugerir una cruz en la mitad superior y una biblia abierta en la mitad inferior. Estas puertas se fabrican en nuestro país y se compran por personas religiosas como símbolos de piedad.

La puerta original no se originó, sin embargo, como un símbolo de piedad principalmente, sino como un mensaje hospitalario al viajero cansado buscando hospedaje seguro. Tampoco era un articulo fabricado en serie para el dueño del siglo 18 o 19, sino era una puerta arduamente labrada y tallada, puesta al frente de la casa, llevando un mensaje silencioso pero poderoso al transeúnte, «Cristianos viven aquí; puedes pasar la noche aquí en paz y seguridad. ¡Bienvenido!».

¡Estamos impresionados! Aquí había gente que no de mala gana extendieron hospitalidad, sino hicieron un gran esfuerzo para marcar sus puertas como una invitación al viajero cansado a parar en su casa para su descanso necesario. Nos recuerda de la pareja sunamita que construyó un aposento pequeño en su casa, y lo amueblaron con una cama, una mesa, una silla y un candilero, para que cuando Eliseo pasaba por su área, él tendría un lugar cómodo donde quedarse (\ibibleverse{IIKings}(4:8-10)), y estaría en su casa. «Hospitalidad sin murmuraciones» lo llama la biblia (\ibibleverse{IPeter}(4:9)). 

No estamos alentando la compra de puertas cristianas como símbolos de piedad, pero sí estamos alentando la filosofía de la «puerta marcada» que ve la hospitalidad como un amado privilegio en vez de un deber despreciado. Apreciamos a los que pueden emocionarse porque alguien estará en su casa. La mayoría de los argumentos en la junta de varones son repulsivos, pero hay algo refrescante en una discusión sobre «¿quién va a hospedar al predicador?» cuando tales discusiones ocurren, no por ambición egoísta u orgullo, sino por un amor genuino en que la gente quiere tener el predicador en su casa.
 
El desarrollo de una filosofía «puerta marcada» puede transformar una iglesia muerta en una congregación activa y funcional. Cuando las personas comienzan a agregar unas pocas papas extras al almuerzo del domingo para que puedan invitar visitantes a su casa, cuando las personas abren sus casas para grupos de estudio bíblico, cuando las personas dan la bienvenida a nuevos miembros con una fiesta para que todos se conozcan, cuando a las personas simplemente les gusta tener buena gente en su casa, cuando los hogares de una congregación generalmente se convierten en lugares de regocijo para los que regocijan o refugios de compasión para los que lloran, los efectos serán dramáticos y los frutos eternos.

La filosofía «puerta marcada» solo funciona dentro de los desinteresados. Nosotros «tallamos nuestras puertas» por apagar la televisión, dejar el periódico, hacer un poco extra esfuerzo, e incluso aguantar de vez en cuando un viejo cascarrabias que aprecia nada y especializa en hacerles a todos miserables. Pero a largo plazo los que la desarrollan son ricamente bendecidos, igual ahora como en la eternidad. 

\section{La Llamada Telefónica Que Nunca Se Hizo}
Es la tarde del domingo, y tu y tu familia se invitan a la casa Smith para sándwiches y postre después del servicio de la noche. La invitación se extendió mas temprano en la semana y la aceptaste. Pero ahora surgió otra cosa que preferirías hacer, entonces vas al teléfono para cancelar tu participación. Pero antes de marcar el número, piénsalo un momento\ldots

Al mismo tiempo que estás pensando en cancelar, otros están teniendo pensamientos parecidos. Dentro de una hora, Sra. Smith recibirá llamadas de cinco de las siete parejas que ella ha invitado, diciendo que, ellos tampoco vendrán. Está muy tarde para invitar a otros. Está muy tarde para cancelar los planes por completo. Comida se desperdiciará; tiempo se habrá desperdiciado; la ocasión no estará tan alegre y relajada a causa del pequeño número; y Sra. Smith se quedará lastimada y profundamente decepcionada. \textbf{Si tú llamas, tu cancelación será injusta a Sra. Smith.}

Si estuvieras cancelando por una emergencia, tu cancelación seria legítima. Pero no estás cancelando por esa razón. Estás cancelando porque hay algo que tú preferirías hacer. Solo estás pensando en el yo en vez de otros. Estabas preparado a disfrutar la hospitalidad de los Smith si nada que se escuchara mas emocionante y agradable surgiera. \textbf{Si llamas, tu cancelación demostrará tu egoísmo y falta de respeto hacia los Smith.}

Sra. Smith estará tan desanimada por las cancelaciones que le será difícil invitar a otro grupo. Ella sabe que la biblia dice, «Sed hospitalarios los unos para con los otros, sin murmuraciones» (\ibibleverse{IPeter}(4:9)). Y que el pueblo de Dios debe ser «hospitalario» (\ibibleverse{Titus}(1:8)). De hecho, por eso ella te había incluido en el grupo de sus invitados. Ella quería conocerte más, construir una amistad contigo, y ayudarte a formar vínculos estrechas con otros en la iglesia. Pero nada de eso se podrá realizar. Los únicos que llegarán al final serán sus mejores amigos. El propósito principal de extender hospitalidad será frustrado. Será difícil intentarlo de nuevo. \textbf{Si llamas, tu cancelación será un desaliento para Sra. Smith en sus esfuerzos de hacer la voluntad de Dios.}

Sra. Smith sacudirá, con tiempo, su desanimo e invitará a otro grupo, pero, para bien o para mal, probablemente no estarás incluido. Ella estará invitando a los que ella esté segura de que apreciarán la invitación y que no la decepcionarán. Te encontrarás criticando a Sra. Smith por siempre invitar «los mismos de siempre», pero esos «mismos de siempre» son las personas que aceptan invitaciones, cumplen sus citas, y verdaderamente disfrutan su tiempo en los hogares de otros. Pronto observarás que esos «mismos de siempre» son los que se invitan por todos, porque a ellos obviamente les gusta estar con personas. Mientras tanto, te encontrarás más y más aislado de cristianos. \textbf{Si llamas, el resultado de tu cancelación, con tiempo, será que pases menos tiempo con el pueblo de Dios y te sentirás más negativo hacia ellos.}

Con cancelar, no estás siendo bondadoso; estás actuando indecorosamente hacía Sra. Smith; estás buscando tus propios intereses en vez de los intereses de otros. \textbf{Si llamas, tu cancelación demostrará una falta de amor por otros} (\ibibleverse{ICorinthians}(13:4-8)). 

Entonces, antes de usar ese teléfono, piensa en los demás, y puede ser que decidas ir al hogar de los Smith después de todo. De hecho, puede ser que experimentes un cambio total de actitud, desarrollando un sentimiento positivo hacia la noche. Habiendo cambiado tu actitud, disfrutarás el tiempo con otros cristianos. Verás que Sra. Smith está muy contenta de que viniste. Puede ser que incluso quieras tener el mismo grupo en tu casa alguna vez, y estarás esperando – ¡oh cuánto estarás esperando! – que ninguno de ellos llame para cancelar. Te convertirás en una mejor persona, más cariñoso, más hospitalario, y más involucrado con otros cristianos, todo a causa de la llamada telefónica que nunca se hizo.

\section{Cinco Minutos y Diez Centavos}
Hace poco, pasando por una aldea en mi camino a una reunión, me acordé de una dama cuya madre recientemente había experimentado una enfermedad muy grave. Preocupado, me desvié de la autopista, encontré una cabina telefónica que estuvo cerca, y llamé a la dama. Solo llevó cinco minutos y diez centavos, pero mientras yo conducía de regreso a esa autopista, yo sentía una verdadera calidez en mi corazón, una calidez que no se puede comprar con dinero – sabiendo que yo había expresado en una pequeña manera mi preocupación por alguien más.

Pero mientras yo seguía mi camino en esa autopista, no pude evitar preguntarme cuantas veces yo había enfrentado una situación parecida en que cinco minutos y diez centavos hubieran logrado mucho bien, solo para seguir andando en mi propia manera egoísta, no pausando para ayudar a mi prójimo. Tendemos a poner muy poca importancia en los gestos pequeños y cotidianos que pueden ser tan significativos. Jesús dijo una vez, «Y cualquiera que como discípulo dé de beber aunque solo sea un vaso de agua fría a uno de estos pequeños, en verdad os digo que no perderá su recompensa» (\ibibleverse{Matthew}(10:42)). 

María, la hermana de Marta y Lázaro, una vez ungió a Jesús con un perfume – un hecho pequeño, realmente – pero dondequiera que el evangelio se predica, ese incidente se cuenta en memoria de ella (\ibibleverse{Matthew}(26:13)).
Pero déjame dar crédito a la buena dama mencionada arriba. Ella pareció apreciar tanto mi gesto pequeño. Muy a menudo la gente es ingrata y dan por hecho las cosas pequeñas que se les hacen. Ellos quitan el gozo de hacer los hechos pequeños y amables por su actitud ingrata. Pero no fue así con esta dama. Su demostración de aprecio contribuyó bastante a esa calidez en mi corazón. Yo paré para ayudarla, pero ella en cambio me ayudó a mí; de hecho, me alegró el día.

Que recordemos pensar en los demás. Esa tarjeta de «Mejórate», esa llamada telefónica, esa visita corta, esa palabra de elogio, esa expresión de preocupación puede proveer alegría duradera a alguien en necesidad. Y no llevará mucho tiempo ni costará mucho dinero. También, que recordemos aceptar los pequeños hechos de bondad con gracia, con aprecio genuino. Porque estas son las pequeñas cosas que traen la luz del sol a nuestras vidas y nos ayudan a prepararnos para la eternidad.

\section{Observad a Estas Personas}
Rápido ahora – ¿qué clase de personas se debe observar por cristianos según las escrituras? ¿Los que causan disensiones y tropiezos contra la doctrina de Cristo? Sí, porque así nos enseña \ibibleverse{Romans}(16:17).
Hay otra clase de personas que se debe observar, sin embargo. «Hermanos, sed imitadores míos, y observad a los que andan según el ejemplo que tenéis en nosotros» (\ibibleverse{Philippians}(3:17)). La palabra «observar» no es sinónima de la palabra «apartarse». Según W.E. Vine la palabra significa «mirar, contemplar, observar». Los que son malvados, entonces, deben \textbf{observarse} y \textbf{evitarse}; mientras tanto, los que andan en el camino de Dios deben \textbf{observarse} y \textbf{seguirse}. 

\textbf{Ancianos piadosos se deben observar.} Ancianos deben ser ejemplos para el rebaño (\ibibleverse{IPeter}(5:3)). Deben ser hombres de carácter intachable, que gobiernan bien su propia casa, que son hospitalarios, y cuya sana enseñanza puede refutar a los que contradicen. Conocemos tales ancianos, y su ejemplo es inestimable. Incluso después de que tales hombres hayan pasado de esta vida, se deben recordar: «Acordaos de vuestros guías que os hablaron la palabra de Dios, y considerando [una palabra parecida a observar] el resultado de su conducta, imitad su fe» (\ibibleverse{Hebrews}(13:7)). ¡Gracias a Dios por ancianos piadosos; obsérvalos y sigue su buen ejemplo!

\textbf{Mujeres piadosas se deben observar.} Si hay mujeres en la congregación que sobresalen por su piedad y humildad, cuya mayor atracción es sus «espíritus tiernos y serenos», que se han adornado de buenas obras, que aman a sus maridos e hijos, que encuentran gozo y contentamiento en ser una esposa y madre y gobernadora de casa, que no sienten ningún rencor hacía su posición de sumisión al hombre, que han dedicado sus vidas a hacer la voluntad de Dios – y hay tales mujeres en cada congregación – entonces observa a estas mujeres piadosas y sigue su ejemplo. En estos días cuando el movimiento de la liberación de la mujer está afectando a tantas e incluso intimidando a muchas mujeres que quieren hacer lo correcto, es maravilloso tener mujeres piadosas en la iglesia que pueden ser guías y proveen un modelo para otras mujeres cristianas. ¡Gracias a Dios por mujeres piadosas; obsérvalas y sigue su buen ejemplo!

\textbf{Predicadores piadosos se deben observar.} No todos los predicadores son piadosos, pero la mayoría de los predicadores del evangelio que nosotros conocemos son hombres piadosos cuyas vidas hablan tan efectivamente como sus labios. Ellos hacen su trabajo, no como empleados, sino como hombres preocupados por la verdad y por las almas de hombres y mujeres. ¡Observa tales hombres y sigue su ejemplo!

Padres piadosos, jóvenes piadosos, mayores piadosos, personas piadosas que sufren, personas piadosas que están muriendo –¡los fieles piadosos! Que triste que unos son tan cegados por las fallas de los pocos que no pueden ver las virtudes de los muchos.

Los piadosos nos rodean. Que los busquemos, que contemplemos sus buenas cualidades, que los observemos, y que sigamos su ejemplo.

\section{¿Honrar A Nuestros Enemigos?}
Una de las inconsistencias más raras que existe entre el pueblo de Dios es su tendencia de honrar a sus mayores enemigos. Considera los cristianos en el día de Jacobo. Ellos se sentían tan halagados cuando hombres ricos entraban en sus asambleas que pomposamente les decía, «Tu siéntate aquí, en un buen lugar», olvidando que ellos representaron los mismos que los habían oprimido, los habían arrastrado a los tribunales, y habían blasfemado el buen nombre por lo cual se llamaron (\ibibleverse{James}(2:1-7)). Ellos admiraban los enemigos de verdad y justicia mientras despreciaban los pobres que eran ricos en fe. 

Podemos ser igual de culpables en nuestra generación. ¿No se sentiría la mayoría de nosotros muy honrados si tuviéramos algún renombrado entretenedor como invitado en nuestra casa? No importa la corrupción de sus películas ni la suciedad de sus canciones. Este es un verdadero célebre y él ha venido a \textbf{nuestra} casa. ¡Qué halagados nos sentiríamos! Que orgullo sentiríamos mientras contáramos a nuestros amigos del honor que fuera nuestro.

Nosotros amontonamos gran honor sobre famosas figuras de deporte con poca consideración por donde están moralmente o espiritualmente. Nos esforzamos para conocer a políticos y encontramos gran placer en mencionar los nombres de los famosos que hemos conocido. Parece que ni estamos conscientes de la gloria que les conferimos a las mismas personas que están socavando la fibra moral de nuestra nación y ejercen una gran influencia en nuestras propias vidas y en las vidas de nuestros hijos.

¿Por qué admiraron los cristianos a los ricos en el día de Jacobo? La respuesta seguramente debe encontrarse en nuestras propias prioridades equivocadas. Con nuestras palabras hablamos de la necedad de ganar el mundo entero al costo del alma de uno, mientras al fondo glorificamos la fama, la fortuna y el placer. Intelectualmente, entendemos que el verdadero éxito en la vida se debe medir en términos de la preparación de uno para la eternidad, pero es difícil quitar de nuestros corazones los conceptos mundanos del éxito y de la grandeza. No queremos implicar que todos los ricos y famosos son corruptos (tal no sería verdadero), pero estamos diciendo que prioridades equivocadas nos pueden cegar a la influencia corruptora que la maldad y suciedad en las alturas puede tener sobre nuestras propias almas.

Los grandes en el día de Jacobo no eran los participantes hábiles en los juegos romanos, ni los gigantes militares, ni los cesares. Cualquier honor que se les debía a tales hombres era el honor relevante al oficio que ocupaban (\ibibleverse{Romans}(13:7)). Los grandes de aquel día eran tales personas como Bernabé, Dorcas, María, Timoteo, Pedro y Pablo. Los verdaderos grandes de nuestro día son los que son fieles en servicio a Dios y al hombre. Nosotros adoramos con unos de ellos cada día del Señor. Nuestros hogares se pueden honrar con su presencia en casi cualquier día. Ellos son hombres y mujeres «de los cuales el mundo no [es] digno». Que reconozcamos su grandeza, aprendamos a verdaderamente amarlos y apreciarlos, y darles el honor que se les debe.

\section{Mirando Adentro Desde Afuera}
Nuestra niña de siete años casi no pudo esperar llegar a Canadá donde ella podría andar en bus a la escuela. A menudo ella había mirado esos buses grandes y amarillos, llenos de niños, y había pensado en todo lo que ella perdía de la vida porque no andaba en un bus así. Toda la emoción estaba adentro, y ella estaba afuera mirando adentro. Ya nos hemos mudado a Canadá. Ella tiene tres meses de andar en ese bus a la escuela, y está aprendiendo una lección importante: lo que parece tan emocionante cuando estás afuera mirando adentro a menudo resulta ser solo una ilusión; la realidad nunca iguala totalmente al sueño. 

Así es en las esferas espirituales. Nos preguntamos cuantos cristianos, nacidos y criados «en la iglesia», sienten que se han privado de la verdadera diversión en la vida por su disciplina temprana; que seguramente los verdaderos «buenos tiempos» se pueden tener en el mundo con sus aulas de baile, vida nocturna, fiestas obscenas, e intrigas emocionantes. Sus convicciones siendo demasiado fuertes y las presiones demasiado grandes para permitir que realmente participen en tales actividades, ellos se quedan afuera mirando adentro con algún anhelo. Si solo pudieran darse cuenta de que la fruta de tal conducto es indescriptiblemente amarga, y, además, que la realidad nunca llega a igualar al sueño. Si solo escucharan a un «hijo prodigo» de nuestra generación, ellos aprenderían que las promesas del pecado no son solo una ilusión, sino una ilusión muy cruel. 

Nos preguntamos cuantas personas en iglesias fieles se impresionan con los grandes programas promocionales y estadísticas hinchadas de iglesias mas «progresivas», y como resultado sienten un poco de vergüenza sobre los esfuerzos bíblicos, pero a veces algo deslucidos, de la congregación de la cual son una parte. Uno solo necesita observar la naturaleza cíclica y temporánea de tales programas para darse cuenta de que no hay nada en ellos de valor permanente, pero pueden parecer tan apasionante cuando uno está afuera mirando adentro.

Lo que se necesita es fe: fe para aceptar que la palabra de Dios es completa, capacitándonos para toda buena obra (\ibibleverse{IITimothy}(3:16-17)); fe para aceptar que el plan de Dios es el mejor, y mientras puede ser que Su plan no parezca tan emocionante a los ojos humanos (el evangelio funciona como levadura, no como dinamita), ello no se puede mejorar por medio de sabiduría e ingenio humano; fe para aceptar que las promesas del evangelio no son tan ilusorias como las del mundo, sino que lo que Dios «había prometido, poderoso [es] también para cumplirlo» (\ibibleverse{Romans}(4:21)); fe para aceptar que la palabra de Dios provee la única verdadera formula para una vida feliz y realizada; fe para aceptar que la vida y felicidad eterna son mucho mas deseables que los placeres y la emoción de este mundo; fe para aceptar que Dios sabe lo mejor en todo.

¡No estés engañado! ¡No mires con envidia al mundo con sus riquezas y placer! Si tú eres cristiano – no solo en nombre, sino en vida y afección – tienes dentro de la mano la mayor felicidad y realización que se pueden obtener en esta vida. 

\section{Super Cristianos}
Ten cuidado con los que se jactan de superiores cualidades espirituales o habilidades intelectuales. Su conducta invariablemente trae lastima. 

Una pareja de novios jóvenes sale de viaje un fin de semana. Su plan es quedarse en el mismo cuarto de motel, pero dormirán en camas separadas. Ellos realmente no entienden el problema. Ellos están conscientes de lo terrible del pecado. Ellos saben de los riesgos de la fornicación. Ellos han dado sus vidas a Cristo. Ellos pueden manejar cualquier tentación que puede presentarse. 

Un hombre en la congregación se cansa de las lecciones simples del evangelio que él escucha domingo tras domingo. Él anhela algo profundo, algo que desafíe su intelecto. Cuando él tiene la oportunidad de enseñar una clase de biblia, él le da a la clase algo que masticar. Nada de ese viejo material repetitivo del libro de Hechos o de la vida de Cristo se tratará en su clase jamás. 

Una mujer casada constantemente está en la presencia de un hombre soltero. Dondequiera que esté ella, el hombre joven estará allí a su lado. Pero no te preocupes. Ellos son personas de gran espiritualidad y el uno sacará mucho provecho de las cualidades espirituales del otro.

No estamos tratando de lo hipotético; estas son personas reales y vivas que hemos descrito. Y la triste verdad es que tales se encuentran en congregaciones en todo el país, creando problemas y causando miseria entre los fieles al Señor. 

Su contrapartida aparece en las escrituras en la forma de los corintios (\ibibleverse{ICorinthians}(8:1-13)), que se gloriaron en su conocimiento superior; que eran tan «fuertes» que podían entrar en un templo idolatro y comer de las fiestas idolatras; que pensaban que no podían caerse; que no podían preocuparse con los espiritualmente débiles en la iglesia que se podían llevar al pecado por su conducta; que aparentemente se motivaban por la actitud «después de todo, tales personas deberían madurarse y dejar de ser tan débiles – la iglesia probablemente estaría mejor sin ellos de todos modos».

Es interesante observar que Pablo no estuvo muy impresionado con este aire de espiritualidad y conocimiento manifestado por estos corintios. A ellos él escribió, «El conocimiento envanece, pero el amor edifica» (\ibiblechvs{ICorinthians}(8:1)) y, además, no saben «como lo debe[n] saber» (\ibiblechvs{ICorinthians}(8:2)). Él los recordó que el hermano débil que se destruía por su «conocimiento» era uno «por quien Cristo murió» (\ibiblechvs{ICorinthians}(8:11)). En contraste a su arrogancia, él habló de sus propias acciones en golpear su cuerpo, y hacerlo su esclavo… no sea que él sea descalificado (\ibiblechvs{ICorinthians}(9:27)). Después en una advertencia final él dijo, «Por tanto, el que cree que está firme, tenga cuidado, no sea que caiga» (\ibiblechvs{ICorinthians}(10:12)).

Si te encuentras entre ellos descritos arriba, te urgimos a arrepentirse. Si no estás entre ellos, ten cuidado con los que sí, y «restaurad[los] en un espíritu de mansedumbre»\ibible{Galatians}(6:1). Verdaderamente, «Delante de la destrucción va el orgullo, y delante de la caída, la altivez de espíritu» (\ibibleverse{Proverbs}(16:18)).

\section{Viviendo Con Las Críticas}
De la revista \textit{Reader’s Digest} tomamos esta cita, escrita por Roger Rosenblatt: «He visto muchas cosas maravillosas en mi vida, pero nunca he visto alguien que puede aceptar bien a la crítica. Toda crítica, sea desenfadada o feroz o constructiva, es desagradable. Sí, puedes beneficiar de las críticas a la larga y a lo doloroso. Pero aceptarlas es otra cosa. Aceptarlas significa dejar que se vaya para abajo como pudin – sin parpadear, sin estremecerse, sin desear morir».

¡Cuán cierto! Pero estoy contento de que mis críticos no siempre han guardado silencio. «Eres un buen predicador, pero te repites demasiado», un hombre me dijo cuando comencé a predicar. No se fue «para abajo como pudin», y no estoy seguro de que mi «gracias» fuera tan sincera como debió ser, pero yo intenté más después de eso de no repetirme innecesariamente. «Cantas tan fuerte que me das un dolor de cabeza», una dama se quejó en otra ocasión. ¡Ay! Esa critica dolió mas que su dolor de cabeza. Pero, yo sí necesitaba moderar el volumen. 

Hay críticas deshonestas. La crítica de Judas tocante a María sonó noble: «¿Por qué no se vendió este perfume… y se dio a los pobres?» (\ibibleverse{John}(12:5)). Pero fue deshonesta, y el sonido noble solo fue un disfraz intencional para esconder sus propósitos deshonestos. Además, la crítica no hubiera sido válida incluso si hubiera sido honesta. Con la ayuda de Jesús María pudo ignorar esta crítica. De igual manera a veces nosotros deberíamos ignorar las críticas. Yo quisiera siempre poder discernir cuando. 

La manera en que las críticas se ofrecen hace una gran diferencia. Si yo debo ser criticado, haz que la crítica sea breve y ve al grano, y después déjala. Las críticas que se repiten una y otra vez se convierten en el molestar, y molestar usualmente produce la terquedad en vez de mejoramiento. No actúes como si estuvieras bromeando cuando estás criticando. Es cierto, también, que ser dulce ayuda a la persona a aceptar lo difícil que tienes que decirle. No me halagues antes de «matarme», pero una expresión sincera de aprecio ofrecida con la crítica hace que sea más fácil de tragarla. Si no sientes ningún aprecio por mí, probablemente deberías dejar las críticas a alguien que sí. 

He conocido a personas que podían beneficiar de las críticas y continuar de amar a sus críticos, y los he admirado. Pedro era un hombre así. De hecho, su habilidad de aceptar las críticas era un factor principal en la grandeza que él logró.

El Espíritu Santo tiene algo que decir sobre este tema: «No reprendas al escarnecedor, para que no te aborrezca; reprende al sabio, y te amará» (\ibibleverse{Proverbs}(9:8)). ¡Exactamente! Puede ser que el verdadero carácter de un hombre se pueda determinar por su habilidad de aceptar las críticas. 

Entonces, si tienes que ser un crítico, aprende a suavizar el golpe mientras a la vez seas franco. Siempre seas honesto. Y, si te deben criticar, ten la humildad como para hacer las correcciones necesarias. Mientras dure la tierra habrá las críticas. Que aprendamos a vivir gentilmente con ellas.

\chapter{ÁNIMO}

\section{Abundante en Misericordia}
Al estudiar la relación que existía entre Jesús y sus apóstoles, nos maravillamos de dos hechos: (1) la debilidad de los apóstoles, y (2) la paciencia de Jesús.

Los apóstoles eran hombres verdaderamente débiles y torpes. Jesús les contó una y otra vez de la naturaleza espiritual de Su reino, pero ellos siguieron buscando un reino terrenal y discutieron como niños cual de ellos sería el más grande. Él intentó prepararlos por su crucifixión, pero ellos nunca entendieron. Él realizó grandes milagros ante ellos, pero aun así ellos podían asustarse, incluso cuando Él estuvo entre ellos. Él a menudo tuvo que llamarlos «hombres de poca fe».

Hubo Pedro: hablando cuando debía guardar silencio, hundiéndose en el mar, dormido cuando debía orar, cortando la oreja de Malco, siguiendo de lejos, y negando a Jesús. 

Sin embargo, a pesar de todas las metidas de pata de Pedro, uno nunca puede dudar del amor de Pedro por el Señor, su fe en Él como el Hijo de Dios, o su deseo de complacerle. Contestando a la pregunta de Jesús, «¿Acaso queréis vosotros iros también?», Pedro soltó, «Señor, ¿a quién iremos? Tú tienes palabras de vida eterna» (\ibibleverse{John}(6:67-68)). ¡Verdaderamente palabras bien dichas! Pedro en sus momentos impetuosos podía decir las cosas equivocadas, y en su debilidad podía pecar, pero él amó al Señor y no estuvo por dejarlo. Él siempre estaba dispuesto a aceptar las reprimendas del Señor, y pudo llorar amargamente cuando él sí pecó. Y así era con los demás apóstoles.

Hombres inferiores se hubieran desesperado, dejando a estos discípulos débiles, frágiles y torpes. Pero no nuestro Señor. Él no estaba tan preocupado por lo que eran en aquel entonces como por lo que podían llegar a ser. Y por medio de Su paciencia Él los transformó en los hombres mas fuertes conocidos por este mundo. Él los reprendía y los disciplinó, pero ellos nunca cuestionaron su amor ni dudaron que le pertenecían. Ellos se le habían dado a Él por Dios, y Él había guardado a cada uno de ellos – excepto Judas (\ibibleverse{John}(17:12)). 

¡Excepto Judas! ¡El «hijo de perdición»! La paciencia del Señor tuvo que cortarse con respecto a él, porque él le había dado la espalda, apostató, y rehusó la reprimenda y disciplina del Señor. ¡Y el Señor lo dejó!

¿No podemos ver la actitud de Dios hacia nosotros mientras observamos a Jesús y Sus apóstoles? Jesús en la tierra era «Dios con nosotros» (\ibibleverse{Matthew}(1:23)), «la imagen del Dios invisible» (\ibibleverse{Colossians}(1:15)), «el resplandor de su gloria y la expresión exacta de Su naturaleza» (\ibibleverse{Hebrews}(1:3)). La actitud que Jesús demostró hacia los apóstoles en sus debilidades es la misma actitud que Dios tiene hacia nosotros en nuestras debilidades. Nunca debemos cuestionar. Su amor y paciencia hacia los que viven según Su enseñanza, que aceptan Su disciplina y reprimendas, y qué arrepintiéndose, buscan constantemente Su perdón. Y nunca debemos cuestionar su habilidad de transformarnos (¡Sí, incluso yo!) en los siervos fuertes y útiles que Él quiere que seamos. «Compasivo y clemente es el SEÑOR, lento para la ira y grande en misericordia» (\ibibleverse{Psalms}(103:8)).

\section{¿Piensas En Dejarlo?}
Sin duda, alguien que leerá este artículo está desanimado, cansado de luchar, decepcionado por sus hermanos, enfrentando severos obstáculos en su servicio al Señor, y está en peligro de dejarlo. Pero yo haría este llamamiento: «Antes de dejarlo, mira de nuevo en las incentivas que el Señor te tiene guardadas; mira de nuevo en los cielos».

\textbf{Los cielos significan victoria} – la victoria en la lucha con Satanás y sus alianzas. La batalla es dura a veces; el enemigo formidable. Nuestra propia fuerza parece tan pequeña. Nos desanimamos. ¡Pero levantémonos los ojos! De nuestro lado está el Señor, Él que ya ha ganado la batalla (\ibibleverse{Genesis}(3:15)). Con su ayuda podemos ser victoriosos. Podemos ser «más que vencedores por medio de aquel que nos amó» (\ibibleverse{Romans}(8:37)).

\textbf{Los cielos significan hermosura} ¬– hermosura sin igual – una calle de oro, un muro de jaspe, un cimiento de piedras preciosas, portones de perla, un rio cristalino saliendo del trono de Dios.

\textbf{Los cielos significan un hogar} – la felicidad no se encuentra en nuestro entorno material, sin en estar con los que amamos. Experimentamos la nostalgia, pero nuestra nostalgia no es por una casa ni por nada material sino por una esposa e hijos, por queridos. De igual manera, hablamos de la hermosura de los cielos, y verdaderamente su hermosura aumenta a nuestra anticipación, pero seguramente el mayor gozo se hallará en estar con nuestro Señor, Él «que no habiendo visto amamos»\ibible{IPeter}(1:8), y con nuestro Dios, con el Espíritu, con los ángeles, y con los redimidos de todos los siglos. Esto será nuestro regreso al hogar.

\textbf{Los cielos significan felicidad} – no habrá nada allá que quite de nuestra felicidad. Ya no habrá lágrimas, ni enfermedades, no mas penas, no mas muerte, «ni habrá más duelo, ni clamor, ni dolor, porque las primeras cosas han pasado» (\ibibleverse{Revelation}(21:4)). 

\textbf{Los cielos significan santidad} – jamás volveremos a escuchar que el nombre de nuestro Señor se tome en vano. Alla no habrá mas adulterio, suciedad, culpa, homicidios, crimen. La verdad abundará. Ya no habrá mentiras, engaño, halagos, hipocresía. Las mujeres ya no tendrán que temer a los que quisieran abusarlas y profanarlas. El amor será puro y sin engaño. El Señor nos asegura, «y jamás entrará en ella nada inmundo, ni el que practica abominación y mentira, sino solo aquellos cuyos nombres están escritos en el libro de la vida del Cordero» (\ibibleverse{Revelation}(21:27)).

\textbf{Los cielos significan eternidad} – este bendecido estado jamás terminará. «Cuando hayamos estado allá diez mil años, brillando como el sol, no tendremos menos días para cantar las alabanzas de Dios que cuando comencemos» (Joseph Scriven).

Puedes dejarlo si quieres, porque Dios no forzará a nadie a servirle. Pero, acuérdate, el día que lo dejas es el día que renuncias toda esperanza de felicidad eterna y escoges en su lugar la condenación eterna. Y la eternidad es un largo tiempo.

\section{Considerando El Resultado}
No tenemos simpatía por los que en cada funeral «predican los muertos hasta los cielos», que ignoran el obvio fracaso de parte del difunto de servir el Señor, que intentan pensar que todos de alguna manera llegarán a los cielos a pesar de su desobediencia. Hay un infierno, y los muchos de esta tierra (en contraste a los pocos) van allá (\ibibleverse{Matthew}(7:13-14)).

Pero no podemos dejar que un extremo engendre a otro. Mientras la mayoría se perderá, hay los en este mundo cuyo único propósito en la vida es servir el Señor y van a los cielos cuando mueren; que se dedican diariamente a estudiar la palabra de Dios y a vivir según sus preceptos. Cuando tales personas mueren es correcto hablar de su buena vida y fidelidad al Señor; es correcto extender esperanza a sus queridos; es correcto recomendar su ejemplo de fidelidad y perseverancia a los que pueden desanimarse; es correcto hablar de su galardón eternal.

El escritor de Hebreos lo dijo así: «Acordaos de vuestros guías que os hablaron la palabra de Dios, y considerando el resultado de su conducta, imitad su fe» (\ibibleverse{Hebrews}(13:7)). Tres cosas se implican en este versículo: (1) Hubo hombres entre ellos que fielmente sirvieron al Señor hasta la muerte. (2) Los cristianos hebreos debieron confiar que el resultado de las vidas de esos hombres era feliz. (3) Considerando esto, debieron imitar su fe.

Nosotros, como los hebreos, hemos tenido la buena fortuna de conocer a muchos que han vivido vidas piadosas y que ahora han pasado a su galardón. Unos eran ancianos; unos eran predicadores; unos eran maestros de clases bíblicas; otros simplemente eran buenos, fieles y confiables discípulos del Señor que pusieron Su reino en el primer lugar en sus vidas. Nosotros no estamos hablando de los tibios e indiferentes. No estamos hablando de los que se dicen ser cristianos, pero carecen de los frutos de la cristiandad. Estamos hablando de los verdaderamente dedicados que constantemente reflejaron el carácter de su Padre y de su Señor Jesucristo. Su influencia en nuestras vidas era tremenda. Ellos eran hombres y mujeres de quienes este mundo no era digno. Y la lista crece con cada año que pasa. 

Cada uno de estos tuvo que vencer obstáculos serios para ser fiel al Señor. El diablo desafió a cada uno de ellos, pero él falló. Ellos ganaron la victoria por medio de Cristo. Ahora están felices eternamente porque ellos perseveraron. Si pudiéramos preguntar a cualquier de ellos, «¿Valió la pena?», su respuesta inmediata sería, «¡Sí, mil veces sí que valió la pena!» Ni uno de ellos se arrepiente de un solo momento que él pasó en el servicio del Señor.

Damos gracias a Dios por tales personas. Nos regocijamos en su salvación. Miramos en su fallecimiento no como los «que no tienen esperanza». Y «considerando el resultado de su conducta», nos dedicamos a imitar su fe.

\section{La Esperanza Se Encuentra en Cristo}
«Yo quisiera pensar que yo haya venido solo una corta distancia hacia estar tan preparada como él estuvo», una mujer me dijo hace poco, refiriéndose a un hombre que había muerto. Su declaración se pretendió ser una expresión de confianza en el difunto, y yo compartía su confianza, pero la verdad es: si esa dama esta viviendo una vida fiel en Cristo, ella está tan preparada como cualquier otra persona en Cristo.

La entrada a los cielos no se basará en una lista larga de méritos que se acumulan a través de los años (tantos nuevos conversos, tantos pasajes memorizados, tantas vidas influenciadas, tantos años en el servicio del Señor, tantos sermones predicados, etc.), la probabilidad de ir a los cielos siendo mejorado con cada nuevo mérito. La entrada a los cielos se basará en la sangre de Cristo. Uno se prepara para los cielos por entrar en Cristo por medio de fe, arrepentimiento y bautismo (\ibibleverse{Galatians}(3:27); \ibibleverse{Romans}(6:3)), viviendo una vida fiel en Cristo, y muriendo en Cristo. 

«Bienaventurados los muertos que de aquí en adelante mueren en el Señor» (\ibibleverse{Revelation}(14:13)). Esta es la verdadera base de la esperanza de uno, si ha sido un cristiano fiel por cincuenta años, o acaba de levantarse del bautismo en novedad de vida. 

Ahora si la dama hubiera dicho, «Yo quisiera pensar que yo haya venido solo una corta distancia hacia estar tan como Dios como él estuvo», eso hubiera sido diferente. Aquí está el propósito principal de todo cristiano, el ser mas como Dios cada día. «Y todo el que tiene esta esperanza puesta en Él, se purifica, así como Él es puro» (\ibibleverse{IJohn}(3:3)). Adoración regular, contribuciones liberales, oraciones incesantes, benevolencia pensativa, amor no fingido de los hermanos, autodominio, etc. – todos de los cuales se mandan por Dios – son un medio al fin de ser como Él, o como dicho por Jesús, «para que seáis hijos de vuestro Padre que está en los cielos» (\ibibleverse{Matthew}(5:45)). Uno que ignora estos mandamientos muestra su desprecio por Dios, se vuelve infiel, termina de ser como Dios, y pierde su esperanza de los cielos. Al otro lado, uno que sí obedece conscientemente los mandamientos de Dios vuelve a ser mas como Dios, y la persona que lo ha hecho por cincuenta años obviamente habrá logrado una mayor semejanza a Dios que uno que acaba de comenzar la vida cristiana. Pero mientras reconocemos diferentes niveles en lograr ser como Dios, todos los que son fieles, y entonces se están purificando como Él es puro, comparten igualmente la esperanza de los cielos. Ninguno se ha ganado su camino a los cielos. Todos dependen de la gracia de Dios, y su gracia se extiende a todos los fieles.

Que los jóvenes en la fe, entonces, no sean intimidados por los logros de los mas maduros en la fe. Que los maduros no vuelvan orgullosos y demasiado confiados. Todos deben ser fieles. Todos deben estar creciendo. Todos deben ser mas y mas como Dios. Y todos deben morir en el Señor. Y a todos los que lo hacen, el Señor dirá, «Bien, siervo bueno y fiel; en lo poco fuiste fiel, sobre mucho te pondré; entra en el gozo de tu señor».

\section{No Lo Pueden Vivir}
Muchos que nunca han entrado en el servicio del Señor explican su vacilación con las palabras, «Temo que no lo pueda vivir». Usualmente intentamos tranquilizar su miedo, pero la verdad es, ellos no lo pueden vivir, porque su concepto de lo que están intentando vivir es un concepto completamente erróneo.

Considera al hombre cuya esposa se convirtió en cristiana. Él estaba seguro de que ella no pudo «vivirlo». La miraba cuidadosamente, y en efecto, un día bajo estrés considerable, ella perdió control, gritó a los niños, y dijo unas cosas que una cristiana no debe decir. «Si ella fuera una cristiana», el hombre pensó, «ella no hablaría así; yo sabía que no lo pudo vivir». En otro da el predicador visitó y reaccionando a algo dicho por alguien, se puso algo rojo, aunque además de eso él controló su genio. Pero el hombre vio ese matiz rojo, e inmediatamente concluyó, «Ese predicador tampoco puede vivirlo». Él eventualmente observó fallas en otros cristianos, y finalmente concluyó que ninguno de ellos pudo «vivirlo», que la iglesia entera era un montón de hipócritas. Claro que él ni intento hacerlo, porque él conocía a sus debilidades demasiado bien. Él sabía que él no pudo «vivirlo».

¡Pobre hombre! Él piensa que la vida cristiana es una vida de perfección; que todos los «cristianos» se pueden clasificar en dos grupos: o son perfectos o son hipócritas. Él sabe que él no puede vivir perfectamente, el estándar que él ha establecido para otros, y está pensando con una actitud casi santurrón, «Una cosa, Predicador, yo no voy a ser hipócrita».

Pero hay perdón para las imperfecciones del cristiano sincero. «si alguno peca, Abogado tenemos para con el Padre, a Jesucristo el justo» (\ibibleverse{IJohn}(2:1)). Y ese cristiano que lucha sinceramente a vivir por el Señor, y diariamente busca Su perdón, aunque tenga mil debilidades, no es ningún hipócrita. Hay hipócritas en la iglesia, y no les ofrecemos ninguna defensa. Pero debilidades no implican necesariamente a la hipocresía. 

Que el cristiano fiel, entonces, no sea intimidado por la acusación constante de «hipócritas en la iglesia». Y que el pecador olvide eso de «vivirlo», si por eso él quiere decir la perfección, y que venga a Cristo, por medio de quien él puede ir a los cielos.

\section{Desalentados Por Los Hombres}
Leímos hace poco un artículo muy bueno por J.D. Jeffcoat titulado «Desalentador de Hombres». Ciertamente, iglesias en todos lados se perturban por los que constantemente desalientan a otros. Pero algo se debe decir tocante a los que se dejan desalentar también. 

Unos se desalientan por las críticas. Toda persona que ha intentado hacer la voluntad de Dios se ha criticado. Jesús se criticaba constantemente. Así Moisés también. Pablo dijo una vez, «Todos los que están en Asia me han vuelto la espalda» (\ibibleverse{IITimothy}(1:15)), y aparentemente él temía que Timoteo podría afectarse por esta crítica adversa contra Pablo (\ibibleverse{IITimothy}(1:8)). ¿Escuchas a otros cuando te critican? ¿Escuchas cuando critican a la iglesia local de la cual eres una parte o cuando critican a alguna buena persona dentro de la congregación? ¿Dejas que sus comentarios negativos te afecten los sentimientos? ¿O descartas sus comentarios desalentadores para que puedas mantener una actitud positiva hacia el Señor, Su iglesia, y Su obra?

Las críticas pueden ser justas. Pero, al otro lado, pueden ser el resultado de envidia hacia otros; o puede ser un intento deshonesto de destruir a otro; o puede ser simplemente el producto de ignorancia con respecto al bien y al mal, y entonces ignorancia de lo que merece las críticas y lo que merece el elogio. Uno no debe desalentarse en la obra del Señor por las críticas injustas.

Otros se desalientan por el pesimista crónico. Este es el hombre que siempre está recordando a la iglesia de que «no hará ningún bien», y que «esta simplemente es una área difícil y nunca vamos a convertir a personas aquí». Él en una reunión de negocios puede desmoralizar a una iglesia entera, y puede destruir en un momento lo que llevó un año para construir. Él es el verdadero «desalentador de hombres». Pero si él está mal en desalentar, otros están mal en dejarse desalentar por sus comentarios pesimistas. 

Los apóstoles sirven como el ejemplo perfecto de hombres que rehusaron desalentarse. Ellos constantemente enfrentaron reveses; su enseñanza se atacaba; ellos sufrían por los de adentro y por los de afuera. Se dijo de ellos: «Hasta el momento presente pasamos hambre y sed, andamos mal vestidos, somos maltratados y no tenemos dónde vivir; nos agotamos trabajando con nuestras propias manos; cuando nos ultrajan, bendecimos; cuando somos perseguidos, lo soportamos; cuando nos difaman, tratamos de reconciliar; hemos llegado a ser, hasta ahora, la escoria del mundo, el desecho de todo» (\ibibleverse{ICorinthians}(4:11-13)). ¡Pero ellos siguieron predicando! 

Bienaventurado aquel hombre que no se desalienta fácilmente por otros; que puede ver a través de las críticas indignas a lo que realmente son; que tome una posición en defensa de la verdad y el derecho y no deja que otros lo mueven de esa posición. Que evitemos igualmente el ser un «desalentador de hombres» como el ser un «desalentado por los hombres». Al contrario, que estemos hallados trabajando por el Señor, alentando y siendo alentados en la esperanza que tenemos en Cristo.

\section{La Clave Del Contentamiento}
«Y si tenemos qué comer y con qué cubrirnos, con eso estaremos contentos», dijo el apóstol (\ibibleverse{ITimothy}(6:8)); pero puede ser que el Señor nunca haya dado un mandamiento mas difícil de obedecer para cristianos del siglo veinte.

El secreto de contentamiento se puede ver en el carácter hermoso de Rut. Muy lejos de su hogar y con pocas provisiones, ella no se quejó, sino determinó perseguir la única manera disponible de proveer para si misma y para su suegra – la humilde, pero legítima, tarea de espigar. Comentando sobre la decisión de Rut, Pulpit Commentary dice esto de Rut: «Cuando ella no pudo levantar sus circunstancias a su mente, (ella) bajó su mente a sus circunstancias».

¡Ella bajó su mente a sus circunstancias! Allí está la clave. Si no puedo pagar un Cadillac, entonces yo necesito bajar mi mente para estar agradecido por el Reliant que sí manejo; si no puedo permitirme salir a comer tres veces por semana, entonces necesito bajar mi mente para disfrutar las salidas ocasionales que si puedo pagar; si no puedo pagar \$150 por llamadas telefónicas de larga distancia cada mes, entonces debo estar agradecido que aun hay un servicio postal y bajar mi mente a escribir cartas. Y así con la ropa que me pongo, la casa en que vivo, el recreo que disfruto, etc.

No encontramos la felicidad o la piedad en mirar con anhelo a lo que se les ajusta todos los demás; solo encontramos miseria e ingratitud. No tenemos que buscar lejos para encontrar esposas que están infelices con las circunstancias que sus maridos pueden permitirse, o hijos que se sienten privados a causa de las circunstancias que sus padres pueden permitirse, o predicadores que están miserables y buscando otro empleo a causa de su descontentamiento con los salarios que las iglesias o pueden o quieren pagar. Mientras reconocemos que hay una línea bajo la cual uno no puede vivir cómodamente, para la mayoría de estos, la felicidad se podría obtener. Ellos solo necesitan bajar sus mentes a sus circunstancias.

Que nadie piense que tal contentamiento viene automáticamente. Pablo dijo que él había aprendido «a contentarme cualquiera que sea mi situación» (\ibibleverse{Philippians}(4:10-13)). Y él había aprendido bien a esa lección. Él podía «bajar su mente» a circunstancias que incluirían una cárcel romana y «necesidades» que debían suplir. Él podía estar contento cuando lleno o cuando con hambre, cuando abundaba, o cuando sufría necesidad. El contentamiento de Pablo bajo tales circunstancias nos avergüenza de nuestro propio descontento; y cuando él habla de la verdadera fuente de su contentamiento, él nos hace cuestionar nuestra propia espiritualidad: «Todo lo puedo en Cristo que me fortalece».

Entonces, si somos ricos o pobres en los bienes de este mundo, que volvamos al Señor, aprendamos por medio de Él a bajar nuestras mentes a nuestras circunstancias, y así descubrir cuan verdaderamente «ricos» somos realmente… y entonces estar contentos. Que experimentemos la comprensión de que «la piedad, en efecto, es un medio de gran ganancia cuando va acompañada de contentamiento» (\ibibleverse{ITimothy}(6:6)).

\chapter{LA IGLESIA}

\section{La Iglesia Que No Se Encuentra En La Biblia}
La palabra «iglesia» se usa en dos sentidos en la biblia. (1) Se usa en el sentido universal, refiriéndose a los salvados de todos los siglos, y (2) se usa en el sentido local, refiriéndose a una congregación de personas salvas en cualquier localidad.

Entonces, leemos de Jesús dándose a sí mismo por la iglesia (\ibibleverse{Ephesians}(5:25)) y de Él comprando la iglesia con Su propia sangre (\ibibleverse{Acts}(20:28)). En estos dos pasajes la palabra «iglesia» se usa en el sentido universal. Jesús se dio a sí mismo por los salvos de todos los siglos y los compró con Su propia sangre.

Pero también leemos de la iglesia en Corinto (\ibibleverse{ICorinthians}(1:2)), la iglesia en Jerusalén (\ibibleverse{Acts}(11:22)), y las siete iglesias de Asia (\ibibleverse{Revelation}(1:11)). Estos pasajes se refieren a congregaciones de personas salvas en estas diferentes localidades. Estas iglesias locales se organizaron, cada cual con sus propios «obispos y diáconos» (\ibibleverse{Philippians}(1:1)).

La biblia no usa la palabra «iglesia», sin embargo, en el sentido denominacional. Uno no puede leer en las escrituras de conferencias, asociaciones, o sínodos denominacionales. Tampoco puede leer de jerarquías, nombres, credos o doctrinas denominacionales. Estos son los inventos de los hombres, no de Dios. Han sido la causa de muchas problemas en la religión.

¿Qué del lector? ¿Se considera a sí mismo una parte de esa iglesia universal compuesta de todos los salvos? ¿Es miembro de una iglesia local? ¿Su membresía en esa iglesia local lo lleva a una asociación con una organización denominacional – una organización mas grande que la iglesia local, pero mas pequeña que la iglesia universal? Si es así, él es una parte de una «iglesia» que no se encuentra en la biblia. 

Debe ser nuestra meta solo convertir a las personas a Jesucristo, traerlos a salvación por medio de Él, y guiarlos a una afiliación con una iglesia local que es una parte de ninguna organización denominacional – una iglesia local que se organiza como las iglesias locales de las escrituras. Este es el camino bíblico. ¿Para qué deberíamos continuar de perpetuar lo que Dios nunca fundó ni autorizó en Su palabra, la biblia?

\section{Ningún Órgano Rector Central}
Frecuentemente tenemos contacto con personas que no pueden entender como iglesias de Cristo funcionan sin algún órgano rector central para mantenerlas en línea. «Tienes una iglesia creyendo una cosa aquí, y otra iglesia creyendo otra cosa allá; tienes tantas diferencias entre tus iglesias», ellos dicen.

El error en esa declaración radica en el hecho de que «yo» no tengo una iglesia «aquí» y otra «allá». La iglesia de Cristo no se debe ver como una conglomeración de iglesias locales, la conglomeración teniendo tantas escuelas, tantos periódicos religiosos, tantos miembros en tantas iglesias. Yo no soy una parte de cualquier conglomeración así y nadie debe de serla. 

Como cristiano, yo soy una parte de dos cosas. (1) Yo soy una parte de un cuerpo celestial, la iglesia de Cristo, compuesta de todos los salvos de Dios de todos los siglos – unos aun en la tierra, otros en el paraíso. Este cuerpo celestial no tiene escuelas, no periódicos, no organizaciones. Solo Dios sabe de cuantos se compone este gran cuerpo. (2) Yo soy una parte de una iglesia local, unas de las cuales (yo confío) son partes de ese cuerpo celestial y unas de las cuales probablemente no las son. Iglesias locales parecidas existen por todo el mundo, pero cada cual existe como una entidad aparte bajo la autoridad de Jesucristo, el Príncipe de los pastores (\ibibleverse{IPeter}(5:4)). Pero ellas no son mis iglesias; ni soy una parte de cualquiera de ellas. 

La iglesia local de la cual yo soy una parte mira a Cristo (por medio del Nuevo Testamento) por sus instrucciones, no a algún órgano rector central. Si esa iglesia local deja de estar de acuerdo con Cristo, Él la quitará, ya no considerándola Suya (\ibibleverse{Revelation}(2:5)); no se quitará por algún órgano rector central ni por alguna otra iglesia local o grupo de iglesias. Si hay diferencias entre iglesias que se llaman a si mismas como «iglesias de Cristo», es porque unas ya no están de acuerdo con Cristo. Cristo quita esas iglesias que no están de acuerdo con Él, pero como ellas no son «mis» iglesias y yo no soy una parte de ellas, yo no llevo ninguna responsabilidad en esa decisión. Mi responsabilidad, entonces, es servir fielmente al Señor como una parte de Su cuerpo celestial; alentar a otros cristianos, dondequiera que estén, ser fiel a Cristo; y ayudar a mantener a la iglesia local de la cual yo soy una parte como una iglesia fiel de Cristo.

Ahora, habiendo explicado esto, preguntaríamos a nuestros inquiridores: ¿Preferirías ser una parte de una iglesia local que mira a un órgano rector central por sus instrucciones o una que mira solo a Jesús por sus instrucciones? ¿Preferirías ser una parte de una iglesia local que se puede quitar por un órgano rector central, compuesto de hombres falibles, o una que solo se puede quitar por Jesucristo? De mayor importancia, ¿de cual quisiera Cristo que seas una parte? Es maravilloso tener hermanos y hermanas en Cristo por todo el mundo, pero mi relación con ellos es por medio de una relación común con Cristo, no por medio de algún órgano rector central o conglomeración de iglesias.

\section{¡Qué Comunión!}
«¿Tiene la iglesia de Cristo un sumo sacerdote?», la «hermana» joven mormona preguntó. «Sí», fue nuestra respuesta enfática. «¿Tiene la iglesia de Cristo apóstoles?». Otra vez, «Sí» fue nuestra respuesta.

No dudamos que nuestra respuesta fue correcta, pero parece que somos naturalmente dispuestos a dar una respuesta opuesta. Nuestros conceptos de la iglesia del Señor pueden ser tan angostos. Hablamos de la «iglesia universal», y pensamos solo de los santos vivos alrededor del mundo. Pero la iglesia del Señor no se limita por el tiempo o el lugar. Es todos los cristianos de todos los siglos. Unos están en la tierra; unos en el paraíso; pero todos son una parte de la iglesia del Señor.

Cuando obedecemos el evangelio, el Señor nos salvó, y nos agregó a Su cuerpo. Aquí tenemos comunión con Bernabé, Esteban, Dorcas, Pedro, y otros grandes caracteres de las escrituras (\ibibleverse{Hebrews}(12:22-24)). Aquí tenemos comunión con los santos maravillosos quienes hemos conocido en nuestras vidas que ya se han ido a «estar con el Señor». Aquí tenemos comunión con los cristianos en la tierra hoy, unos conocidos, otros desconocidos. Esta es la iglesia de Cristo. Es el pueblo comprado por el Señor, comprado con Su propia sangre (\ibibleverse{Acts}(20:28)).

«¿Dejó de existir la iglesia del Señor durante los Siglos Oscuros?», alguien pregunta. ¡No! ¡No! Sabemos poco de los que vivieron en la tierra durante ese periodo. Pero para que la iglesia del Señor dejara de existir, el diablo hubiera tenido que entrar de alguna forma en el paraíso y destruir completamente a esos fieles que habían fallecido. Recuérdate, ellos son una parte de este gran cuerpo.

«¿Cuántos miembros hay en la iglesia de Cristo?». Solo Dios lo sabe. Seguramente no podríamos nosotros llegar a una figura precisa por sumar a todos los que adoran con iglesias locales «conocidas» por todo el mundo. Esto excluiría a los fieles que ya han fallecido. Además, puede haber cristianos fieles reuniéndose en alguna área remota desconocida por cualquier otro cristiano fiel. Y, por supuesto, solo Dios sabe quienes son verdaderamente fieles y así constituyen Su iglesia.

¿Quién es el sumo sacerdote de la iglesia? Es Jesucristo, Él «llamado por Dios» (\ibibleverse{Hebrews}(5:4-5)), perfeccionado por sufrimiento (\ibibleverse{Hebrews}(5:7-9)), que ahora ha transcendido los cielos y está sentado «a la diestra de Dios» (\ibibleverse{Hebrews}(4:14-16); \ibiblechvs{Hebrews}(10:10-12)), habiendo ofrecido Su propia sangre como una expiación por nuestros pecados (\ibibleverse{Hebrews}(9:11-12)). ¿Quiénes son los apóstoles de la iglesia? Son Pedro, Andrés, Jacobo, Juan, Felipe, etc. Estos son los embajadores especialmente escogidos quienes hablan oficialmente por el Rey, y ellos siguen hablando a la iglesia del Señor a través de las escrituras. Ellos son nuestros apóstoles, también.

¿Somos demasiado mundanos para apreciar lo que tenemos en Cristo? ¿Acaso no podemos levantarnos por encima de este mundo con sus conceptos angostos para vernos como una parte de un gran cuerpo que llega desde la tierra hasta el paraíso e incluso entra en los mismos cielos? Todos nos sentamos con Cristo «en los lugares celestiales». ¡Qué comunión! ¡Que privilegio tan glorioso!

\section{Buenas Iglesias Tienen Problemas}
«No problemas internas», el hombre dijo. Y nuestra primera reacción era una de admiración por esta iglesia «ideal» que no conocía a los problemas. Pero con más observación nuestro pensamiento cambió. 

La biblia habla de una iglesia que no tuvo «ningún problema». La iglesia en Laodicea era «rico, se había enriquecido y de nada tenía necesidad» (\ibibleverse{Revelation}(3:17)). Al otro lado la iglesia de Jerusalén se enfrentó con varios problemas. Tuvieron que ser testigos de la muerte de una pareja mentirosa e hipócrita (\ibibleverse{Acts}(5:1-11)). Hubo murmuraciones a causa de la negligencia hacia las viudas griegas (\ibibleverse{Acts}(6:1-7)). Hubo problemas doctrinales sobre la cuestión de la circuncisión (\ibibleverse{Acts}(11:1-18); \ibiblechvs{Acts}(15:4-5)). Jerusalén tenía sus problemas mientras Laodicea estaba «libre de problemas» - pero todo estudiante de la biblia sabe que Jerusalén era la iglesia aprobada mientras Laodicea daba nausea al Señor. 

Además, cuando uno observa los problemas de la iglesia de Jerusalén, reconoce que los problemas eran un resultado directo de la obra y actividad de esa congregación. Si no hubiera existido el espíritu de benevolencia que prevalecía entre sus miembros, no hubiera habido ocasión para la mentira de Ananías y Safira ni por las murmuraciones sobre negligencia. Si no hubiera habido evangelización entre los gentiles, no hubiera habido ningún problema sobre la circuncisión. Jerusalén tenían problemas porque ellos eran una iglesia trabajadora, activa, viva, creciente. Y bien puede ser que la ausencia de problemas en Laodicea fuera resultado directo de su tibieza y falta de vitalidad.

Concluimos que una iglesia perezosa, «hacer nada» bien puede estar libre de problemas, pero una iglesia activa y trabajadora puede esperar ciertos problemas. Una iglesia que logra convertir a alcohólicos, drogadictos, divorciados; que busca una «mujer samaritana» de nuestro día, o un «Simón el mago», o una «María Magdalena» puede anticipar unos problemas.

Pero esa iglesia que escoge la alternativa, predicando y convirtiendo solo a los de buenos morales que encajan bien en sus propios círculos sociales y económicos, mientras evitando unos problemas, enfrenta el problema mas grande de todos en no obedecer al mandamiento del Señor (\ibibleverse{Mark}(16:15)) y no seguir Su propio ejemplo. Una iglesia que desarrolla a personas pensadoras que estudian objetivamente toda cuestión bíblica por sí mismas puede esperar que unas diferencias surjan en su búsqueda sincera de la verdad. Una iglesia hospitalaria debe estar preparada para acusaciones de negligencia en su muestra de hospitalidad. Verdadero celo por el Señor engendrará problemas pero ay de esa iglesia que desatiende la obra del Señor para evitar problemas. La anatema del Señor cae sobre esa iglesia. 

No es la existencia o inexistencia de problemas, entonces, que determina la fuerza de una iglesia, sino como la iglesia maneja sus problemas. Amor por los otros, preocupación mutua, paciencia, humildad, amor de la verdad, determinación de hacer la voluntad de Dios – estas son las cualidades que hacen fuerte a una iglesia. No pueden prohibir que los problemas se desarrollen, pero sí pueden capacitar a una iglesia a llevar sus problemas a soluciones aprobadas de Dios.

\section{Los Espectadores Abucheadores}
¿Alguna vez has notado quienes abuchean en un juego de pelota? No son los jugadores en la cancha. Ellos han cometido sus propios errores y no están inclinados a abuchear a su compañero cuando el comete el suyo. Ellos se apoyan los unos a los otros, se animan, se ayudan. Ellos juegan como un equipo, ganan o pierden como un equipo, sufren juntos como un equipo, se regocijan juntos como un equipo. Son los espectadores que abuchean. Así es en toda área de la vida: son los espectadores, como regla, quienes critican, no los participantes. 

Desgraciadamente, en toda congregación existen los espectadores y los participantes. Los espectadores nunca enseñan una clase bíblica, ni predican, ni dirigen los cantos, ni presiden sobre la mesa; ellos realmente no se involucran mucho en la adoración. Pero más a menudo que no, ellos son los mismos que se encuentran criticando al predicador o al líder de los cantos o al maestro de la clase bíblica. Ellos son los que están tan avergonzados y enojados cuando alguien por accidente comete un «error» en sus esfuerzos por dirigir al grupo. Ellos vienen queriendo escuchar algo interesante que haga que el tiempo pase rápido. Si lo escuchan, ellos «aclaman»; si no, «abuchean».

No es asi con los verdaderos participantes, los que realmente están involucrados en la obra del Señor. Ellos son los que «aclaman» a ese «novato» quien está predicando su primer sermón o dirigiendo su primer canto. Cuando un hermano intenta enseñar su primera clase bíblica, ellos están buscando maneras de ayudarle. Ellos son simpáticos; se regocijan en el éxito de otros; lloran sobre las penas de otros; lo sienten por él que ha fallado, le hacen concesiones, lo alientan a intentarlo de nuevo, y lo aseguran que él hará mejor la otra vez. Ellos se regocijan especialmente en el desarrollo de los hombres y mujeres jóvenes en el trabajo del Señor. Ellos están tan nerviosos y emocionados cuando los jóvenes hacen su primer intento de presidir en la mesa o enseñar una clase como lo estarían si fueran sus propios hijos. 

¿Cuántos predicadores han decidido moverse a causa de los espectadores abucheadores justo al momento cuando los participantes disfrutaban su mayor crecimiento espiritual? ¿Cuántos ancianos han planificado la obra pensando mas en las demandas de los espectadores que en las necesidades de los participantes?

Los espectadores necesitan convertirse en participantes y descubrir cómo es estar allí «en la cancha». Los participantes necesitan seguir haciendo lo mejor que puedan, ignorando el «abucheo» mientras buscan la aprobación de su «Gerente» que todo ve y entiende. Todos necesitan estar preparándose para el juicio, donde serán los «hacedores de la palabra», no los críticos, que se salvarán. «En conclusión, sed todos de un mismo sentir, compasivos, fraternales, misericordiosos y de espíritu humilde» (\ibibleverse{IPeter}(3:8)). 

Cristianos, como jugadores de pelota, cometen un triste error cuando escuchan demasiado a los «abucheadores».

\section{Mejorando Nuestra Adoración}
Nuestra falta de fervor y reverencia en la adoración es un asunto de grave preocupación a toda persona de mente espiritual. Muchas veces nos hemos encontrado cantando, pero no adorando; inclinando nuestras cabezas, pero no orando; sentándonos por un sermón, pero no participando en un estudio de la palabra de Dios. Tal acción es una burla, trayendo condenación sobre el «adorador» en vez de la aprobación de Dios.

¿Qué es la solución del problema? Unos han buscado la solución en cantos espontáneos y cadenas de oración. Un grupo se reúne por un periodo de devoción. Ningún numero de canto se anuncia; alguien (cualquier persona) simplemente comienza un canto, y los demás siguen. En vez de una persona dirigiendo la oración, todos los hombres participan, cada uno agregando su poquito hasta que el último hombre en el circulo da el «Amén» final. El propósito detrás de esta práctica es ayudar a la gente a sentirse mas cerca a Dios mientras adoran.

Yo no estoy cuestionando si estas prácticas son bíblicas, pero si alguien piensa que ellos tienen la solución a nuestros problemas de adoración, yo creo que él está completamente equivocado. O si tales prácticas guían a los participantes a mirar con desprecio en los cantos «dirigidos» o oraciones «dirigidas», considerando tales una forma inferior de adoración, se vuelven realmente peligrosas. Mejoramiento en la adoración no es el resultado de cambiar el orden o lo externo de la adoración, sino por cambiar los corazones de los hombres. Viene por una fe mas fuerte y un mayor amor por el Señor. 

Cuando llegamos a amar el Señor y apreciar Su sacrificio como lo debemos, tales palabras como, «Estoy maravillado en la presencia de Jesús el Nazareno, y me pregunto cómo Él pudo amarme, un pecador condenado, inmundo», despertarán una respuesta inmediata en nuestros corazones, para que con fervor genuino cantaremos, «¡Cuán maravilloso! ¡Cuán magnífico! Y mi canto siempre será; ¡Cuan maravilloso! ¡Cuán magnífico! Me es el amor del Salvador». Y si el canto se canta espontáneamente o se anuncia y se dirige por un líder de cantos se convertirá en un asunto de indiferencia.

Cuando desarrollamos una verdadera consciencia de Dios – una consciencia de Su grandeza, Su presencia, Su preocupación, Su amor, Su consciencia, Su oído atento – y una apreciación de nuestra propia pequeñez e indignidad, comenzaremos a orar como debemos.

Cantos espontáneos y oraciones en cadena solo proveen ayuda temporánea en tratar la síntoma. Pero lo que necesitamos es llegar a la raíz de nuestros problemas, nuestra propia falta de fe y amor por el Señor.
Intentaremos, pero nunca en esta vida llegaremos a la perfección en la adoración. Pero algún día veremos a nuestro Señor. Una consciencia de lo que Él ha hecho por nosotros llenará nuestras almas como nunca antes. Una consciencia de nuestra desesperanza, si no fuera por Él, moverá nuestros espíritus. Y entonces – y posiblemente solo entonces – irrumpiremos en loor con la adoración sincera que Él merece. Y estamos bastante seguros de que no requerirá ninguna forma artificial o arreglo de adoración para motivar esa explosión de loor.

\section{¿Oras?}
El adorador que desea orar en la asamblea debe hacer mas que inclinar la cabeza y cerrar los ojos. Él debe orar. «De otra manera, si bendices solo en el espíritu, ¿cómo dirá el Amén a tu acción de gracias el que ocupa el lugar del que no tiene ese don, puesto que no sabe lo que dices?» (\ibibleverse{ICorinthians}(14:16)). Este versículo sugiere cuatro requisitos si uno va a entrar en una oración. 
\begin{enumerate}
\item Él debe escuchar a la oración. Uno no puede decir legitimamente «Amén» a la conclusión de una oración si él no ha escuchado la oración. «Vagancia de la mente» es un problema que siempre se presenta. Cantamos, pero no observamos las palabras del canto. Inclinamos nuestras cabezas, pero no escuchamos la oración. Nos sentamos por el sermón, pero nuestras mentes vagan a cosas de una naturaleza terrenal. Consequentemente, asistimos a periodos de adoración, pero no adoramos como debemos. Si uno va a orar, con la congregación, él debe escuchar la oración. 
\item Él debe entender la oración. Cuando un hombre en el primer siglo dirigió una oración en una lengua desconocida, el adorador no pudo decir «Amén», porque él no pudo entender el lenguage en lo cual la oración se habló. Tampoco puede el adorador decir «Amén» hoy en día si el líder no ha hablado lo suficientemente alto para ser escuchado o si él ha usado palabras o frases que el adorador no entiende. Los que dirigen oraciones en la asamblea deberían estar conscientes de las necesidades de la congregación entera, hablando alto para que todos puedan escuchar y usando palabras que todos puedan entender. 
\item Él debe estar de acuerdo con la oración. Hace unos anios, sentado al lado de un predicador mayor, observé que él decía «Sí» o «Sí, Señor» a la conclusión de cada frase distinta de la oración mientras se dirigía. Él hablaba las palabras tan suavemente que probablemente yo era el único en la asamblea que podía oirlas, pero me impresionó. Obviamente, este hermano escuchaba a cada fraase, determinando si él estaba de acuerdo con la frase o no, y entonces suavemente expresaba su acuerdo. Él no solo se sentaba durante la oración; él estaba orando. Ocasionalmente escuchamos sentimientos expresados en una oración con las cuales no estamos de acuerdo. A estos sentimientos no podemos decir «Amén».
\item Él debe decir «Amén». La palabra «Amén» significa «que así sea». Anhelamos escuchar el fuerte y sonoro «Amén» a la terminación de oraciones que antes escuchabamos. Tememos que el alejamiento de esta práctica es solo un paso más hacia la formalidad fria y muerta en nuestros periodos de adoración. No estamos contendiendo, sin embargo, que uno debe decir la palabra «Amén» de forma audible; pero sí estamos sugiriendo que por lo menos en su mente él debería decir «Amén», así haciendo que la oración sea su propia oración. Él ha escuchado la oración; él ha entendido la oración; él está de acuerdo con la oración; ahora él habla a Dios su «Amén» o aprobación de la oración como su propia oración. En esta manera él se une con otros adoradores en oración común a Dios.
\end{enumerate}

\section{Malgastando Tiempo En La Clase Bíblica}
Nos estamos volviendo más y más alarmados por el tiempo malgastado en las clases bíblicas discutiendo asuntos de ninguna importancia. Hace poco, en un estudio del «hombre de pecado» (\ibibleverse{IIThessalonians}(2:2)), escuchamos a un maestro preguntar acerca de la fuerza restrictiva que había detenido al hombre de pecado. La clase tuvo que haber pasado quince minutos discutiendo esa cuestión. Respuestas abarcaron desde lo dogmático hasta el «yo pienso» y el «pudo haber sido».

La verdad es, no sabemos qué era esa fuerza restrictiva. Dios no nos dijo. Entonces, realmente no importa, porque si hubiera importado, Él que nos ha dado «todo cuanto concierne a la vida y a la piedad» nos hubiera dado esa información también. Es una de esas preguntas como, «¿Por qué vino Nicodemo a Jesús de noche?» o , «¿Qué era el aguijón en la carne de Pablo?» o, «¿De qué color eran los caballos que jalaron el carruaje del eunuco?». Tiempo pasado discutiendo tales preguntas es tiempo malgastado. 

«Las cosas secretas pertenecen al SEÑOR nuestro Dios, mas las cosas reveladas nos pertenecen a nosotros y a nuestros hijos para siempre, a fin de que guardemos todas las palabras de esta ley» (\ibibleverse{Deuteronomy}(29:29)).
Debemos enseñar y predicar con signos de exclamación, no con signos de interrogación. Debemos predicar la palabra, no nuestras opiniones o los «yo pienso» o los «pudo haber sido» o los «puede ser». «Pero rechaza los razonamientos necios e ignorantes, sabiendo que producen altercados» (\ibibleverse{IITimothy}(2:23)). 

Ahora, yo no estoy planificando retirarme del maestro mencionado arriba, porque ese maestro resulta ser el escritor de este artículo. De alguna forma me es fácil ser paciente con él. Pero yo sí le estoy pidiendo un buen suministro de cilicio y cenizas, y lo estoy forzando a resolver a nunca volver a hacerlo. Y espero que su resolución alentará a otros maestros a resolver a usar sabiamente su tiempo en clase. Tiempo pasado en clases bíblicas es demasiado precioso para malgastarlo en cuestiones no reveladas.

«El que habla, que hable conforme a las palabras de Dios» (\ibibleverse{IPeter}(4:11)).

\section{El Propósito De La Iglesia}
¿Qué es el propósito de la iglesia del Señor? ¿Será erradicar la pobreza, las enfermedades, la injusticia social, o el analfabetismo entre los hombres? ¿Será llevar a cabo una cesación de guerra y conflicto? ¿Será luchar por una sociedad libre de tentaciones en la cual cristianos puedan vivir? 

Si la iglesia tuviera como una de sus grandes metas la erradicación de las \textbf{enfermedades}, el Señor fácilmente pudo haberla capacitado para lograr esa meta. ¿Acaso no pudo el mismo poder que permitió a un ciego a ver haber permitido a todos los ciegos a ver; que permitió a un cojo a caminar haber permitido a todos los cojos a caminar; que curó a muchas personas de varias enfermedades haber curado a todas las personas de todas las enfermedades? ¿Y acaso no pudo haberse dado este mismo poder a la iglesia en todas las generaciones?

Si la iglesia tuviera como una de sus grandes metas la erradicación de la \textbf{pobreza}, el Señor fácilmente pudo haberla capacitado para lograr este propósito. Después de todo, Él alimentó a los cinco mil con cinco panes y dos peces. De igual manera Él alimentó a cuatro mil en otra ocasión. ¿Acaso no pudo Él que hizo estas obras maravillosas haber capacitado a Su iglesia en todas las generaciones para alimentar, vestir y albergar a las multitudes empobrecidas del mundo por medio de poderes milagrosos?

Si el Señor hubiera querido que Su iglesia se convirtiera en un grupo de cabildeo para aplicar presión política con el fin de una \textbf{sociedad sin tentaciones y persecución} en la cual vivir, Él hubiera dado instrucciones acerca de eso. Él ni guio a Su iglesia en un esfuerzo directo para destruir la esclavitud, pero enseñó al esclavo cristiano a ser un mejor esclavo y al amo cristiano a tratar a sus esclavos como él quisiera que su Amo celestial le tratara a él (\ibibleverse{Colossians}(3:22-4:1)).

El propósito de la iglesia es salvar almas y preparar a personas por la eternidad. Extiende a los empobrecidos la esperanza de algún día caminar una calle de oro, a los que están sufriendo un tiempo cuando no habrá el dolor, a los que lloran un momento cuando Dios «enjugará toda lagrima de sus ojos». Dice a los tentados y perseguidos que hay valor en estas aflicciones, que la prueba de su fe es mas preciosa que el oro, y que regocijen. Dice a todos que vivan vidas piadosas en cualquier ámbito en que se encuentran. Busca cambiar a las personas por medio del poder del evangelio, no a la sociedad por medio del poder de legisladores. Sus armas «no son carnales sino poderosas en Dios». Su tema motivador: «¿qué provecho obtendrá un hombre si gana el mundo entero, pero pierde su alma?».

Cuando iglesias se involucran en el trabajo de hospitales y clínicas de salud, o cuando construyen escuelas por la educación de sus hijos, o cuando ellas ven como una de sus grandes misiones el proveer para la pobreza del mundo, o cuando ellas se sienten obligadas a crear convulsiones sociales y hacer campaña por derechos humanos, o cuando se sienten llamadas a expresar sus perspectivas sobre el uso del gobierno de las armas nucleares o lo que sea, ellas tienen una perspectiva distorsionada del propósito de la iglesia. 

\section{No Incentivos Carnales}
El uso de incentivos carnales para atraer a la gente a los periodos de adoración se está volviendo más y más común. Una iglesia ofrece gaseosa, donas, y globos gratis a sus pasajeros de bus. Otra ofrece un premio al niño que trae mas visitantes. Aun otra usa alguna figura del deporte o entretenedor para atraer una multitud. Cenas y eventos sociales, edificios muy elaborados; publicidad de «Iglesia más amigable en la ciudad» o «Iglesia de más rápido crecimiento en la ciudad»; la lista de trucos es casi sin fin. 

Un estudio de 1 Corintios, capítulos 1 y 2, sugiere que la gente de nuestra generación no son los primeros de demandar incentivos carnales. Los judíos y griegos en el día de Pablo los demandaron. «Porque en verdad los judíos piden señales y los griegos buscan sabiduría» (\ibibleverse{ICorinthians}(1:22)). Pablo pudo haber producido ambos, pero rehusó hacerlo: «Pero nosotros predicamos a Cristo crucificado, piedra de tropiezo para los judíos, y necedad para los gentiles; mas para los llamados, tanto judíos como griegos, Cristo es poder de Dios y sabiduría de Dios» (\ibibleverse{ICorinthians}(1:23-24)).

Pablo reconoció el hecho de que unas personas simplemente no son «convertibles»: «Pues considerad, hermanos, vuestro llamamiento; no hubo muchos sabios conforme a la carne, ni muchos poderosos, ni muchos nobles» (\ibiblechvs{ICorinthians}(1:26)). ¿Por qué son llamados infrecuentemente los tales? Porque la mayoría ponen su confianza en la carne en lugar de Dios. Ellos piensan carnalmente. Y Pablo no se interesó en atraer a personas «inconvertibles» que piensan carnalmente «a la iglesia» por el uso de señales, sabiduría, superioridad de palabra, palabras seductivas, o cualquier otro incentivo carnal. Hacer tal resultaría en lograr poner sus nombres en una «lista de membresía», pero no en llevarlos a Cristo para la salvación de sus almas. 

Si Pablo rehusó usar señales y sabiduría como incentivos carnales, ¿qué hubiera sido su reacción al uso de gaseosa y donas? Si él vino «no con superioridad de palabra o de sabiduría» (\ibibleverse{ICorinthians}(2:1)), ¿qué hubiera dicho él tocante a intentos de atraer multitudes por el uso de figuras del deporte y entretenedores? Si Pablo, guiado por el Espíritu Santo, siguió una regla de «no incentivos carnales», ¿no debe de ser esa nuestra regla? Y ¿no deberíamos siempre poder decir con Pablo, «pues nada me propuse saber entre vosotros, excepto a Jesucristo, y este crucificado» (\ibiblechvs{ICorinthians}(2:2))?

\section{Ancianos Locales O Junta Institucional}
Controversia considerable ha existido sobre el apoyo de iglesias para orfanatos. Muchos contienden que no hay autoridad en el Nuevo Testamento para tal (este escritor está entre ese número), mientras otros defienden la práctica. En este artículo hacemos varias preguntas relacionadas con la controversia, esperando clarificar la verdadera cuestión.

\textbf{¿A qué clase de orfanatos te refieres?} Nosotros estamos escribiendo acerca de orfanatos bajo el control de juntas institucionales, como Child Haven y Tennessee Orphan Home. Estas juntas constituyen agencias centrales que solicitan, colectan y distribuyen fondos de iglesias de Cristo y proveen supervisión para esas iglesias en la obra que se hace. No hay ninguna autoridad en el Nuevo Testamento por tales agencias centrales entre iglesias de Cristo. El Nuevo Testamento no sabe nada de organizaciones entre iglesias.

\textbf{¿No autoriza \ibibleverse{James}(1:27) el apoyo de iglesias para tales juntas en el cuido de viudas y huérfanos?} \ibibleverse{James}(1:27) dice, «La religión pura y sin mácula delante de nuestro Dios y Padre es esta: visitar a los huérfanos y a las viudas en sus aflicciones, y guardarse sin mancha del mundo». Nosotros valoramos la enseñanza de \ibibleverse{James}(1:27) e intentamos practicarla. Ningún hombre puede ir a los cielos que no practica religión pura y sin mancha como definida en este pasaje. Pero no podemos ver autoridad en este pasaje por apoyo de la iglesia por juntas institucionales cuando ninguno de los dos se menciona en el pasaje. De hecho, una lectura atenta de este versículo junto con el versículo anterior convencerá al lector de que la religión de un hombre está bajo consideración en vez de la acción de una iglesia.

\textbf{¿Estás diciendo que hay una distinción que se debe hacer entre la responsabilidad individua en la obra del Señor y la responsabilidad de la iglesia local?} Verdaderamente. Esta distinción se hace claramente en \ibibleverse{ITimothy}(5:16): «Si alguna creyente tiene viudas en la familia, que las mantenga, y que la iglesia no lleve la carga para que pueda ayudar a las que en verdad son viudas». Aquí hay algo que el individuo debe hacer que la iglesia no debe hacer.

\textbf{¿La iglesia sí tiene una responsabilidad de ser benevolente, verdad?} Sí. El versículo que se acaba de citar dice que una de las responsabilidades de la iglesia en la benevolencia es la de aliviar a las viudas verdaderas. Otros pasajes que enseñan con respecto a la responsabilidad de la iglesia en la benevolencia son \ibibleverse{Acts}(2:44-45); \ibiblechvs{Acts}(4:34-35); \ibiblechvs{Acts}(6:1-6); \ibibleverse{Romans}(15:25-33); \ibibleverse{ICorinthians}(16:1-4); y \ibibleverse{IICorinthians}(8-9:). Recomendamos que estos pasajes y cualquier otro pasaje que se tratan de la acción de la iglesia en esta área se considere cuidadosamente. 

\textbf{¿Pero no implica la «benevolencia» un hogar para el cuido de los necesitados?} Puede o no, pero nosotros no nos opondríamos a que la iglesia provea instalaciones o comida o ropa o supervisión, si estas se necesitaran para cumplir sus responsabilidades benevolentes. Ves, la verdadera cuestión no es si iglesias pueden proveer estas cosas, sino si iglesias pueden rendir fondos y supervisión a una junta central que de su parte debe proveer estas cosas. La junta institucional, o en otras palabras, la organización entre iglesias, es la cosa bajo consideración. El Nuevo Testamento no autoriza estas agencias centrales. 

\textbf{¿No participaron las iglesias del Nuevo Testamento en obras benevolentes?} Si, pero cada iglesia hacía su obra benevolente por medio de su propio marco organizacional. Cuando la tarea de servir mesas se hizo demasiado grande para los apóstoles (\ibibleverse{Acts}(6:1-6)), ellos no sugirieron el nombramiento de una junta institucional fuera del marco de la iglesia local, la cual en turno serviría en el futuro como una organización entre iglesias para todas las iglesias que después se establecerían. Al contrario la iglesia se mandó «escoged de entre vosotros siete… a quienes podamos encargar esta tarea». Simplemente estamos diciendo que cada iglesia debería hacer su propio trabajo benevolente bajo la supervisión de sus propios ancianos asistidos por sus propios diáconos. Este es el plan bíblico.

\textbf{¿Supongamos que surjan condiciones que hacen que sea imposible para la congregación proveer para sus propios necesitados?} Tales condiciones sí surgieron en el Nuevo Testamento (\ibibleverse{Acts}(11:27-30); \ibibleverse{Romans}(15:25-26)). En esos casos, iglesias que podían hacerlo enviaron apoyo a las iglesias necesitadas, permitiendo que esas iglesias necesitadas proveyeran para los suyos. Pero el lector haría bien en observar que los fondos no se enviaron a una junta institucional, sino a los ancianos de las iglesias necesitadas (\ibibleverse{Acts}(11:27-30)). 

En resumen, alentamos a todo cristiano a practicar la «religión pura y sin mancha» al máximo de su capacidad. Alentamos a iglesias a proveer por sus propios necesitados bajo la supervisión de sus propios ancianos. Y, alentamos a cristianos en todo lugar a pensar de nuevo sobre las agencias centrales que se han creado entre iglesias de Cristo. «Y todo lo que hacéis, de palabra o de hecho, hacedlo todo en el nombre del Señor Jesús…» (\ibibleverse{Colossians}(3:17)).

\section{El Evangelio Es Para Todos}
La iglesia que agrada a Dios se extenderá a toda clase de hombre: a los pobres, los desafortunados, los depravados, a todos los que necesitan salvación por la sangre de Cristo. Ella no limite su influencia a personas que son compatibles socialmente y económicamente. El evangelio es para todos.

Jesús enseñó esta lección igual en palabra como en acción. Él mostró la misma preocupación por una samaritana inmoral como por un gobernador joven y rico; la misma preocupación por un publicano llamado Zaqueo como por un fariseo llamado Simón. Sus seguidores mas estrechos eran de las clases pobres de la sociedad. Pero había hombres ricos que se influenciaron por Él también. Él buscó traer a todos los hombres a Sí mismo. Y cuando Él comisionó a Sus apóstoles, Él dijo, « Id por todo el mundo y predicad el evangelio a toda criatura…» (\ibibleverse{Mark}(16:15)).

Cristianos no necesitan encontrar puntos en común en sus posiciones sociales o económicas, sino en su amor mutuo por el Señor, en su fe mutua, y en su esperanza compartida de la vida eterna. Cuando personas realmente aman al Señor, distinciones de una naturaleza mundana se desvanecen a la nada. El rico se regocija «en su humillación» mientras el hermano de condición humilde se regocija «en su alta posición» - los dos llegando a una igualdad en Cristo (\ibibleverse{James}(1:9-10). Si uno no puede encontrar espacio en su corazón para personas de toda clase, carece gravemente de algo en su fe y devoción al Señor. 

La iglesia primitiva aprendió bien esta lección. Ellos predicaron igual a los pobres como a los ricos; a los de baja condición como a los gobernadores. Sus conversos incluyeron incluso a los samaritanos y los gentiles. Los que eran repugnantemente inmorales se cambiaron por el evangelio. Y cuando alguna descortesía se hizo a los pobres o los marginados, reprimendas severas se administraron (\ibibleverse{Galatians}(2:11-14); \ibibleverse{James}(2:1-9)). 

¿Hay alguien leyendo este artículo que se siente desanimado, quien está seguro de que nadie lo quiere porque él es tan pobre, o ha sido tan inmoral, o es discapacitado, o sin amigos? Si tú eres aquel, déjame asegurarte que el Señor te quiere; y hay personas que te quieren, también. El evangelio es para todos.

Hay alguien leyendo este artículo que se avergüenza cuando alguien obviamente pobre, o alguien de reputación inmoral o alguien en circunstancias desafortunadas entra en la asamblea de adoración? Si tú eres aquel, tienes que arrepentirte. Con ese tipo de actitud, te hubieras sentido muy incomodo en la presencia del Señor mientras Él estaba en la tierra.

La iglesia se compone de personas santas y piadosas que luchan para ser como su Maestro. Pero la iglesia se extiende a personas impiadosas de toda clase en la esperanza de que ellos puedan transformarse en la imagen de Cristo y ser salvos eternamente. «Cristo Jesús vino al mundo para salvar a los pecadores…» (\ibibleverse{ITimothy}(1:15)).

\section{Actitudes Hacia Los Débiles}
La actitud del Mesías hacia los espiritualmente débiles se representa por Isaías en las palabras siguientes: «No quebrará la caña cascada, ni apagará el pabilo mortecino; con fidelidad traerá justicia» (\ibibleverse{Isaiah}(42:3)). 

En nuestro celo por pureza y fuerza en la iglesia podemos ser culpables de hacer la misma cosa que nuestro Señor no quiere que se haga. Creamos una imagen mental de lo que la iglesia ideal debe de ser, y entonces comenzamos a establecer una. Todo miembro va a asistir a todo servicio. Todo miembro será «sano» en sus convicciones. Mundanidad no se tolerará. Esta iglesia no va a tener las debilidades que caracterizan a otras iglesias que conocemos. Esta será una iglesia fuerte, una iglesia modelo.

Un nuevo converso se hace, e inmediatamente él se indoctrina en lo que es esta iglesia, y la contribución que se espera que él haga para mantener este ideal. 

Cada recién llegado se mira como una amenaza potencial. Si él no va a conformarse, no lo queremos. Miembros más débiles se manejan con una actitud de «mejórate o ándate». La gente pronto reconoce que hay mucho mas preocupación por la imagen de la iglesia como una organización que por ellos como hijos de Dios débiles y batallados. Mientras el Mesías tiernamente y delicadamente está cuidando a estas «caña cascadas» para que vuelvan a la salud, puede ser que nosotros estemos allí quebrándolas. Mientras Él ahueca Sus manos alrededor de estas llamas aleteantes y tenuemente ardientes para proteger el fuego que queda, puede ser que nosotros estemos allí apagándolas.

No estamos sugiriendo que falsos maestros impenitentes y miembros inmorales se deban tolerar. Hay que advertirlos, observarlos, y retirarse de ellos. Tampoco estamos sugiriendo que los débiles se deben dejar solos en su debilidad. Hay que enseñarlos, alentarlos, reprobarlos, reprenderlos, y exhortarlos, pero con toda paciencia, y con el fin de esforzarlos. «Amonestéis a los indisciplinados, animéis a los desalentados, sostengáis a los débiles y seáis pacientes con todos» (\ibibleverse{IThessalonians}(5:14)). No preguntes que pueden hacer ellos por la iglesia, sino que puede la iglesia hacer por ellos. 

Mientras haya un poco de vida en esa «caña cascada», hay esperanza. ¡No la aplastes! Mientras quede un poco de fuego, puede ser posible avivarlo a arder mas intensamente. ¡No lo apagues!

\section{¿Reinstaurado?}
Estamos escuchando unas expresiones bastante raras estos días. Un hombre nos contaba hace poco que él se había presentado ante la congregación el domingo anterior para ser «reinstaurado». Una mujer dijo que ella pensaba en «volver a estar en la iglesia». ¿Son tales expresiones un problema de terminología? ¿O nos enfrenta un problema de entendimiento? ¿Está la gente pensando en la iglesia como una organización parecida a la asociación de padres y maestros de la escuela, adentro y afuera de lo cual ellos pueden ir como quieran, volviendo simplemente para ser «reinstaurado»? 

Infidelidad no es solo un asunto de salir de la iglesia por un tiempo. Es un asunto de hollar «bajo sus pies al Hijo de Dios», de ultrajar «al Espíritu de gracia», de dar la espalda al Señor y a Sus promesas, de vivir en pecado y coquetear con la condenación eterna. Imagínate, una persona estando en tal posición, y entonces ¡¡¡presentándose ante la congregación para ser «reinstaurado»!!! Lo que esa persona necesita hacer es arrepentirse, caer de rodillas ante Dios, y confesar a Él y a sus hermanos, «yo he pecado», y clamar a Dios por misericordia. 

No estamos cuestionando la misericordia de Dios. Él está preparado a perdonar a Su hijo descarriado; de correr a él, echarse sobre su cuello, besarlo, vestirlo con la mejor ropa, poner un anillo en su dedo y zapatos en sus pies, matar el becerro engordado y regocijarse. Pero este perdón abundante es para aquel hijo descarriado que vuelve con un pleno reconocimiento de su pecado e indignidad, quien confiesa sus pecados, quien pide, no ser «reinstaurado» como un hijo, sino que lo reciba de nuevo solo como un trabajador. Perdón es para los penitentes. 

Que el Señor nos ayude a ver el pecado en su verdadera fealdad, a «aborrecer lo malo» y «adherir a lo bueno», a recordar el precio pagado por nuestro Señor por nuestro perdón, a ser fieles, y a reconocer humildemente nuestros pecados cuando sí caigamos. Y cuando seamos perdonados, que no hablemos con labia de «volver a estar en la iglesia», sino que hablemos de la gracia de Dios que pudo salvar «un miserable como yo».
«Bienaventurados los que lloran, pues ellos serán consolados» (\ibibleverse{Matthew}(5:4)).

\chapter{ENSEÑANDO/DIRIGIENDO}

\section{Buen Liderazgo}
¿Cuáles son las características de buen liderazgo? Buen liderazgo tiene visión para ver lo que hay que hacer. Buen liderazgo se mueve hacia adelante, es positivo en su enfoque, promueve la confianza en otros, y los convence de que la tarea imposible se puede realizar.

Buen liderazgo tiene fe en la gente. Buen liderazgo cree que otros quieren trabajar y que responderán cuando se desafían apropiadamente; pone la mejor interpretación posible en las acciones de otros. Buen liderazgo «todo lo sufre, todo lo cree, todo lo espera, todo lo soporta», porque buen liderazgo ama.

Buen liderazgo no corre adelante de los otros. No hace todo por sí mismo. De hecho, a menudo ello se pone a un lado y espera – a veces con ansias – mientras otros se dan una oportunidad de realizar las tareas que ellos son capaces de realizar. Buen liderazgo no se preocupa tanto por realizar cosas como por desarrollar personas en siervos útiles y maduros del Señor. Buen liderazgo constantemente está produciendo liderazgo en otros.
Buen liderazgo realmente se preocupa por otros y tiene la habilidad de comunicar esa preocupación. Buen liderazgo es paciente, comprensivo; No es demasiado rápido para reprender ni indulgente con el pecado. Buen liderazgo se pone en la posición del otro hombre para ver las cosas desde su punto de vista.

Buen liderazgo es humilde, es dispuesto a reconocer errores; puede aceptar las críticas y separar las constructivas de las destructivas. Buen liderazgo busca el elogio de Dios en lugar del elogio de los hombres; sacrifica la popularidad para hacer la voluntad de Dios. 

Buen liderazgo tiene convicción, pero no es terco ni testarudo. Escucha a otros y ve sus ideas objetivamente. Buen liderazgo trata a todos igual; es imparcial. Buen liderazgo es franco y sincero, pero es amable.
Buen liderazgo es seguro de sí mismo, pero no orgulloso; no tiene que promocionar a sí mismo.

La iglesia necesita hombres y mujeres que son líderes, pero cuán grande la diferencia entre los que buscan liderar y los que verdaderamente lideran.

\section{El Costo De La Influencia Y La Reputación}
Hay personas en este mundo que poseen una habilidad natural de liderar y exigir el respeto de otros. Llámalo encanto, carisma, magnetismo, o lo que sea, tales personas ejercen una influencia poderosa sobre los que los admiran como la realización de lo que ellos mismos quisieran ser.

Aparentemente Pedro poseyó tales cualidades entre los apóstoles; y hubo David, Débora, Nehemías, y otros. Hemos conocido tales personas en nuestro día y hemos sido influenciados por ellos. Cada lector probablemente puede pensar de algún «héroe» de fe que él o ella ha admirado a través de los años.

Las oportunidades para bien que tales personas poseen son tremendas; pero así también son las responsabilidades. Es verdad que pecado es pecado, quienquiera que lo cometa; que el pecado separará a uno de Dios tan rápido como a otro. Pero las consecuencias adversas del pecado de uno se aumentan dramáticamente con el aumento de la influencia y reputación que él disfruta entre otros. La confianza de otros es una confianza que se debe proteger cuidadosamente. Una vez que esa confianza se establece, la persona a la cual se entrega tiene responsabilidades que otros de influencia y reputación mas normal no tienen; y entre mas personas se involucran en la confianza, mayor la responsabilidad.

Los de reputación deben estar preparados para un mayor escándalo publico cuando pecan. Natán le dijo a David que a causa de su adulterio, él había «dado ocasión de blasfemar a los enemigos del Señor» (\ibibleverse{IISamuel}(12:14))). Otros habían cometido adulterio en Israel, y su adulterio no se había notado por los enemigos de Dios. ¡Pero este era David! Era inevitable que el pecado de este solo hombre de influencia y reputación resultaría en mayor escándalo que los pecados de una multitud de personas de menor influencia y reputación.

Los de reputación deben estar preparados para reprimendas mas severas cuando pecan que los de menor reputación. Pablo habla de oponerse a él «cara a cara» cuando Pedro se apartó de comer con los gentiles (\ibibleverse{Galatians}(2:11-14)). La reprimenda de Pedro fue «delante de todos». Pedro no era el primer cristiano judío de rehusar comer con cristianos gentiles, pero Pablo obviamente reconoció la gravedad de las acciones de Pedro a causa de su mayor reputación e influencia. Otros seguían su ejemplo en esta ocasión, incluyendo a Bernabé. Pedro no pudo disfrutar el lujo de una reunión privada con Pablo; Pedro tuvo que enfrentar la picadura de una reprimenda inmediata y abierta. Pedro había traicionado una confianza. Nada menos que una reprimenda abierta pudo contrarrestar el daño que resultaba. Reprimendas mas severas simplemente son un costo – un costo inevitable – de influencia y reputación.

Los de reputación deben vivir con mas cuidado que otros si quisieran mantener su influencia y buen nombre. Todo cristiano se advierte a no poner una piedra de tropiezo en el camino de su hermano (\ibibleverse{Romans}(14:13); \ibibleverse{ICorinthians}(8:9), pero uno que se conoce y se admira por miles de hermanos en muchos lugares obviamente tendrá que tener mas cuidado que uno que se conoce y se admira por solo unos pocos hermanos en su local. Pablo tuvo que sacrificar mucho más para ser «todo para todos» que algún cristiano que nunca había estado fuera de su propia comunidad. Eso es simplemente el costo de la influencia y la reputación. Si uno no está dispuesto a pagar ese costo, si él está determinado a ser inflexible en su conducta «sin importar lo que piensen los demás» - él necesita llegar a un mayor aprecio del valor de un buen nombre (\ibibleverse{Proverbs}(22:1)).

Los de reputación especialmente deben tener cuidado de construir sobre Jesucristo, el verdadero fundamento, en lugar de sobre sí mismos. Las palabras, «Porque no nos predicamos a nosotros mismos, sino a Jesucristo como Señor», deben convertirse en su lema (\ibibleverse{IICorinthians}(4:5)). Los que ponen su lealtad en hombres de renombre y reputación están en error. Su fe no es lo que debe de ser. Pero los que deliberadamente usan carisma y adulación para atraer seguidores también están en error (\ibibleverse{IThessalonians}(2:1-13)). Entre más carisma con que se bendice (?) uno, más precaución él debe tener.

Cuando «Descalzado» Joe Jackson, una estrella outfielder de los Chicago White Sox, se involucró en el escándalo «Calceta negra» de los años 1920, y estaba en camino al juicio, un niño pequeño, lastimado, decepcionado, con lagrimas en los ojos, se reportó decir, «Diga que no es así, Joe, diga que no es así». Cada lector probablemente es el héroe de alguien. Otros lectores son hombres y mujeres de influencia extendida. Que cada uno, cuando esté tentado, y antes de ceder, mire adelante a las lágrimas, y el dolor y la desilusión que él va a traer a los que lo admiran; que él escuche sus gritos potenciales de «Diga que no es así, Joe»; y, motivado por la confianza de ellos y su propio amor por el Señor, que «resiste al diablo». Porque, si él traiciona la confianza que se le ha cometido, él se puede salvar eternamente por medio del arrepentimiento y perdón – ¡por esto estamos tan agradecidos! – pero probablemente él nunca recuperará la confianza que haya perdido. Correcto o incorrecto, es la realidad. Es el costo – el costo inevitable – de la influencia y la reputación.

\section{Creemos, Por Lo Cual Hablamos}
La predicación de los apóstoles precedió de corazones llenos de convicción. Ellos habían observado los milagros de Jesús; habían escuchado Su enseñanza, tres de ellos habían estado con Él en el monte, y habían escuchado aquellas palabras habladas desde los cielos, «Este es mi Hijo amado en quien me he complacido; a Él oíd»\ibible{Matthew}(17:1-8); ellos habían visto su compostura cuando él se arrestó en el huerto; habían examinado cuidadosamente las pruebas de Su resurrección, y habían hablado con Él, habían comido con Él, lo habían palpado; habían mirado mientras Él ascendió hasta que una nube lo recibió y lo ocultó de sus ojos. ¡Ellos creían! Y eran tan llenos de fe que su fe rebosó en palabras. Ellos no podían detener ese mensaje que ardía en sus corazones. 

Lo predicaron en todos lados: en las sinagogas y en el templo, en las calles y de casa en casa, en las mansiones de gobernadores y en las cárceles, en los mercados y desde el Areópago, en barcos y en carruajes, en aposentos y en las riberas. 

Lo predicaron a los ricos y a los pobres, a adoradores de Dios y a adoradores de ídolos, a los humildes y a los orgullosos, a los de poca fuerza y a los poderosos, a los educados y a los deseducados, a los buenos y a los malos, a los morales y a los inmorales.

Lo predicaron a borrachos, a adúlteros, a homosexuales, a idolatras, a magos, a reyes, a oficiales del gobierno, a oficiales del ejército, a carceleros, a capitanes de barcos, a mendigos. ¡Lo predicaron y lo predicaron y lo predicaron!

Porque lo predicaron se encarcelaron, se golpearon, se azotaron, se apedrearon, se ridiculizaron, se amenazaron; fueron víctimas de mentiras, engaño, conspiración, alborotos, violencia callejera, naufrago, emboscada; vivieron en pobreza, a menudo con hambre, y con «ninguna morada segura»; pero nada pudo detenerlos de predicar mientras tenían aliento. Ellos creían, y su fe los forzaron a predicar sin importar las circunstancias. Ellos creían, por lo cual hablaban (\ibibleverse{IICorinthians}(4:13)).

¿Qué predicaron? Predicaron aquel mensaje que les fue revelado por el Espíritu Santo, el mensaje divino de Dios (\ibibleverse{ICorinthians}(2:6-13)). No tenían tiempo para opiniones, filosofías humanas, o lo político, ni se centró su fe en tales cosas. El mensaje que ardía en sus corazones era un mensaje del Cristo y de la salvación por medio de Él. Ellos creían en Cristo, en la eficacia de Su sangre, y en el poder de Su evangelio. Ellos creían, y entonces hablaron ese mensaje y solo ese mensaje.

¡Fe! Puede ser que dentro de esa sola palabra radique el elemento más importante de la evangelización eficaz. Cuando lleguemos a creer como esos apóstoles, cuando estemos tan llenos de fe que apenas podemos detenernos, cuando el mensaje de salvación arda dentro de nosotros como lo hacía en ellos, estaremos enseñando a otros y estaremos haciéndolo eficazmente. Hasta aquel entonces nuestras palabras pueden contener una vacuidad inconfundible que les quitará el poder de cambiar los corazones de los hombres. 

\section{Extendiendo La Palabra De Vida}
Hola, Simón, ¿oíste del incidente arriba en el templo ayer? Llegamos un poco tarde para la hora de oración, y cuando llegamos allí, escuchamos este terrible alboroto en el pórtico de Salomón. Pedro y Juan justo habían curado a un cojo – ellos siempre están intentando atraer la atención, sabes – y Pedro hacía la predicación como normal.

Pues, Simón, debiste haber escuchado el sermón. No, al pensarlo mejor, me alegro qué te salvaste de la agonía. ¿Sabes que Pedro predicó el mismo viejo sermón que él usó en el Pentecostés? La gente se cansa de escuchar la misma vieja cosa. Y él era tan ofensivo. Él actualmente los acusó de ser ignorantes y dijo que ellos habían sido culpables en la crucifixión de Jesús. Eran culpables, por supuesto, pero Pedro no tiene que ser tan franco en sus caras	Y ¡escritura, escritura, escritura! Me canso tanto por escuchar escrituras todo el tiempo. Yo pensé que él nunca terminaría. Y, puedes creerlo, fue la primera vez que yo había decidido llevar a Josef conmigo. ¡Estuve simplemente avergonzado! Yo me disculpé por lo que pasó, pero estoy seguro que él jamás volverá a ir conmigo. De hecho, puede ser que ni me vuelva a hablar. 

¡Pero eso no es todo! Antes de terminar Pedro de predicar, el sacerdote y el capitán del templo y los saduceos vinieron y arrestaron a Pedro y Juan allí mismo. Simón, ¡estuve tan avergonzado! Nunca haremos nada bueno con toda de esta mala publicidad. Ellos juntan una buena multitud y después lo arruinan todo por ser arrestados. Los apóstoles no saben ni una cosa de ganar el favor público.

Y todo lo que ellos quieren es dinero. Probablemente quieren que más personas vendan sus posesiones. ¡No van a sacar el mío! No soporto los fanáticos; además, aquellas personas solo están presumiendo dando todo ese dinero. Y, a propósito, Simón, yo oigo que están contando el numero de conversos de nuevo. Hombre, realmente les gustan los números. Pero deja que cuenten. Como van las cosas, no quedará una iglesia alrededor de aquí. Simplemente podemos alistarnos para cerrar las puertas. Nadie quiere ser parte de una iglesia donde los predicadores predican la misma vieja cosa todo el tiempo y se arrestan constantemente. 

¿Dijiste algo, Simón? ¿Qué dijiste? ¿El número es qué? ¿El número de los hombres ha llegado a… a… 5,000\ibible{Acts}(4:4)?

\section{Si Estás Planeando Predicar}
Si estás planeando predicar, presta mucha atención a los siguientes tres elementos básicos de predicación exitosa. 
\begin{enumerate}
\item \textbf{Espiritualidad.} El predicador fiel tiene una profunda devoción a Dios y Su palabra. Él ora a menudo. Él vive con una consciencia constante de la proximidad de Dios; él «anda con Dios». Su carácter es intachable. Él odia el pecado y el error, pero ama la verdad y la justicia. «en la ley del Señor está su deleite, y en su ley medita de día y de noche». Si tú quieres predicar, pero careces de estas cualidades, espera entonces. Desarrolla estas primero. Difícilmente puedes motivar a otros a convertirse en lo que tú no te has convertido. 
\item \textbf{Preparación.} Aprende a \textbf{leer}. Comprensión lectora es una herramienta esencial para entender las escrituras y libros relacionados a ellas. Aprende a \textbf{pensar}. Muchas de las cosas que leerás en comentarios e incluso en periódicos dentro de la hermandad serán falsas. Ay de ese hombre que en su predicación y enseñanza de la clase bíblica solo repite como loro lo que él haya leído. Aprende a \textbf{diferenciar}: entre verdad y error, entre hecho y suposición, entre lo que se enseña claramente y lo que está en el área «gris», entre lo que se aplica a la congregación y lo que se aplica al individuo, entre el tiempo para abordar un tema sensitivo y el tiempo no adecuado para abordar un tema sensitivo. «Sed astutos como las serpientes e inocentes como las palomas» (\ibibleverse{Matthew}(10:16)). Aprende \textbf{valor}. Estarás enfrentado con presiones e intimidación desde todos lados. Para unos, predicarás demasiado duro; para otros, demasiado suave. Unos te presionarán a hacer concesiones; otros intentarán forzarte a tomar alguna posición «de línea dura» que realmente no puedes encontrar en las escrituras. Complace a Dios. Encaja bien con los hermanos en todo lo posible, pero no al riesgo de perder tu alma. Si problemas se desarrollan, échate un largo y duro vistazo al «yo». Tememos que muchos hombres hayan creado una crisis «doctrinal» en un intento de cubrir su propia mala disposición y actitud dictadora. 
\item \textbf{Presentación.} Ya no eres un niño de la escuela actuando una parte en el escenario. Tampoco eres cómico siendo pagado para entretener una audiencia. Eres un hombre moribundo predicando a hombres y mujeres moribundos el único mensaje en la tierra que puede salvar sus almas. Habla a \textbf{ellos}. Ayúdalos a ver sus pecados, y señalarlos el Salvador. Habla con la calidez, y amor, y sinceridad que les dejará saber que ellos mismos te importan y el destino eterno de ellos. Lo que dices puede ser mas importante, pero como hablas es importante, también.
\end{enumerate}
L.A. Mott, Jr. (\textit{Notebook on Jeremiah}, página 67), escribe: «Alguien ha dicho que hay tres tipos de predicadores. El primero tiene que decir algo – él es un hablador pagado que tiene que llenar una cierta cantidad de tiempo cada semana. El segundo tiene algo que decir, y eso es mucho mejor. Pero el mejor de todos es el tercero – el hombre que tiene algo que decir \textbf{y tiene que decirlo}. Eso es el tipo de predicador que era Jeremías». Y eso es el tipo de predicador que tú debes ser.

\section{Consígame Un Trabajo Predicando}
Se nos ha pedido escribir acerca del peligro del profesionalismo entre predicadores, un peligro de lo cual muchos están inconscientes. 

Debemos prologar nuestras advertencias con un reconocimiento agradecido que hay muchos hombres fieles y dedicados por todo el mundo que han devotado sus vidas a predicar. Muchos de estos son hombres jóvenes cuya espiritualidad, amor por la verdad, y preocupación por almas los ha compelido a compartir el evangelio con otros. Ellos están dispuestos a ir a donde se necesitan en vez de los lugares donde el salario y el prestigio son mas grandes. Ellos ven la predicación como un servicio que debe realizarse en un espíritu de sacrificio y desinterés. Hay bastante espacio en el reino de Dios para tales hombres.

Pero la predicación también se puede ver como una carrera que se prosigue con metas y ambición egoísta. Hubo un tiempo cuando pocos hubieran predicado por consideraciones económicas. Pero los tiempos han cambiado. Los salarios de predicadores se han aumentado. Iglesias están clamando por predicadores. Hay oportunidades allí, mientras empleo en otros campos es difícil de encontrar. De hecho, probablemente hay pocos trabajos donde un hombre joven puede comenzar con un salario mas alto que el la predicación, y especialmente si él decide presionar a la iglesia por cada centavo que «él puede sacar de ellos». Además, no hay ningunos requisitos de educación o de formación profesional. Los únicos requisitos en unos casos son algo de conocimiento bíblico, una buena personalidad, y el don de hablar. Agrega a esto el glamour de pararse enfrente de audiencias y ser admirado, y obviamente la tentación puede ser grande por un hombre joven que no puede encontrar trabajo de otra manera decidir a predicar. 

Tales hombres pueden realizar algún bien si ellos realmente predican a Cristo, y nos regocijaremos en ese bien (\ibibleverse{Philippians}(1:12-18)). Pero ellos mismos son tan vulnerables a la tentación de hacer concesiones, de halagar, de «diluir» su enseñanza (o endurecerla en unos casos), o en pocas palabras, hacer cualquier cosa que sea necesaria para mejorar sus posición y «avanzar» a su carrera. Ellos también son muy vulnerables al desilusiono. Ellos han comenzado a predicar por las razones equivocadas, y su error pone en peligro igual a sus propias almas como al bienestar de la iglesia del Señor.

A todos los que predican o están pensando en predicar, sugeriríamos: si tú, como Jeremías, hallas que la palabra de Dios está en tu corazón como un fuego ardiendo, y no puedes guardar silencio; si tú, como Pablo, sientes una compulsión motivador de predicar el evangelio; si tú, como los apóstoles, no puedes dejar de decir lo que ellos vieron y oyeron; si tú, como Timoteo, tienes una preocupación genuina por el estado de otros; si tú predicarías, apoyado o no apoyado, si en tiempo o fuera de tiempo, si admirado por la iglesia o perseguido por la iglesia; si tú predicarías, como todos los predicadores del primer siglo, incluso bajo la amenaza de muerte, entonces por supuesto predica, y recibe y está agradecido por cualquier apoyo moral y económico que las iglesias te dan. Pero, si no es así, por tu propio bien y el del evangelio, busca alguna otra forma de ganarse la vida. 

\section{Predicación Equilibrada}
El predicador fiel predica un mensaje de esperanza y gracia, mientras a la vez repudiando un espíritu de autocomplacencia y descuidado. Él presenta ante otros un Dios que es igual de bueno como severo. 

Una dama dijo una vez a un predicador, «$\rule{2cm}{0.15mm}$, la mejor cosa que podrías hacer por la gente cuando los bautizas sería dejarlos abajo por un ratito más». Su declaración manifieste una creencia de que el camino a los cielos es tan difícil que hay poca probabilidad que incluso los mas sinceros puedan llegar allí. Tal actitud es peligrosa, verdaderamente. Provocaría a un cristiano a tropezar sobre el primer obstáculo echado en su camino. «¿Para qué voy a luchar?» sería su pregunta obvia, «Probablemente no voy a llegar a los cielos de todos modos».

Un predicador fiel no quiere albergar una actitud tan pesimista y fatalista en su predicación. Él quiere predicar un evangelio de esperanza y aseguranza. Él quiere convencer a la gente de que pueden ir a los cielos si así determinan, que su «trabajo en el Señor no es en vano», que ellos pueden «hacer firme [su] llamado y elección», y pueden tomar pasos definidos para asegurar esa entrada amplia «al reino eterno de nuestro Señor y Salvador Jesucristo», que pueden «morir en el Señor» y incluso pueden tener «confianza en el día del juicio». Él quiere que sus oidores sepan que sirven un Dios quien es «compasivo y clemente… lento para la ira y grande en misericordia», que tiene «misericordia para los que le temen» «como un padre se compadece de sus hijos», que «sabe de qué estamos hechos» y «se acuerda de que somos solo polvo» (\ibibleverse{Psalms}(103:8-14)). Si él logra éxito, ofrece a sus oyentes grandes incentivas a la fidelidad. Convence a la gente de que realmente van a los cielos, y ellos sufrirán ridículo, deprivación, encarcelamiento, e incluso la muerte para llegar allí.

Al mismo tiempo un predicador fiel no quiere albergar una actitud de autocomplacencia o descuidado. La gracia de Dios no debe verse como una licencia de pecar. «Diligencia», «sinceridad», «abundar», «fe»: estas son palabras claves en las escrituras. La entrada amplia al reino eterno es para los que suplen con diligencia las «gracias cristianas» y abundan en ellas (\ibibleverse{IIPeter}(1:5-11)). El perdón de Dios es para los que andan en la luz y confiesan sus pecados (\ibibleverse{IJohn}(1:6-10)). Tal debe ser la conducta continua de un cristiano si él quiere vivir en esperanza y aseguranza continua.

Un predicador fiel busca un equilibrio sensitivo en estos asuntos. Tememos que cuestiones actuales acerca de la gracia de Dios están empujando hombres a polos opuestos, unos casi exclusivamente predicando la severidad de Dios; otros casi exclusivamente predicando Su bondad. Todo hombre debe examinarse a sí mismo y a su propia enseñanza. Que cada uno predique para llevar otros a decir, «Me haces contento de ser un cristiano; me ayudas a ver mis fallas; me motivas a hacer un mayor esfuerzo». ¡Qué contraste entre esta respuesta y la mencionada al principio del artículo! ¿Cuál respuesta probablemente resultaría de tu predicación?

\section{Erudito, Pero Práctico}
La predicación de Cristo y de Sus apóstoles era erudito, pero práctico. La erudición con ellos era un medio para un fin, nunca un fin en sí. Su meta en la predicación era cambiar a los hombres, a «presentar a todo hombre perfecto en Cristo» (\ibibleverse{Colossians}(1:28)).

Verdadera erudición entre predicadores fieles tenderá a ocultarse. Será escondida detrás de la cruz de Cristo y el amor propio del predicador por las almas de los hombres. El verdadero erudito no tiene que llamar la atención a sí mismo. El verdadero cristiano no lo hará.

Ninguna obra más erudita fue escrita que el libro de Romanos. Ese libro, sin embargo, no es ninguna mera discusión de pasajes difíciles que impresionaría al mundo con su erudición. Es práctico por todo, mientras Pablo persuade a los hombres a buscar la salvación, no por un sistema de ley, sino por fe en Jesucristo.

¿Quién cuestionaría la erudición del Señor en Su sermón del monte? La belleza de ese sermón, sin embargo, se encuentra en su sencillez, su aplicabilidad, su franqueza, su reto a las consciencias de los hombres. Estas son cualidades que lo han hecho querido a sus lectores y han producido cambios en las vidas de millones de personas.

La biblia abunda en ejemplos de predicación poderosa: de Moisés, el «Deja ir a mi pueblo»\ibible{Exodus}(5:1), de Josu\'e, el «Escoged hoy a quién habéis de servir»\ibible{Joshua}(24:15), de Natán, el «Tú eres aquel hombre»\ibible{IISamuel}(12:7), de Elías, el «¿Hasta cuando vacilaréis entre dos opiniones?»\ibible{IKings}(18:21), de Daniel, el «Has sido pesado en la balanza y hallado falto de peso»\ibible{Daniel}(6:27), de Juan, el «No te es lícito tenerla»\ibible{Matthew}(14:4), de Pedro, el «A este Jesús a quien vosotros crucificasteis, Dios le ha hecho Señor y Cristo»\ibible{Acts}(2:36). Puede ser que los que escucharon estos predicadores no pusieran la erudición de ellos por las nubes, pero entendieron lo que ellos dijeron y realmente nunca eran lo mismo después de haberlos escuchado. 

No es nuestra intención desalentar la erudición; al contrario la alentamos. Ningún hombre debería predicar quien no haya buscado con diligencia «manejar con precisión la palabra de verdad». Pero cuando uno entra en el púlpito, es el tiempo de «redarguye, reprende, exhorta» en lugar de demostrar la brillantez. Predicación erudita, pero práctica, siempre ha sido la necesidad, y sigue siendo la necesidad hoy en día.

\section{Esos Puntos Adicionales}
«Yo pensé de un punto que cabría bien con ese sermón», el buen hermano dice mientras le estrecha la mano al predicador. Por supuesto el predicador ya ha pensado en un número de maneras de mejorar el mismo sermón y puede ser que se sienta desalentado en sus esfuerzos. Pero él puede hacer bien en recordar que tales pensamientos adicionales viniendo de su audiencia no son indicativos, como regla, de pobre predicación, sino de buena predicación.

Buena predicación no está dirigido a personas quienes, como pajaritos, se sientan con sus bocas abiertas y listas para tragar cualquier cosa que se les dé de comer. Buena predicación desafía una audiencia a pensar y evaluar. Buena predicación planta semillas de pensamiento que pueden germinar y crecer. Buena predicación con tiempo familiariza a la gente con las escrituras, escrituras que probablemente recordarán cuando puntos relevantes se discuten. No debería sorprender cuando esas personas pensativas quieren compartir sus pensamientos con su maestro a la conclusión de una lección. Y cuando vienen con verdaderas gemas (¡Qué tan a menudo puede pasar!) que pueden contribuir grandemente a sermones, bien puede ser una indicación fuerte de que el programa de enseñanza de esa congregación está trabajando con rara eficacia. 

Además, buena predicación comienza con un propósito y se dirige hacía una conclusión. Material que contribuye a ese propósito se retiene; material – incluso buen material – que no contribuye se corta. Buena predicación no tiene como su meta agotar un tema (o una audiencia), sino picar los corazones de oyentes y llevarlos a conclusiones deseadas. Buena predicación a menudo realiza sus propósitos con unos pocos, bien escogidos pasajes, desarrollados plenamente, en vez de citas trepidantes de todo pasaje que se puede recordar sobre el tema. No es sorprendente, entonces, cuando la audiencia se recuerda de escrituras y pensamientos que no se incluyen en el sermón. Tal se debe desear. 

Buena predicación enfoca la atención en el Cristo en lugar del orador. Buena predicación es el tipo «\ibibleverse{Acts}(2:)» de predicación. ¿Acaso no es obvio el propósito del sermón de Pedro mientras él se dirige a la conclusión de que «a este Jesús a quien vosotros crucificasteis, Dios le ha hecho Señor y Cristo»\ibible{Acts}(2:36)? ¿Acaso no podríamos todos de nosotros pensar en otros pasajes relevantes del Antiguo Testamento además de los tres que Pedro usó? ¿Y acaso no fue posible una mayor discusión de tales temas como el «hades» y el «previo conocimiento de Dios»? No hay duda de que el sermón de Pedro incluyó más de lo que se registra en \ibibleverse{Acts}(2:) (versículo 40 dice tal)\ibible{Acts}(2:40), pero lo que el Espíritu Santo registra para nosotros es un sermón con un propósito definido que se dirige hacía una conclusión lógica apoyada por escrituras y discusión, todo de lo cual contribuyó a ese propósito. Esta es buena predicación porque es predicación inspirada por el Espíritu Santo.

No alcanzamos el ideal. Pero podemos mejorar, y el mejor indicador de mejoramiento bien puede ser los comentarios adicionales que proceden de una audiencia retada.

\section{Entendiendo Las Necesidades De Un Predicador}
¿Cuánto se debe pagar a un predicador? Él debería recibir suficiente apoyo para que \textbf{con administración razonable} él puede vivir libre de grandes preocupaciones financieras. Esto variará considerablemente, dependiendo del numero de niños un hombre pueda tener, la cantidad de enfermedad sufrida en la familia, su responsabilidad hacía parientes necesitados, etc. Pero ningún predicador puede hacer un trabajo efectivo si él está molestado constantemente por preocupaciones financieras. 

Iglesias deberían entender los gastos extras incurridos por un predicador. Él tendrá que gastar mas dinero en ropa y combustible que el miembro promedio de la iglesia promedia. Él recibirá invitados mas frecuentemente. Él debe mantener un buen carro en buena condición mecánica, porque a menudo él se llama fuera de la ciudad por funerarios y otras necesidades. Él usa su propio carro privado todos los días en la obra del Señor.

Predicadores disfrutan pocos beneficios marginales. Hay unos. Otros cristianos a menudo comparten los productos de sus jardines, regalan la ropa que les queda pequeña a sus hijos, y muestran amabilidad en muchas otras maneras. Estos gestos atentos no se deben minimizar. Ocasionalmente un hombre de negocios hará algún favor para predicadores. Series de predicación normalmente resultan en apoyo adicional para los que se piden predicar en ellas. Pero un predicador usualmente arregla su propio seguro medico y plan de retiro (si lo tiene). Estos involucran un gasto pesado. Él paga Seguridad Social como una persona autónoma. Estos hechos se deberían considerar por iglesias.

A veces escuchamos de injusticia grosera hacia predicadores. Hemos escuchado de iglesias deliberadamente privando a sus predicadores hasta que se muden. Qué manera tan cobarde de manejar un asunto. Otras iglesias pagan bien a un predicador, pero unos de los miembros ven de mala gana cada centavo que él recibe y se aseguran de que él sabe de su disgusto.

Alentamos a iglesias a apreciar buenos hombres que fielmente predican el evangelio, a ser atentos de sus necesidades financieras, y a apoyarlos con alegría. Alentamos a predicadores a estar agradecidos por la consideración de buenas iglesias. Que no dejemos que pobres relaciones entre iglesias y predicadores dificulten la realización de la gran tarea puesta ante nosotros por Dios.

\section{La Clase Más Baja De La Sociedad}
¿Qué clase de gente constituye el elemento mas bajo de la sociedad? ¿Traficantes de drogas? ¿Pornógrafos? ¿Abusadores de niños? ¿Gorrones?

Según Isaías, no es ninguna de estas: «El Señor, pues, corta de Israel la cabeza y la cola [“desde el más alto hasta el más bajo”, nosotros diríamos – BH], la hoja de palmera y el junco en un mismo día» (\ibibleverse{Isaiah}(9:14)). Después en el siguiente versículo, él identifica «la cola», la clase más baja de la gente: « El anciano y venerable es la cabeza, y el profeta que enseña la mentira, es la cola». 

No hay ningún elemento mas bajo de la sociedad que los falsos maestros. Ellos se dicen ser guías a los cielos, pero señalan el camino al infierno. Sus víctimas no son los que buscan la maldad; ellos son los que buscan la salvación. Los espiritualmente ciegos vienen a ellos por dirección, solo para ser guiados sobre un precipicio. Los espiritualmente enfermos vienen por curación, solo para que se les de veneno. Los perdidos vienen por verdad que puede salvar el alma, solo para que se les dé error que condenará el alma.

Verdaderos maestros de la palabra sufren a causa de ellos. Se acercan a los perdidos con el puro evangelio, solo para encontrarlos ya «apagados» con respecto a la religión, a causa de falsos maestros. Y cuando sí encuentran el buscador sincero, sus dos o tres horas por semana con él se contrarresta por horas y horas de exposición a las promesas falsas, testimonios dramáticos, y tácticas engañosas de los evangelistas de televisión hoy en día. El producto final no es un converso al Señor, sino una víctima desilusionada quien está preparada a condenar toda religión como hipocresía y fraude. Con razón Isaías los marca como lo más bajo de lo bajo. 

Unos son engañadores intencionales mientras otros son accidentales. Los engañadores intencionales se motivan por avaricia, ambición egoísta, y, en unos casos, sensualidad (\ibibleverse{IIPeter}(2:1-3)). Ellos hacen mercadería de sus víctimas. Ellos suplican dinero, entristecidos por sus «apuros», cuando en realidad ellos tienen enormes cantidades de dinero invertido en acciones y bonos. Ellos se dicen ser amadores de Dios cuando en realidad son amadores de sí mismos. Por afuera parecen ser «ángeles de luz» pero por dentro son «falsos apóstoles» y «obreros fraudulentos» (\ibibleverse{IICorinthians}(11:13-15)).

Otros son falsos maestros por accidente. Ellos mismos son víctimas del engaño, y ellos salen engañando a otros. Ellos lo hacen pensando que están sirviendo a Dios (\ibibleverse{John}(16:2). Son sinceros, pero también son culpables, culpables de guiar otros al infierno. Por esta misma razón Jacobo escribió: «Hermanos míos, no os hagáis maestros muchos de vosotros, sabiendo que recibiremos un juicio más severo» (\ibibleverse{James}(3:1)). La advertencia es clara: entusiasmo por compartir el evangelio con otros es admirable, pero ay de ese hombre que se apura al trabajo sin la consideración debida de las responsabilidades y el juicio que lo acompañan.

Falsos maestros no son buena gente. Son el elemento mas bajo de la sociedad. ¡Qué cosa temerosa si yo soy uno – o estoy engañado por uno – o apoyo a uno!

\section{Ojo – Corrupción Adelante}
Rumores de pecado y corrupción entre lideres religiosos se están aumentando. Leemos de predicadores que suplican constantemente por dinero, a menudo recibiendo donativos generosos que representan gran sacrificio de parte de los pobres, mientras ellos mismos disfrutan de lujurias y enormes superávits financieras. Escuchamos de adulterio y robo, mentiras y el juego, y en muchos casos los lideres religiosos son culpables. Y echa sal a la herida cuando leemos «Reverendo» ante sus nombres. ¡Que día tan terrible será cuando tales hombres se presentan ante el Señor en juicio!

Mientras estamos entristecidos por la misma mala conducta, estamos aún más entristecidos por el hecho de que tal mala conducta causa que mucha gente pierde confianza en toda religión. Ellos dan la espalda a Cristo, Su palabra, y Su iglesia. Ellos suponen que todo él que se dice ser cristiano es como los corruptos líderes religiosos de quienes ellos han oído. Ellos suponen que no hay verdadero valor en el cristianismo. Esto constituye una perdida terrible para todos los involucrados.

Hacemos las siguientes observaciones por tu consideración:
\begin{enumerate}
\item No toda religión ni todo líder religioso es corrupto. Hay siervos de Cristo en tu comunidad y por todo el mundo que son piadosos, dedicados y concienzudos. Ellos no llegan al resumen de noticias. Ellos van haciendo su trabajo calladamente y sin fanfarria. Pero están allí y tú necesitas estar consciente de su presencia y ejemplo. Búscalos. Ellos incluso pueden estar entre tus conocidos.
\item Maldad de parte de lideres en religión no comprueba que la religión en sí es falsa. La mala conducta de uno de los apóstoles de Jesús (Judas) no era prueba de que Jesús mismo fuera corrupto o de que Su religión fuera en vano. De igual modo, mala conducta de parte de unos de Sus supuestos seguidores no es prueba de que Su religión es corrupta y vana.
\item Si permites que la mala conducta de algún líder en la religión te desvíe del Señor y que estés perdido eternamente, tú eres un perdedor junto con él. Señalándolo a él como el culpable no hará que tu sufrimiento eterno sea más fácil de aguantar.
\item Tu fe debe estar en Jesucristo, no en algún hombre. Tu fidelidad debe ser a Cristo. Un hombre, incluso un buen hombre, te puede decepcionar, pero Jesucristo jamás te decepcionará. Él nunca te abandonará ni te desamparará siempre y cuando tú mantienes fidelidad y lealtad a Él.
\item Puedes ir a los cielos, hagan lo que quieran los demás. Mientras servicio fiel al Señor nos lleva a una comunión con otros en Su iglesia, fidelidad en sí es un asunto individuo. Te vas a presentar solo ante el Señor en juicio. Puedes ir a los cielos incluso si todos los demás de tu generación abandonan a Cristo. Piénsalo. La cuestión en que tienes que pensar no es donde va a pasar la eternidad algún líder religioso, sino donde \textbf{tú} pasarás la eternidad.
\end{enumerate}
Que el Señor nos ayude a todos a subir por encima del mundo y sus influencias para servirle en sinceridad y verdad.

\section{Si Él Tiene Un Alma}
Un posible converso maravilloso se describe en el décimo capítulo de Hechos: «Piadoso y temeroso de Dios con toda su casa, que daba muchas limosnas al pueblo judío y oraba a Dios continuamente». Si este hombre solo podría aprender la verdad la obedecería. 

Sin embargo, había tremenda presión empujando Pedro a no predicarle. Él era un gentil, y los judíos odiaban los gentiles. ¿Pudo Pedro predicar a uno de esta clase de gente odiada? ¿Qué dirían sus hermanos? ¿Dañaría su influencia entre los judíos si él predicara a los gentiles? Quizás Pedro no debería ir. Quizás Cornelio debería aprender el evangelio, de alguna forma, por sí mismo.

Pedro pudo haber seguido esta misma línea de razonamiento si el Señor lo hubiera permitido. Pero a través de una visión el Señor le enseñó a Pedro que él «a ningún hombre [debe] llamar impuro o inmundo» (Hechos 10.28). Como Pedro aprendió esta lección y predicó valerosamente a Cornelio y a su casa, la gran obra entre los gentiles, que ha resultado en la salvación de miles a través de los siglos, se comenzó.

¿Acaso no hay clases de gente odiada en nuestra sociedad? ¿Será posible que en nuestra búsqueda de posibles conversos nos hayamos limitado a los de la misma posición económica, racial, y educacional que nosotros ocupamos?  ¿Y será posible que en limitarnos así, podamos estar pasando por alto unos de las mejores posibilidades que tenemos? ¿Necesitamos nosotros ser enseñados que no debemos llamar a ningún hombre impuro o inmundo?

Una cita de Henry Ficklin, un hermano anciano en el este de Kentucky, ahora difunto, me impresionó recientemente. El hermano Ficklin y otro hermano estaban visitando a personas en sus casas. El hermano Ficklin estaba algo inseguro con respecto a la ubicación de una casa a la cual iban, entonces el otro hermano, viendo a un hombre en un lote de granero, se orilló y preguntó, «¿Supones que este es el hombre que buscamos?» «Supongo que sí», hermano Ficklin replicó, «si él tiene un alma, él es el hombre que buscamos».

\textbf{Si él tiene un alma\ldots} el escritor de Hebreos dijo que Jesús probó «la muerte para todos» (\ibibleverse{Hebrews}(2:9)). Que no estemos culpables a causa de nuestros prejuicios de privar del evangelio a los «Cornelios» de nuestra sociedad.

\section{Ellos Deben Ver La Necesidad}
Un conocido mío, oyendo recientemente de una familia que había sufrido considerable desgracia, fue a la casa para investigar. Él encontró un padre sufriendo de la esclerosis múltiple. La madre se había involucrado en un serio accidente automovilístico. Un hijo, lastimado en el accidente, estuvo en el hospital. Conmovido por las desgracias de la familia, él escribió un cheque pequeño y lo dio a la madre. Lagrimas surgieron en los ojos de ella mientras le agradecía a él y expresaba cuan bondadoso todos habían sido. Ella estuvo tan agradecida. 

El mismo hombre fue esa semana a ver una pareja joven que no eran cristianos. Él ofreció estudiar las escrituras con ellos, sugiriendo en específico el libro de Hechos. Él intentaba dar a esta pareja algo de mucho mas valor que un cheque pequeño. Él quería llevarlos a la salvación de sus almas, a un hogar eterno en los cielos, a un escape de los fuegos del infierno. La pareja joven le agradeció, pero dijeron que no estaban interesados en estudiar en aquel entonces.

Antes de criticar la pareja joven, necesitamos analizar el \textbf{por qué} de la diferencia; en otras palabras, la primera reconoció su necesidad de lo que se les ofreció, mientras la segunda no. Esa pareja joven probablemente nunca se dieron cuenta lo que en realidad rehusaban. Si en algún momento se despiertan a su necesidad, ellos aceptarán el evangelio con un gozo y una emoción que superará la de la primera familia mencionada.

Contactamos a diferentes tipos de personas en nuestro trabajo por el Señor. Unos saben que están perdidos y están buscando ayuda. Otros no saben que están perdidos, pero están interesados en la biblia, y consentirán a estudiar las escrituras. Ellos se pueden llevar a entender su condición perdida a través de estudio. Pero hay otros que no están preocupados, no tienen interés en estudio bíblico, y realmente no quieren ser molestados. Estos requieren un enfoque completamente diferente, una gran cantidad de paciencia. , y mucha oración. No debemos abandonar estos. Con tiempo muchos se pueden llevar a ver su condición perdida y se arrepientan.

Los 3,000 quienes se bautizaron en el Pentecostés no se preocupaban por aprender mas de Jesús cuando el día comenzó. Lo mismo se podría decir del carcelero de Filipo. El único interés de Saulo mientras él salía para Damasco fue encarcelar y perseguir cristianos. Él difícilmente hubiera consentido a un estudio en ese tiempo. La mujer de Samaria tenía poca preocupación por su alma mientras ella iba al pozo de Jacob por agua; pero ¡oh la habilidad con que el Maestro experto la llevó a ver su condición pecaminosa! Cada uno de estos mostró un cambio total de actitud \textbf{tan pronto como vio su condición perdida}. 

Este cambio total de actitud de parte de tantas personas despreocupadas, incluso beligerantes, del primer siglo nos da esperanza para los despreocupados de nuestro siglo. De hecho, quizás algún día, con la ayuda del Señor, podamos reportar la obediencia de la pareja joven mencionada arriba.

\chapter{Preguntas Difíciles}

\section{Las Consecuencias De La Verdad}
Las consecuencias de la verdad son amargas a veces. Muchos hombres han perdido sus trabajos o hogares, o amigos o vida a causa de su defensa de la verdad. Muchos predicadores se han expulsado del pulpito, teniendo ni casa ni salario, porque ellos predicaron la verdad. Muchas personas han tenido sus nombres calumniados y difamados por la verdad. Con todas de tales personas, el amor de la verdad es mayor que el amor de la comodidad, la seguridad o incluso la vida misma. 

Verdaderamente, desgraciado es el hombre que mira adelante para evaluar las consecuencias de una posición antes de evaluar la posición misma. Tal hombre raramente llegará a un conocimiento de la verdad. Sus pensamientos acerca de «¿Qué pensará mi esposa?» o «¿Dónde predicaré?» o «¿No estaría yo condenando a mi buena madre al infierno?» o «¿Cómo explicaré mi cambio al buen hermano Jones?» o «¿Cómo sostendré a mi familia?» o «Todos pensarán que soy un loco» bien pueden cegar su mente a cualquier evidencia que haya a mano. El hombre que realmente demuestra un amor por la verdad es el hombre que estudia todo tema objetivamente y entonces deja que las consecuencias – sean buenas o malas – se cuiden a sí mismas.

Desgraciado también es el hombre que se queja y se entristece sobre las consecuencias de la verdad, porque la verdad debe llevar gozo al corazón, cualesquiera que sean sus consecuencias. Autocompasión puede guiar uno a «vender la verdad» y a profanar este artículo precioso. Si hay que sentir compasión, debería sentirse por esa persona que nunca ha sufrido las consecuencias de la verdad, porque tal hombre obviamente ha amado el elogio de los hombres más que el elogio de Dios.

Ningunos hombres sintieron más las consecuencias de la verdad que los apóstoles, pero ellos enfrentaron todas de tales consecuencias «regocijándose de que hubieran sido tenidos por dignos de padecer afrenta por Su nombre» (\ibibleverse{Acts}(5:41)). ¡Dignos! ¡Allí está la clave! El hombre que deja que un miedo de las consecuencias dicte su posición en toda cuestión nunca sufre, porque él no es digno de sufrir. ¡Tenle compasión! Pero la persona que defiende la verdad sin importar las consecuencias sufrirá, porque él es digno de sufrir. ¡Regocíjate con él! 

¡Cuán grande la diferencia entre el hombre que se orienta hacía los cielos y él que se orienta hacía este mundo!

\section{Pensamiento Superficial – Conclusiones Falsas}
Pensamiento superficial a menudo lleva a conclusiones falsas. Este hecho fácilmente se ilustra por el rechazo de los judíos de Jesús como el Cristo. A veces nos maravillamos a su incredulidad, pero ¿cómo pudieron haberlo aceptado? Después de todo, Jesús había venido de Galilea, y las escrituras claramente habían declarado que el Cristo vendría de Belén (\ibibleverse{John}(7:41-42)). Y si Él realmente era el Cristo, ¿para qué alentaría Él a la gente a quebrar el día de reposo (\ibibleverse{Matthew}(12:1-13); \ibibleverse{John}(5:1-16)), e incluso no guardarlo Él (\ibibleverse{John}(9:16))? Era verdad que Jesús realizó grandes milagros, y Su alimentación de los 5,000 era especialmente curiosa, pero Moisés había alimentado a los israelitas con maná, y él nunca proclamó haber venido de Dios (\ibibleverse{John}(6:30-31)). Y si Él realmente era el Cristo, ¿por qué permitía que los pecadores lo tocaran? ¿Acaso no hubiera sabido Él de qué clase de gente ellos eran (\ibibleverse{Luke}(7:39))? Además, Su muerte fue una prueba obvia de que Él no era el Cristo, porque la ley había dicho que el Cristo viviría para siempre (\ibibleverse{John}(12:34)). Ahora, si Él hubiera bajado de repente de la cruz, ellos hubieran creído en Él (\ibibleverse{Matthew}(27:42)), pero allí estaba Él, «azotado… herido por Dios y afligido» (\ibibleverse{Isaiah}(53:4)). ¡Él no pudo haber sido el Cristo!

Por supuesto podemos ver el error en su pensamiento. Si ellos hubieran tomado el tiempo - y no hubiera llevado tanto tiempo – todas de sus objeciones fácilmente pudieran haberse quitado, y ellos pudieran haber creído en Jesús como el Cristo para la salvación de sus almas. Pero ellos prefirieron aferrarse a su razonamiento superficial. Ellos rehusaron investigar mas profundamente en el asunto.

Vemos el error de ellos repetido una y otra vez en nuestra generación, cuando personas dejan que objeciones superficiales les prohíban de ver tales cosas como la necesidad del bautismo («¿Qué del ladrón en la cruz?»), o el error de música instrumental («David adoró con instrumentos»), o lo malo de institucionalismo («La biblia no dice “cómo”»), o el error de bautismo infantil («¿Acaso no se bautizaron familias enteras?»). Y nos preguntamos si posiblemente puede ser que unas de nuestras propias convicciones sean basadas en algún razonamiento superficial que hayamos escuchado y aceptado sin cuestionarlo. ¿Será posible que no hayamos cavado más profundamente en algún tema bíblico por miedo (quizás inconscientemente) de que un estudio mas comprensivo pueda guiarnos a conclusiones impopulares?

Las verdaderas razones detrás del rechazo de los judíos de Jesús: (1) No se preocuparon por honrar a Dios, sino buscaron el honor entre ellos mismos (\ibibleverse{John}(5:39-47)), y (2) ellos tenían ojos cegados y corazones endurecidos (\ibibleverse{John}(12:40)). ¡Esto da miedo! ¿Se pueden decir tales cosas de cualquier de nosotros? El punto es este: o nos podemos satisfacer en la basis de pensamiento superficial o podemos buscar honrar y complacer a Dios por abrir nuestros ojos y corazones, cavando mas profundamente en Su palabra, y encontrando un fundamento seguro sobre lo cual edificar nuestras convicciones. Es nuestra elección, pero si somos sabios, cavaremos profundamente, y sentaremos nuestras bases sobre la roca solida de la verdad. 

\section{La Posición Dura}
Pocos hombres son mas conservadores por naturaleza que este escritor. Su inclinación es aceptar la posición dura en casi toda cuestión, y él entiende el pensamiento de otros que tienden a hacer lo mismo. Pero una cosa que todos debemos aprender es que la posición dura no es siempre la posición correcta. 

La posición dura entre los primeros cristianos era oponerse a comer carne (\ibibleverse{Romans}(14:2)) y buenos argumentos se pudieron ofrecer por esa posición. Por ejemplo, ¿cómo pudo saber uno si la carne que él comía no había sido sacrificado a ídolos? Y ¿cómo se había matado el animal? ¿Era posible que el animal se hubiera estrangulado? Además, ciertas carnes se habían prohibido por la ley, y aún era difícil para unos judíos comer esas carnes con la consciencia limpia. Lo mas «seguro» era simplemente no comer carne en absoluto. Esa era la posición dura, pero la posición dura no era la posición correcta. 

Oh, si alguien quería mantener esta posición dura como su propio estándar privado de conducta, eso estaba bien. Él podía abstenerse de comer carne si él consideraba esa la opción segura para él. Y nadie podía despreciarlo por seguir ese procedimiento (\ibibleverse{Romans}(14:3)). De hecho, si él mantenía dudas serias acerca de comer carne, se advirtió no comer (\ibibleverse{Romans}(14:23). Pero él no tenía ningún derecho de mandar a otros a seguir esta posición dura y a abstenerse de carne. Hacer tal era apostarse «de la fe, prestando atención a espíritus engañadores y a doctrinas de demonios» (\ibibleverse{ITimothy}(4:1-5)). La posición dura simplemente no era la posición correcta. 

La verdad en cualquier cuestión no se puede determinar por seguir la posición dura, la posición suave, ni la posición moderada; solo se puede determinar por un «así dice el Señor». Debemos predicar cualquier cosa que la biblia enseña, hablando «conforme a las palabras de Dios» (\ibibleverse{IPeter}(4:11)). Si esto nos lleva a seguir lo que el mundo o incluso otros cristianos consideran ser la posición dura en cualquier cuestión, que así sea. El predicador fiel «predicará la palabra», sabiendo que « vendrá tiempo cuando no soportarán la sana doctrina, sino que teniendo comezón de oídos, acumularán para sí maestros conforme a sus propios deseos» (\ibibleverse{IITimothy}(4:3)).

Al otro lado, sabemos de unas «posiciones duras» que hermanos han tomado que simplemente no se pueden sostener por las escrituras. Podríamos reconocerlas como «seguras» y entonces aceptables para su propio estándar de conducta. Podríamos conceder el derecho de explicar a otros por qué esta posición se está adoptando. De hecho, este escritor en muchas instancias ha adoptado esos mismos estándares para su propio estándar de conducta. Pero que nunca seamos culpables de mandar lo que Dios no ha enseñado, así atando en la tierra lo que Dios nunca ha atado en los cielos. La posición dura no es siempre la posición correcta.

\section{¿Te Apartarías?}
«¿Te apartarías de un hermano que difiere contigo en esta cuestión?» Esta es una pregunta que parece estar surgiendo con creciente frecuencia cuando hermanos no están de acuerdo.

Es una pregunta que se debe considerar seriamente por cristianos sinceros, porque hay falsos maestros que se deben observar y evitar. «Y os ruego, hermanos, que vigiléis a los que causan disensiones y tropiezos contra las enseñanzas que vosotros aprendisteis, y que os apartéis de ellos» (\ibibleverse{Romans}(16:17)). «Todo el que se desvía y no permanece en la enseñanza de Cristo, no tiene a Dios; el que permanece en la enseñanza tiene tanto al Padre como al Hijo. Si alguno viene a vosotros y no trae esta enseñanza, no lo recibáis en casa, ni lo saludéis, pues el que lo saluda participa en sus malas obras» (\ibibleverse{IIJohn}(1:9-11)). Llega un punto cuando ya no podemos extender la mano derecha de comunión, porque estamos convencidos de que hay hombres entre nosotros que están guiando la iglesia a apostasía. Lo que nosotros edificaríamos, ellos intentarían derribar. ¿Cómo podemos extender nuestro aliento y aprobación a tales hombres?

A veces la cuestión de la comunión se usa, sin embargo, para intentar desacreditar un hombre antes de que él se dé una audiencia imparcial. Un hombre presenta sus convicciones acerca de la celebración de la Navidad, o la vela de la mujer, o la participación de un cristiano en la guerra carnal, o el derecho de una mujer de hacer preguntas en una clase bíblica mixta, o acerca de cualquier numero de otras cuestiones que pertenecen a conducta individua y personal, y la primera pregunta con que replican puede ser, «¿Te apartarías de un hermano que difiere contigo en esta cuestión?» El hacer tal pregunta no contesta a ni un solo argumento, pero seguramente incomoda al maestro. Si él contesta «sí» a la pregunta, él se marca como uno que divide la iglesia sobre sus opiniones. Si él contesta «no», se cuestiona la sinceridad de su creencia en lo que él ha presentado. El pobre hombre se atrapa en cualquier caso. Se desacredita y se avergüenza, \textbf{¡¡¡pero su posición, sea verdadera o falsa, no se contesta!!!} A veces es más fácil desacreditar a un hombre y prejuiciar una audiencia en su contra que contestar a sus argumentos.

No podemos tener comunión con los que guían la iglesia a apostasía. Tampoco podemos tener comunión con los que causan divisiones y ofensas contrarias a la doctrina de Cristo. ¿Pero debemos quebrar comunión los unos con los otros sobre toda diferencia que puede surgir, incluso las que pertenecen a conducta individua y personal? Esperamos que no, porque tal acción partiría la iglesia en cien astillas. Seguramente hay espacio para paciencia en tales cuestiones. Y seguramente podemos desarrollar un sentido de lo justo en nuestras discusiones de tales diferencias. Cuestiones prejudiciales y tácticas injustas no tienen lugar en discusiones entre cristianos sinceros.

\section{Natación Mixta}
En la cuestión de natación mixta hay tres posiciones posibles:
\begin{enumerate}
\item \textbf{Es correcto.} Si esta posición es correcta, natación mixta es correcta, si uno sea un anciano, diacono, predicador, o lo que sea, si uno está en Florida o Alabama, si uno está entre amigos en la casa o entre desconocidos en alguna ciudad distante. 
\item \textbf{Es incorrecto.} Si esta posición es correcta, natación mixta es incorrecta, quienquiera que sea, y dondequiera que sea, si en casa o de vacaciones. Es incorrecto igual para el adolescente como para el adulto. 
\item \textbf{Es correcto, pero el principio de \ibibleverse{Romans}(14:) se aplica.} Según este principio, uno debería renunciar la práctica si por ella su hermano tropieza o se ofende o se hace débil (\ibiblechvs{Romans}(14:21)). Si esta tercera posición es correcta, la cuestión es en realidad un asunto de indiferencia, una que no involucra mal inherente, una en que todo hombre se debe ser «plenamente convencido según su propio sentir» (\ibiblechvs{Romans}(14:5)). Si esta posición es correcta, los oponentes de la natación mixta en realidad son hermanos «débiles», que se deben aceptar, «pero no para juzgar sus opiniones» (\ibiblechvs{Romans}(14:1)). Los que están dispuestos a renunciar derechos legítimos por la causa de su influencia entre hermanos se deben elogiar, pero ¿en realidad es esta la posición correcta? ¿Son hermanos débiles los que se oponen a la natación mixta, oponiéndose a lo que en realidad es sano y bueno?
\end{enumerate}
El asunto se centra en la cuestión del vestimento. ¿Se puede justificar un traje de baño en la luz de \ibibleverse{ITimothy}(2:9)? «Asimismo, que las mujeres se vistan con ropa decorosa, con pudor y modestia…». Las siguientes definiciones deberían ser útiles:

\textbf{Decorosa:} «ordenada, bien arreglada, decente» (W.E. Vine).

\textbf{Pudor:} «un sentido de vergüenza, modestia» (W.E. Vine).\footnote{El artículo original incluyó la siguiente discusión de la palabra inglesa \textit{shamefacedness} que se traduce al español como pudor: «\textit{Shamefacedness} es esa modestia que es “\textit{fast}” o arraigado en el carácter» (Davies, \textit{Bible English}, p. 12). La palabra «\textit{fast}» significa «firmemente fijado». La persona «\textit{bedfast}» está firmemente fijada en su cama [«\textit{bed}» en ingles]. La persona \textit{shamefast} está firmemente fijado en un sentido de vergüenza o modestia.} Pudor es lo contrario de la audacia o descaro.

\textbf{Modestia:} «denota sanidad de mente» (W.E. Vine). «Es ese autogobierno habitual del interior, con su freno constante sobre todas las pasiones y deseos, que prohibiría la tentación a estos de surgir\ldots» (Trench).

Combinando estas definiciones, concluimos que una mujer debe vestirse modestamente, decentemente, con un sentido de vergüenza o modestia que se arraiga, firmemente fijado, en su carácter, no con audacia, de tal forma que cualquier persona que la mira tendría la impresión de que aquí hay una mujer que mantiene un freno constante sobre todos sus pasiones y deseos. Estas definiciones nos fuerzan a aceptar posición dos. Y mientras no deseamos ser feos o groseros, sí pedimos con candor a los que aceptan cualquiera de las otras dos posiciones, «Si un traje de baño sería aceptable en la luz de este pasaje, por favor dinos ¿qué vestimento sería inaceptable?».

\section{La Morada Del Espíritu}

Pocas personas cuestionarían el hecho de que el Espíritu Santo de alguna forma mora dentro del cristiano. Pablo escribió a los santos en Corinto: « ¿O ignoráis que vuestro cuerpo es templo del Espíritu Santo, el cual está en vosotros…?» (\ibibleverse{ICorinthians}(6:19)). Él escribió además, «Y si alguno no tiene el Espíritu de Cristo, no es de él» (\ibibleverse{Romans}(8:9)). Hay considerable desacuerdo, sin embargo, con respecto a \textbf{cómo} el Espíritu mora dentro de un cristiano. No es nuestro propósito en este articulo corto tratar esa cuestión, pero sí queremos sugerir tres hechos que se deben recordar mientras uno estudia la cuestión.

\begin{enumerate}
\item La era de los milagros ya pasó. Las únicas personas en la era del evangelio que realizaron milagros eran los que o recibieron el bautismo del Espíritu Santo (\ibibleverse{Acts}(2:1-4); \ibiblechvs{Acts}(10:44-46)) o recibieron dones milagrosos por la imposición de las manos de los apóstoles (\ibibleverse{Acts}(8:5-23); \ibiblechvs{Acts}(19:1-7)).  Nadie recibe ninguno de estos hoy en día. El propósito de los milagros era revelar y confirmar la verdad (\ibibleverse{ICorinthians}(2:7-13); \ibibleverse{Mark}(16:19-20)). Como toda la verdad se ha revelado (\ibibleverse{John}(16:13)), no hay más necesidad de milagros. La conclusión de uno, entonces, con respecto a la morada del Espíritu debe ser compatible con este hecho.
\item El cristiano se guía por el Espíritu a través de las escrituras, la palabra de Dios (\ibibleverse{Psalms}(119:105); \ibibleverse{IITimothy}(3:16-17); \ibibleverse{Ephesians}(3:3-4)). Él no tiene alguna voz interior, separada de las escrituras, que de alguna forma lo guíe a conclusiones infalibles relacionadas con la verdad y lo justo. Tampoco hay algo en las escrituras que sugiera que la providencia de Dios trabaja de alguna forma a través de la morada del Espíritu. Por consiguiente, uno comete un grave error si él interpreta sus sentimientos o pensamiento subjetivo como algún tipo de mensaje provisto por el Espíritu que mora en él.
\item Declaraciones acerca de la morada del Espíritu no se colocaron en las escrituras como problemas con que luchar. Se colocaron allí para la aseguranza y el consuelo de uno. Un cristiano sostiene una comunión muy estrecha con la deidad – tan estrecha que se puede decir que él mora en la deidad y la deidad mora en él. En persecución, pruebas, tentaciones, y muerte su reconocimiento de esta estrecha relación lo sostiene y le ayuda a ser triunfante en Cristo. Los apóstoles nunca sintieron la necesidad de explicar como funciona esta morada. El pentecostalismo y otras ideas equivocadas relacionadas con el Espíritu Santo fuerzan el cristiano de esta generación a preocuparse con este problema. Sin embargo, si declaraciones acerca de la morada del Espíritu se convierten principalmente en un problema con que él tiene que luchar, si su obsesión con el «cómo» de la morada del Espíritu lo ciega al «hecho» de esa morada, él comete un grave error y puede ser que no encuentre el gozo y la consolación que debería ser ganado por la promesa del Señor.
\end{enumerate}

Diferencias seguirán de existir, pero guardando constantemente en la memoria estos tres hechos debería proteger a cualquiera de nosotros de conclusiones peligrosas con respecto a esta cuestión.

\chapter{FAMILIA}
\section{La Bendición De Un Compromiso Absoluto}
Hay dos relaciones en la vida en las cuales Dios demanda el compromiso absoluto: la relación de uno con Cristo como un cristiano y la relación de uno con su compañero en el matrimonio. Uno puede renunciar su ciudadanía por otra, o su trabajo, o su residencia, o su afiliación congregacional. Pero los compromisos de uno a Cristo y a su compañero son compromisos de por vida. Deserción de cualquier de los dos resulta en la desaprobación de Dios.

Cuando uno se convierte en cristiano, él promete su lealtad a Cristo como su Señor y Rey. Persecución puede venir, o desánimo, o tentación, o problemas de la iglesia, pero él promete ser fiel – fiel mientras viva. De igual manera, cuando uno se casa, él promete a su compañera su amor y fidelidad «hasta que la muerte los separe». Problemas pueden surgir, o enfermedad, o dificultades financieras, o presión de miembros de la familia, o malentendidos, pero él promete ser fiel. Él no se irá. El divorcio será impensable. Él está comprometido a su compañera – él es de ella y ella es de él – y el compromiso es absoluto. 

La voluntad de Dios con respecto a la permanencia del matrimonio se ha revelado claramente. En el matrimonio, un hombre \textbf{deja} a su padre y madre y \textbf{se une} a su esposa (\ibibleverse{Genesis}(2:24)). Ellos son \textbf{ligados} (\ibibleverse{Romans}(7:2-3)). Ellos son \textbf{unidos} por Dios y no se deben \textbf{separar} (\ibibleverse{Matthew}(19:6)). Ellos se hacen \textbf{una sola carne} (\ibibleverse{Matthew}(19:6)). Uno no puede pensar en términos más fuertes para describir la permanencia de una relación. Con razón Malaquías dijo que Dios detesta el divorcio (\ibibleverse{Malachi}(2:16)).

La mayor felicidad que un hombre puede experimentar en esta tierra se encuentra en estas dos esferas de compromiso absoluto. La felicidad no se encuentra en un medio-compromiso. Un hombre que constantemente está considerando otros trabajos, nunca siendo comprometido a su trabajo presente, es un hombre inestable, dividido entre dos opiniones. Así es con ese hombre que intenta servir el Señor con un semi-compromiso. Él está aburrido e indiferente, teniendo la mínima de religión necesaria para hacerle miserable. Él intenta aferrarse al mundo con una mano y al Señor con la otra y encuentra gozo en ninguno. Uno solo necesita mirar en los apóstoles y primeros discípulos, al otro lado, para ver que el compromiso absoluto, el tipo de compromiso que puede aceptar persecución y rendir sacrificio, es un elemento básico del gozo en el Señor (\ibibleverse{Acts}(2:41-47); \ibiblechvs{Acts}(5:40-42); \ibiblechvs{Acts}(16:25)).

Así es con el matrimonio. Dios sabía que la felicidad en el matrimonio solo podría encontrarse en compromiso absoluto. Entonces Él decretó por medio del apóstol Pablo, «que cada uno tenga su propia mujer, y cada una tenga su propio marido» - un hombre por una mujer hasta que la muerte los separe (\ibiblechvs{ICorinthians}(7:2)). Su decreto puede ser problemático para los que han ignorado Su enseñanza en el pasado (las consecuencias del pecado siempre son terribles – \ibibleverse{Galatians}(6:7-8), pero es para el bien de la humanidad, procediendo de Dios quien manda para nuestro bien (\ibibleverse{Deuteronomy}(10:13)).
\begin{enumerate}[labelindent=0pt]
\item\textbf{Compromiso absoluto provee confianza en el matrimonio.} El marido no tiene que preocuparse por la fidelidad de su esposa hacía él, ni tiene que preocuparse la esposa por la fidelidad del marido, porque el compromiso del uno al otro es abierto y obvio. A causa de la naturaleza pública de su compromiso, la tentación a la infidelidad es casi inexistente. Los medio-comprometidos, al otro lado, serán tentados frecuentemente, porque la tentación es inherente en el compromiso parcial.
\item \textbf{Compromiso absoluto provee seguridad en el matrimonio.} Seguridad es un resultado de confianza y permanencia. Es cuando uno duda la fuerza y permanencia de su relación con otro que él se siente inseguro.
\item \textbf{Compromiso absoluto provee una vida estable en el matrimonio.} Se han ido los días agitados, inestables y inseguros del noviazgo, reemplazados por una relación duradera y segura con la pareja de uno. Naomi hermosamente describió esta vida estable cuando ella dijo a Rut, \textbf{«Hija mía, ¿no he de buscar seguridad para ti, para que te vaya bien?»} (\ibibleverse{Ruth}(3:1)).
\item \textbf{Compromiso absoluto provee un fundamento solido sobre el cual edificar en el matrimonio.} Sin este fundamento, ningún hogar de calidad se puede edificar. 
\end{enumerate}
Gus Nichols escribió una vez que él y su esposa habían estado presentes por clases bíblicas el domingo anterior y por ambos periodos de adoración. Él dijo además que ellos no habían tomado su decisión ese domingo, sino que habían tomado esa decisión cuarenta años antes cuando primero se convirtieron en cristianos, y que, en estar presentes por todos los servicios, ellos simplemente estaban cumpliendo el compromiso que habían hecho cuarenta años antes. De igual manera, yo acosté mi cabeza en la almohada junto a mi esposa anoche y me desperté esta mañana a su lado. Si lo quiera Dios, eso haré de nuevo esta noche y seguiré haciéndolo mientras ambos vivamos. No es que hasta ahora estemos tomando esa decisión, porque tomamos esa decisión hace veinticinco años y simplemente estamos cumpliendo ese compromiso de por la vida que hicimos hace tantos años en el pasado.

«Sea el matrimonio honroso en todos, y el lecho matrimonial sin mancilla, porque a los inmorales y a los adúlteros los juzgará Dios» (\ibibleverse{Hebrews}(13:4)).

\section{Edificando Hogares Felices}
¿Será cierto que los cristianos de la generación actual son mucho mas vacilantes en casarse que generaciones pasadas? Ocasionalmente oímos tal observación expresada, y tendemos a estar de acuerdo con ella. Conocemos a muchas personas jóvenes en sus últimos años 20 y sus primeros años 30 que son «muy elegibles» pero parece que ni están cerca a casarse.

Las razones por este desarrollo pueden ser muchas, pero no podemos evitar preguntarnos si no será que unos simplemente tienen miedo del matrimonio. Ellos han escuchado tantos sermones sobre el divorcio y sus causas. Ellos han observado tantos matrimonios fracasar, incluso entre líderes de la iglesia en quienes ellos han confiado mucho. Ellos han visto parejas vivir en miseria total mientras se quedan juntos. No es de extrañar que ellos se han vuelto desilusionados y tienen dudas de que haya tal cosa como un hogar feliz. Un hombre joven observante dijo a este escritor una vez, «¿Por qué no predica alguien un sermón sobre “Felicidad en el Matrimonio”?» Muchos pueden estar en necesidad de que alguien los tranquilice. 
Ahora, no estamos haciendo campaña para que todos se casen – eso es un asunto muy personal – y especialmente no alentaríamos a nadie a apresurarse a casarse. Pero cristianos no deberían tener una perspectiva fatalista del matrimonio. Dios ordenó el matrimonio para nuestro bien y felicidad. « El que halla esposa halla algo bueno y alcanza el favor del Señor» (\ibibleverse{Proverbs}(18:22)). Dios dijo en el principio, «Por tanto el hombre dejará a su padre y a su madre y se unirá a su mujer, y serán una sola carne» (\ibibleverse{Genesis}(2:24)). 
La bendición del matrimonio se puede ver en el hecho de que es una relación dadivosa. Uno da materialmente, emocionalmente, sexualmente – en pocas palabras, uno da a sí mismo. En el matrimonio «yo» se convierte en «nosotros»; «mío» se convierte en «nuestro». Los dos se juntan por Dios y se convierten en uno.

Dos \textbf{cristianos} con la ayuda de Dios pueden edificar una relación hermosa, una relación que ni enfermedad, ni tiempos difíciles, ni separaciones forzadas pueden cortar. La base de esa relación se sienta sobre sumisión completa a la enseñanza de Dios combinada con un amor mutuo y un respeto del uno por el otro. Cada experiencia compartida en su turno – sea de gozo o de tristeza – se convierte en nuevo material que contribuye a la estructura total, así que con cada año que pasa la relación se encuentra más estrechamente ligada que nunca. Esto es el matrimonio como Dios quiere que sea. Dentro de tal matrimonio, cristianos pueden encontrar gozo, felicidad, seguridad y ayuda vital para la eternidad. Ellos viven juntos, ellos comparten juntos, ellos oran juntos, ellos se ríen juntos, ellos lloran juntos, ellos se envejecen juntos, y finalmente ellos son «coherederos de la gracia de la vida» (\ibibleverse{IPeter}(3:7)).

Entonces, mis amigos jóvenes, sean cuidadosos, pero no tengan miedo. No esperen lo imposible, pero sí esperen el gozo de amar y de ser amados. No esperen un matrimonio sin problemas, pero sí esperen solucionar sus problemas a través de respeto mutuo por Dios y Su palabra. Hay felicidad en matrimonio para dos que verdaderamente son hijos de Dios.

\section{Seis Pasos A Un Hogar Roto}
Un hogar raramente se destruye «de la noche a la mañana».  Su destrucción usualmente es el resultado de ciertos pasos fatales tomados sobre un largo periodo de tiempo. En estos días, cuando tantos hogares se están desmoronando, nos sería bueno examinar nuestras propias relaciones matrimoniales para ver si hemos comenzado a viajar por el camino a una ruptura inevitable. 
\begin{enumerate}
\item \textbf{Egoísmo.} Esto puede ser el enemigo número uno de una vida feliz en el hogar. Cada persona está haciendo su propia cosa. Ninguno está dispuesto a sacrificar lo que él quiere hacer para disfrutar juntos actividades sanas. Cada uno está buscando su propia satisfacción en materialismo, actividades sexuales, o en tiempo pasado con parientes con poca preocupación por la satisfacción de la pareja en estos asuntos. El largo camino se comienza. 
\item \textbf{Intolerancia.} Fallas en la pareja de uno comienzan a aparecer que de alguna forma se quedaron ocultas durante el periodo del noviazgo. O, si las fallas eran evidentes, se vuelven mucho mas irritantes en una relación día-a-día, viviendo juntos. Los regaños comienzan. Cada cual decide que ha cometido un terrible error en su matrimonio.
\item \textbf{Resignación.} Ambos participantes se vuelven resignados a su situación. «Hemos hecho nuestra cama; simplemente tendremos que acostarnos en ella», ellos piensan. No más esfuerzo se hace para edificar un hogar feliz. La comunicación virtualmente termina. El amor comienza a desvanecerse, y en muchos casos cede paso a la amargura. 
\item \textbf{Fin de relaciones sexuales.} La barrera de comunicación pronto afecta la relación sexual, y la pareja encuentra que ya no están disfrutando y cumpliendo este propósito dado por Dios en el matrimonio. Ellos han dejado que su matrimonio se deteriore en una mera relación de mantener una casa. Tales personas pueden ser muy vulnerables al paso siguiente. 
\item \textbf{Adulterio.} Tentaciones pueden surgir tan inesperadamente, y muchas personas cuyas necesidades físicas no se están cumpliendo en la casa pueden ceder a la tentación. Racionalización viene fácilmente en tales casos: la persona siente que nunca ha recibido trato justo en la casa; él merece esta nueva atención; esto es verdadero amor (?); él está seguro de que alguien lo entiende por la primera vez. ¡Cuán engañoso es el pecado! Tiempo considerable ha pasado desde que nuestra pareja tomó esos primeros pasos hacia un hogar roto, pero ahora su viaje se ha terminado. Solo falta un paso más.
\item \textbf{Separación.} La cosa que obviamente ha destruido este hogar es el pecado, pero no solo el pecado de adulterio. Egoísmo, intolerancia, falta de amor, amargura, y no satisfaciendo necesidades físicas (cuando sea posible), todos constituyen pecado. Hay solo una conclusión a que podemos llegar. El pecado es la causa de hogares rotos. Puede ser pecado de parte de uno o de ambos participantes, pero un hogar está roto a causa del pecado.
\end{enumerate}

\section{Una Familia Ricamente Bendecida}
«Compadezca a la familia Smith. Pobrecitos, ellos tienen que hospedar el predicador cada vez que la iglesia planea una serie de predicación. Y cristianos siempre están pasando por su casa. Y ellos son casi las únicas personas en la iglesia que invitan a la gente después de los servicios. Simplemente no lo aguantaría yo…».

¡Olvida tu compasión! Los Smith son una familia ricamente bendecida. Oh, ocasionalmente ellos reciben algún canalla ingrato, pero las bendiciones de la hospitalidad pesan mucho más que los problemas.

La biblia habla de unas personas maravillosamente bendecidas en este sentido. Por ejemplo, no compadecemos a María y Marta por «tener» que recibir a Jesús en su casa; compadecemos a los que, no queriendo a Jesús, se privaron de esta bendición. No compadecemos a María la madre de Juan Marcos, en cuya casa «muchos estaban reunidos y oraban»\ibible{Acts}(12:12); compadecemos a esos hogares donde cristianos nunca se congregan para estudio bíblico y oración. No compadecemos a Filemón quien debía preparar alojamiento para Pablo; nos encantaría tener a Pablo como invitado en nuestra casa. Cristianos necesitan aprender el gozo y las bendiciones que vienen a los que son hospitalarios.

«Somos hospitalarios», alguien dice, «a menudo tenemos amigos de la iglesia en nuestro hogar para reuniones y fiestas». ¡Eso está bien! ¡Alentamos esto! Pero hospitalidad bíblica va mas allá de tener \textbf{amigos} en la casa por una tarde que por lo menos parcialmente es para nuestra propia placer \textbf{egoísta}. «Fui forastero, y me recibisteis», Jesús dirá en el juicio (\ibibleverse{Matthew}(25:35)). Gayo se elogió por ayudar a «hermanos» y  «extraños» que viajaban por amor al nombre del Señor, y se dijo por Juan, «Harás bien en ayudarles a proseguir su viaje de una manera digna de Dios» (\ibibleverse{IIIJohn}(1:5-7)). Además, la biblia enseña que nos convertiremos en participantes en las \textbf{malas} obras de \textbf{falsos} maestros cuando los recibimos en nuestras casas (\ibibleverse{IIJohn}(1:9-11)). Pero ¿no implica esto que nos convertimos en participantes de las \textbf{buenas} obras de maestros \textbf{fieles} cuando les mostramos hospitalidad?

Un cristiano querrá hacer amigos con otros cristianos y por consiguiente planeará tardes placenteras con amigos estrechos en su casa. Pero el cristiano hospitalario también usará su hogar para conducir clases bíblicas, recibiendo predicadores visitantes y otros obreros del Señor, familiarizándose con recién llegados en la iglesia, consolando a los afligidos y abrumados, y para toda buena obra.

Nuestros hogares son bendiciones del Señor. No debemos usarlos egoístamente, sin para la gloria Suya. El resultado será un rasguño ocasional en una silla, o una mancha en la alfombra, o un vaso desportillado – realmente un pequeño precio que pagar, sin embargo, por la calidez que viene al hogar por medio de nuevos amigos, buenas influencias, ricas discusiones bíblicas, participación en la obra del Señor, y la satisfacción de saber que uno está agradando a Dios y preparándose para la eternidad. No, no son los Smith a quienes compadecemos; son los quienes no conocen el gozo de la hospitalidad.

\section{Escogiendo Un Compañero}
Un hogar feliz comienza con una elección sabia de un compañero de matrimonio. Entonces presentamos las siguientes preguntas como un guía para nuestros jóvenes mientras ellos buscan su compañero de por vida.
\begin{enumerate}
\item \textbf{¿Esta persona es elegible para el matrimonio?} Hay los que tienen el derecho de casarse según las leyes de nuestro estado, pero no tienen el derecho de casarse según la ley de Dios. La autoridad de Dios es suprema, y el cristiano debe guardar Su ley cuando hay un conflicto entre Su ley y la ley del gobierno.

La ley de Dios es que solo los que están divorciados por la causa de fornicación tienen el derecho de volver a casarse. «Y yo os digo que cualquiera que se divorcie de su mujer, salvo por infidelidad, y se case con otra, comete adulterio» (\ibibleverse{Matthew}(19:9)). Si la pareja matrimonial de uno es culpable de infidelidad matrimonial, ese puede repudiar a esa pareja por esa causa, y casarse con otro. Si el divorcio es por cualquier otra causa, ese es inelegible para el matrimonio. 
\item \textbf{¿Esta persona es cristiana?} Muchos problemas pueden surgir cuando un cristiano está casado con alguien que no lo es, especialmente cuando es la esposa que es la cristiana. Antes de que cualquier dama joven se case con alguien que no sea cristiano, ella debería considerar los siguientes problemas que a menudo enfrentan el cristiano en un matrimonio desigual.
	\begin{enumerate}
	\item \textbf{Problemas en asistir a los servicios.} Ella debería preguntarse, «¿Qué haré si mi esposo algún día del Señor sale manejando el carro, dejándome sin transportación a los servicios?» «¿Qué haré si mi esposo anuncia que su compañía nos está mudando a 					alguna ciudad donde ninguna iglesia del Señor se congrega? Muchas mujeres han enfrentado estos problemas.
	\item \textbf{Problemas con la ofrenda.} La dama joven ama el Señor y Su obra, y quiere dar liberalmente en su apoyo. Su esposo, sin embargo, no comparte sus convicciones. Él siente que un dólar es suficiente para dar.
	\item \textbf{Problemas en criar a los hijos.} Unos han logrado criar a sus hijos para ser cristianos sin la ayuda de sus compañeros. Eunice logró hacerlo con Timoteo. Pero la influencia de un padre es grande, y muchas son las mujeres que no han podido vencer esta 					influencia para ver a sus hijos convertirse en cristianos.
	\item \textbf{Problemas en vencer la influencia del compañero de uno.} Casándose con la esperanza de reformar el compañero de uno es una cosa peligrosa. A menudo el compañero se levanta a un estándar mas alto, pero raramente al estándar de un cristiano. En 					lugar de eso, mientras los estándares del que no es cristiano se levantan, los estándares del cristiano se bajan, así que los dos se unen en algún punto en el medio. La dama joven debería reconocer que el hombre con quien ella se casa será la influencia 					mas grande de una 	naturaleza terrenal en su vida, y ella debería preguntarse, «¿Este hombre me va a ayudar a ir a los cielos?».
	\item \textbf{Problemas emocionales.} Recientemente una mujer piadosa, justo habiendo recibido noticias de la muerte de su esposo en un accidente automovilístico, clamó, «¿Por qué no pudo haber sido yo o uno de los niños, porque mi esposo no estaba preparado 					para morir»
			¿Está pensando la lectora que ella convertirá a su esposo después de las bodas? Puede ser que ella lo haga, pero las estadísticas muestran que sus probabilidades no son muy buenas. El riesgo es demasiado grande. El único camino seguro para cristianos 				es casarse con cristianos.
	\end{enumerate}
\item \textbf{¿Tiene esta persona fuerza de carácter?} Muchos chicos y chicas parecen querer hacer lo bueno pero simplemente son influenciados demasiado fácilmente por sus asociados o por las circunstancias a su alrededor. Tal debilidad de carácter produce un buen compañero de matrimonio. Si uno quiere un hogar feliz, él debería escoger un compañero que es confiable, en cuya palabra se puede confiar, y quien tiene la fuerza para hacer lo correcto, incluso cuando todos a su alrededor están haciendo lo incorrecto.

Después de todo, si ese hombre joven o mujer joven miente a sus padres o a su empleador, él algún día te mentirá a ti. Si él es excesivamente celoso y posesivo durante el noviazgo, él manifestará las mismas debilidades después de las bodas. Si él es enojado – o coqueto – o presumido – o gastón – o «codo» - ahora, también lo será después de las bodas. Uno no puede casarse con un debilucho moral y esperar un hogar feliz.
\end{enumerate}
Sí, habrá fallas y excentricidades que se deben aceptar y tolerar en cualquier relación feliz entre dos personas. Pero hay ciertas condiciones que son prácticamente intolerables, y esperamos que este articulo ayude a algún joven a evitar tal.

\section{Cualidades Que Hacen Una Buena Esposa}
¿Qué cualidades debería un hombre joven buscar en la chica con quien él se casaría? ¿Qué cualidades debería la chica buscar en el hombre joven? ¿Qué cualidades deberían demostrarse en las vidas de los que ya estamos casados? El libro de Rut provee unas respuestas a estas preguntas, hablándonos del matrimonio de dos personas maravillosas, Rut y Booz. Las siguientes cualidades se pueden ver en Rut, haciéndola una esposa ideal para Booz). 
\begin{enumerate}
\item \textbf{Ella era leal.} Las palabras, «adonde tú vayas, iré yo, y donde tú mores, moraré. Tu pueblo será mi pueblo, y tu Dios mi Dios\ldots\ibible{Ruth}(1:16)» originalmente no fueron el sentimiento expresado por una novia a su esposo, sino fueron las palabra de una nuera leal (Rut) a su suegra (Noemí). Pero le lealtad expresada por Rut en esas hermosas palabras a su suegra era la clase de lealtad que después la haría una maravillosa esposa para Booz.
\item \textbf{Ella era trabajadora.} Habiendo llegado a una tierra desconocida con su suegra, Rut reconoció que alguna provisión tenía que hacerse por el sostenimiento físico de ellas. Entonces ella salió espigando; en otras palabras, ella recogía cualquier trigo que se dejaba caer por los segadores mientras caminaban. «Por casualidad», el campo en que ella espigó era el de Booz\ibible{Ruth}(2:).
\item \textbf{Ella no era egoísta.} Ello no espigaba solo para sí misma; ella también proveía para su suegra. 
\item \textbf{Ella era temerosa de Dios.} Rut se había criado en la tierra de Moab y se había enseñado a servir los dioses de los moabitas. A través de su primer esposo y sus suegros, sin embargo, ella había aprendido del único verdadero Dios y había llegado a servirle. La muerte de su esposo no había afectado su lealtad a Dios, porque verdadera lealtad a Dios va mas allá de la muerte, la vida, los padres, o cualquier cosa. Este temor de Dios de parte de Rut era particularmente atractivo a Booz, quien, cuando la conoció por la primera vez, la elogió: «Que el Señor recompense tu obra y que tu remuneración sea completa de parte del Señor, Dios de Israel, bajo cuyas alas has venido a refugiarte» (\ibibleverse{Ruth}(2:12)). 
\item \textbf{Ella era agradecida.} Esta hermosa cualidad, tan rara en nuestra propia sociedad, indudablemente brilló en la vida de Rut, mientras ella, respondiendo a la bondad de Booz, «bajó su rostro, se postró en tierra y le dijo: ¿Por qué he hallado gracia ante tus ojos para que te fijes en mí, siendo yo extranjera?» (\ibibleverse{Ruth}(2:10)).
\item \textbf{Ella se preocupaba mas por calidad que por atractividad física en un esposo.} Las indicaciones son que Booz era considerablemente mayor que Rut. Pero Booz era un pariente cercano al primer esposo de Rut, lo cual, junto con sus cualidades morales y espirituales, lo hizo mucho mas adecuado que alguien cuya edad fuera mas cerca a la suya. El reconocimiento de Rut de esto y su deseo de tener a Booz como su esposo provocó otra reacción de elogio de los labios de Booz: «Bendita seas del Señor, hija mía. Has hecho tu última bondad mejor que la primera, al no ir en pos de los jóvenes, ya sean pobres o ricos» (\ibibleverse{Ruth}(3:10)).
\item \textbf{Ella era moralmente pura.} Se podía decir de Rut de una forma general: «pues todo mi pueblo en la ciudad sabe que eres una mujer virtuosa» (\ibibleverse{Ruth}(3:11)). Pero circunstancias, registradas en el mismo libro de Rut, proveyeron una ocasión específica para demostrar cuan moralmente puros ellos realmente eran. La ocasión de que hablamos resultó cuando Rut, equivocadamente pensando que Booz fuera el pariente más cercano a su esposo, y por consiguiente creyendo que ella tuviera todo el derecho de convertirse en su esposa (lee \ibibleverse{Deuteronomy}(25:5-10)), fue en la oscuridad de la noche a Booz con la plena intención de convertirse en su esposa esa noche. Solo había un problema: él no era el pariente mas cercano y ellos entonces no tenían el derecho de convertirse en esposo y esposa hasta que los requisitos legales podían cumplirse. Aunque estaban solos en la noche, llenos de un amor mutuo, y teniendo la plena intención de convertirse en esposo y esposa lo antes posible, ellos mantuvieron su pureza moral, dejando un buen ejemplo para todos los que seguirían después. «Sea el matrimonio honroso en todos, y el lecho matrimonial sin mancilla, porque a los inmorales y a los adúlteros los juzgará Dios» (\ibibleverse{Hebrews}(13:4)).
\end{enumerate}
Estamos bastante seguros de que si hubiera mas «Ruts» en este mundo habría mucho menos divorcios. ¿Qué piensan, chicas? ¿Están intentando ser una «Rut»? Y, chicos, ¿les sería atractiva una «Rut»? Nada se dice del aspecto físico de Rut, pero ella era una mujer hermosa, porque ella tenía un carácter hermoso.

La biblia enseña que tal mujer es de gran valor, porque: «Mujer hacendosa, ¿quién la hallará? Su valor supera en mucho al de las joyas» (\ibibleverse{Proverbs}(31:10)). El hombre joven que encuentra tal esposa «halla algo bueno y alcanza el favor del Señor» (\ibibleverse{Proverbs}(18:22)).

\section{Cualidades Que Hacen Un Buen Esposo}
Como Rut provee un buen ejemplo de las cualidades que hacen una buena esposa (ve el artículo anterior), Booz demuestra las cualidades que una chica debería buscar en el hombre joven que será su esposo.

\begin{enumerate}
\item \textbf{Él era temeroso de Dios.} Especialmente impresionante es el hecho de que su temor de Dios influenció cada relación en la vida. Él saludaba a los que trabajaban por él con las palabras, «El Señor sea con vosotros», y ellos replicaron, «Que el Señor te bendiga» (\ibibleverse{Ruth}(2:4)). Su temor de Dios obviamente influenció su relación con Rut. Tales cualidades espirituales son esenciales si uno será un buen esposo y padre. 
\item \textbf{Él era un buen proveedor.} Él tenía terreno, y personalmente aceptó la responsabilidad de supervisar la cosecha y la trilla de sus cultivos.
\item \textbf{Él era bondadoso hacia los necesitados.} No solo dejó que Rut espigara en sus campos (la ley mandó que él hiciera esto – Deuteronomio 24.19), sino que la alentó a hacerlo, y proveyó comida, agua, y protección de peligro para ella a través del tiempo que ella trabajó allá. Y él incluso fue mas allá de esto y dijo a sus labradores que se dejaran caer un poco del grano a propósito para que Rut pudiera recoger lo que sería suficiente para ella y su suegra (\ibibleverse{Ruth}(2:15-16)).
\item \textbf{Él reconoció calidad en alguien que era pobre.} Cuando vemos a Rut, solo vemos esas hermosas cualidades que caracterizaron su vida. Pero Booz podía ver más que eso. Él podía ver su ropa pobre y circunstancias bajas; él podía ver el trabajo de baja categoría que ella hacía. Booz pudo haber sido cegado a la hermosura interior de Rut por la obvia pobreza por la cual ella se rodeó. Pero Booz era un hombre perceptivo que podía juzgar carácter, no en la basis de riqueza material, sino en la basis de pureza y piedad interior.
A causa de esto, él encontró y reconoció una verdadera joya que después se convirtió en su esposa.
\item \textbf{Él cumplía la ley.} Sabiendo que él no era el pariente mas cercano al esposo anterior de Rut, y sabiendo por consiguiente que él no tenía ningún derecho legal o divino a ella, él rehusó tomarla como su esposa hasta que requisitos legales y divinamente ordenados podían cumplirse.

La gente de nuestra edad necesita aprender que hay los que son inelegibles para el matrimonio, y que elegibilidad legal no establece necesariamente la elegibilidad en los ojos de Dios. Solo tres clases de personas tienen un derecho de casarse: (a) los que nunca se han casado (b) los cuyos compañeros están muertos, y (c) los que han repudiado a sus compañeros por la causa de fornicación (\ibibleverse{Matthew}(19:9)). El casamiento de cualquier otro resulta en la formación de una unión adultera que no tiene aprobación divina.

\item \textbf{Él era moralmente puro.} De hecho él tomó la iniciativa en asegurarse de que él y Rut mantuvieran pureza moral (ve el artículo anterior).
\end{enumerate}
No hay nada malo en que un hombre joven sea guapo o fuerte físicamente. Pero tales cualidades tendrán poca influencia sobre la clase de esposo que él será. Fuerza espiritual, una disposición para el trabajo, amabilidad, respeto por la autoridad, fuerza de carácter, honestidad, respeto a sí mismo, etc., son las virtudes que hacen un buen esposo de un hombre. Booz tenía estas virtudes. Hay hombres jóvenes hoy en día que las tienen, y ellos son los hombres jóvenes que deberían ser atractivos a chicas cristianas.

El matrimonio puede resultar en gran felicidad o gran miseria. Para los que escogen cuidadosamente a sus parejas y trabajan para edificar sus hogares sobre el fundamento seguro de la palabra de Dios, el matrimonio trae felicidad. Pero para los que ven el matrimonio a la ligera y escogen sus parejas solo en la basis de atracción física, el matrimonio a menudo resulta en miseria y desilusión. Deseamos lo mejor para todos de nuestros lectores jóvenes, y confiamos en que con la ayuda de Dios ellos pueden encontrar verdadera felicidad, igual en esta vida como en la venidera. 

\section{Esposos Comprensivos}
«Y vosotros, maridos, igualmente, convivid de manera comprensiva con vuestras mujeres…» (\ibibleverse{IPeter}(3:7)). 
Un buen esposo se preocupa por el bienestar de su esposa y es comprensivo con respeto a todas sus necesidades. 
\begin{enumerate}
\item \textbf{Sus necesidades materiales.} Él entiende que a su esposa le gustaría un atuendo nuevo ocasionalmente así como le gusta a él, que ella disfrutaría tener algo de dinero para gastar que ella puede llamar «suyo propio» y que ella disfrutaría un descanso de estar en la cocina. Entendiendo estas cosas, él pone sus necesidades materiales y las de sus hijos antes de sus propias. 
\item \textbf{Sus necesidades de salud.} Él no se queja cuando su esposa necesita ver un doctor o cuando su mala salud requiere extenso tratamiento médico. Él provee la mejor atención que él puede pagar y lo hace con gozo. 
\item \textbf{Sus necesidades sociales.} Una buena mujer es hospitalaria. Ella reconoce la necesidad de buena compañía por toda su familia. Ella quiere que su casa sea un lugar donde otros cristianos se sientan cómodos y bienvenidos. Un esposo comprensivo reconoce estas necesidades también y a menudo renuncia sus propios intereses personales por las necesidades sociales de su familia. Además, él planea viajes para visitar con los padres de su esposa y con otros de su familia. De hecho, el pueblo de ella se convierte en el pueblo de él.
\item \textbf{Sus necesidades espirituales.} Un buen hombre provee liderazgo espiritual por su familia. Él no solo asiste a periodos de adoración y clases bíblicas con su familia, él convierte la casa misma en un centro espiritual. Él ora con su familia. Él lee y platica de las escrituras con ellos. En varios momentos – mientras andan en el carro, o caminan por el bosque, o se sientan en la mesa del comedor – él exhorta y alienta a su esposa y a sus hijos a ser piadosos en su conducta y espirituales en su perspectiva. Él se preocupa por su propio ejemplo ante su familia y no vacila en decir «lo siento» cuando él está consciente de haber hecho algo malo.
\item \textbf{Sus necesidades físicas.} Un esposo y esposa deben cumplir «el deber conyugal» el uno al otro (\ibibleverse{ICorinthians}(7:2-5)), satisfaciendo las necesidades sexuales el uno del otro. Ninguno de los dos debería ser demasiado exigente; ninguno de los dos debería «privar» al otro. El hombre que se vuelve «casado» a su trabajo y desatiende a su esposa comete un grave error.
\item \textbf{Sus necesidades emocionales.} Un esposo comprensivo aprecia su esposa y expresa igual por palabra como por hecho su aprecio y amor. Se dice de la «mujer virtuosa» de Proverbios 31: «Sus hijos se levantan y la llaman bienaventurada, también su marido, y la alaba diciendo: Muchas mujeres han obrado con nobleza, pero tú las superas a todas» (vv. 28-29)\ibible{Proverbs}(31:28-29).
\end{enumerate}
Dios ha dado al hombre la posición de liderazgo en el hogar y la responsabilidad de mostrarse digno de esa posición. Muchos están fallando – ¡fallando miserablemente! Cuan aplicable se hace la exhortación de David: «¡Sé hombre!»\ibible{IKings}(2:2).

\section{Lo Que Dios Ha Unido}
Solo tres clases de gente son eligibles para el matrimonio según la enseñanza de la biblia: (1) Los que nunca se han casado, (2) los cuyas parejas están muertos, y (3) los que han repudiado a sus parejas a causa de fornicación (infidelidad marital).

La enseñanza de Jesús con respecto al divorcio y al volver a casarse es clara: «Y yo os digo que cualquiera que se divorcie de su mujer, salvo por infidelidad, y se case con otra, comete adulterio» (\ibibleverse{Matthew}(19:9)). 

Cuando dos personas eligibles están casadas, la biblia los describe como «ligados» (\ibibleverse{Romans}(7:2)), «una sola carne» (\ibibleverse{Matthew}(19:6)), «unidos» (\ibibleverse{Matthew}(19:5)), y tales no se deben separar, porque: «Por consiguiente, ya no son dos, sino una sola carne. Por tanto, lo que Dios ha unido, ningún hombre lo separe» (\ibibleverse{Matthew}(19:6)). 

Cuando uno divorcia a su pareja por cualquier causa sino la fornicación y se casa con otro, entra en una unión adúltera con ese con quien se casa.

«No [le] es lícito tenerla» (\ibibleverse{Matthew}(14:4)). Ni el paso del tiempo, ni el fortalecimiento de afección, ni el nacimiento de hijos pueden hacer que sea licito. Ellos están en adulterio en los ojos de Dios.

El perdón por esta relación pecaminosa se puede recibir de la misma manera que para cualquier otro pecado, a través de arrepentimiento y obediencia al evangelio. ¡Pero allí está la parte difícil! El arrepentimiento demanda una cesación del pecado de lo cual uno se arrepienta. Si uno se arrepienta de mentir, él debe dejar su mentir. Si uno se arrepienta de robar, él debe dejar su robar. De igual manera, cuando uno se arrepienta de su adulterio, él debe dejar su adulterar. Así que, cuando uno que ha estado viviendo en una unión adúltera obedece el evangelio, él debe terminar esa relación adúltera. Esta enseñanza parece difícil en la sociedad de hoy, pero es enseñanza bíblica sobre este tema.

Obviamente, el mejor curso de acción es evitar este predicamento tan emocionalmente cargado. «Una onza de prevención vale una libra de cura». Los que son eligibles para el matrimonio deberían salir con otros que son eligibles, evitando compañerismos estrechos con los que son inelegibles. «Pero solo somos amigos», alguien puede estar pensando mientras sale con alguien que no es eligible para el matrimonio. «Yo no pensaría en casarme con esta persona». Tal pensamiento es peligroso. Conocemos un numero de personas cuyas vidas ahora están arruinadas porque ellos no vieron ningún peligro en ser «amigos» con los con quienes no podían casarse con la aprobación de Dios. El único curso seguro es que los que son eligibles para el matrimonio mantengan compañía con otros que son eligibles para el matrimonio. Hacer otra cosa es jugar con fuego.

Los que están casados deben reconocer la permanencia de su relación, y trabajar diligentemente para hacer que sus hogares sean felices y exitosos. Quitando de sus mentes la posibilidad del divorcio, ellos deben solucionar sus problemas con amor, paciencia, y la ayuda de Dios.

Pero sí urgimos a los que actualmente están en relaciones adúlteras a tener la fe y valor para cortar sus vínculos adúlteros y servir el Señor. El Señor bendecirá a los que lo hacen, y «los cielos valdrán lo todo». 

\section{Dios Odia El Divorcio}
«Me estoy divorciando, pero no planeo casarme de nuevo». Estas palabras se están escuchando con creciente frecuencia. Usualmente él que las habla está pensando que Dios permite el derecho del divorcio, pero desaprobaría el casarse de nuevo. La verdad es, sin embargo: El divorcio mismo es pecaminoso a menos que sea por la causa de fornicación. 

\textbf{Considera \ibibleverse{Matthew}(19:3-6)}. La pregunta originalmente presentado a Jesús por los fariseos no tuvo que ver con el volver a casarse, sino con el divorcio: «¿Es lícito a un hombre divorciarse de su mujer por cualquier motivo?». La respuesta de Jesús a esa pregunta: «Por tanto, lo que Dios ha unido, ningún hombre lo separe». Solo fue después de más interrogatorio que Jesús habló del problema de volver a casarse y el adulterio.

\textbf{Considera \ibibleverse{Malachi}(2:16)}. «Porque yo detesto el divorcio —dice el Señor, Dios de Israel». Incluso bajo el viejo pacto Dios no aprobaba el divorcio indiscriminado. Es probable que las «lagrimas» de versículo 13 que cubrieron «el altar» y provocaron el Señor a rehusar su ofrenda eran las lagrimas de los que injustamente se habían repudiado.

\textbf{Considera \ibibleverse{Matthew}(5:32)}. «Todo el que se divorcia de su mujer, a no ser por causa de infidelidad, la hace cometer adulterio». Observa las palabras «la hace». A este escritor le parece que este versículo está diciendo que si uno divorcia a su esposa por cualquier causa excepto la fornicación, él la pone en una posición de tentación de cometer adulterio y comparte la culpa cuando ella sí comete adulterio. Al otro lado, si él la repudia por la causa de fornicación, él la ha repudiado legítimamente y no comparte ninguna culpa en cualquier adulterio que ella comete posteriormente. 

\textbf{Considera \ibibleverse{ICorinthians}(7:10)}. «A los casados instruyo, no yo, sino el Señor: que la mujer no debe dejar al marido». Observa la palabra «instruyo», o «mando» (RVR1960). El siguiente versículo (v. 11) no niega ni anula este mandamiento, sino simplemente reconoce que uno puede desobedecer el mandamiento del Señor (en tal caso peca – 1 Juan 3.4), y declara sus opciones si él ha desobedecido.

No solo manda el Señor que el esposo y esposa vivan juntos, sino también los manda que cumplan las necesidades físicas el uno del otro (\ibibleverse{ICorinthians}(7:3-5)) y que se amen el uno al otro (\ibibleverse{Ephesians}(5:25); \ibibleverse{Titus}(2:4-5)). Si un compañero en un matrimonio falla en estas áreas, el otro siempre debe ser obediente a Dios, buscando ser lo que Dios quiere que él (o ella) sea en la relación matrimonial. La idea de divorcio o separación nunca debería entrar en la mente a menos que ocurre la fornicación.

No estamos sugiriendo que el divorcio mismo es «adulterio», pero sí estamos diciendo que el divorcio por cualquier causa excepto la fornicación es pecado. Cristianos no se deben influenciar por los estándares flojos que prevalecen en el mundo en que viven. 

\section{Desafiando Adúlteros}
Tom es un gran trabajador personal, pero ha topado con un problema. La pareja con la cual él ha estado estudiando parecía estar al punto de responder cuando Tom aprendió que ellos se involucraban en un matrimonio adúltero. ¡Pobre Tom! Él está tan decepcionado. «Ya no tiene sentido seguir trabajando con esta pareja», él piensa, «sabes que ellos no se separarían para romper su relación adúltera».

¿¿¿Ellos no harían qué??? Personas que quieren servir el Señor e ir a los cielos están dispuestos a renunciar \textbf{cualquier cosa} para hacerlo. Los primeros cristianos nunca vacilaron en desafiar a las personas perdidas a renunciar \textbf{cualquier cosa} que se interpusiera entre ellos y los cielos. Los conversos corintios habían renunciado su fornicación, idolatría, borracheras, prácticas homosexuales, y otras actividades pecaminosas (\ibibleverse{ICorinthians}(6:9-11)). Por supuesto que lo habían hecho. Ellos querían ir a los cielos. Pablo no vaciló en decir a los que estaban casados con incrédulos, «Sin embargo, si el que no es creyente se separa, \textbf{que se separe}» (\ibibleverse{ICorinthians}(7:15)). Pablo aparentemente pensó que había personas que estarían dispuestas a vivir vidas célibes por el resto de sus días si eso fue lo que se necesitaba para ir a los cielos, y él no vaciló en desafiarlos a hacerlo.

Pero ¿las personas realmente renunciarán \textbf{cualquier cosa} para ir a los cielos? ¿Realmente hubiera renunciado Abraham a su único hijo? ¡Por supuesto que sí! ¿Realmente hubiera renunciado Daniel a su alta posición en el gobierno de Nabucodonosor? ¿O Moisés al lujo del palacio de faraón? ¿O Saul de Tarso a su grandeza potencial en la religión de los judíos? ¿O los apóstoles a sus vidas? Sí, lo harían, porque ellos querían ir a los cielos.

¿Qué capacitó a estos grandes caracteres bíblicos a sacrificar tanto por la causa del Señor? ¡La fe! «Por la fe Abraham…». «Por la fe Moisés…»\ibible{Hebrews}(11:). «Yo sé en quién he creído, y estoy convencido…»\ibible{IITimothy}(1:12). Estos hombres por medio de la fe pudieron ver más allá de las preocupaciones y el sufrimiento de este mundo para ver «una patria mejor, es decir, celestial». Ellos creían que consideraciones celestiales deberían trascender \textbf{toda} consideración terrenal, y creyendo, ellos pusieron sus manos en el arado y nunca miraron atrás. 

¿Qué debería hacer Tom en el caso de la pareja adúltera? Una cosa es segura, él no debería desesperarse. Al contrario, él debería llevarlos a amar el Señor y a querer ir a los cielos, y si él logra hacerlo, ellos romper su relación adúltera. \textbf{¿No lo harías tú?} ¿No estarías tú dispuesto a renunciar a tu familia – si fuera necesario – o cualquier otra cosa, para ir a los cielos? ¿O será posible que nuestro juicio de lo que otros estarían dispuestos a sacrificar se contamina por nuestra propia falta de compromiso y sacrificio? ¿Será posible que nosotros mismos no tenemos suficiente fe y compromiso para guiar otros a la salvación de sus almas?

«Si alguno viene a mí, y no aborrece a su padre y madre, a su mujer e hijos, a sus hermanos y hermanas, y aun hasta su propia vida, no puede ser mi discípulo» (\ibibleverse{Luke}(14:26)).

\section{El Padre Cristiano}
Poca dificultad se experimenta en pensar en mujeres bíblicas que eran extraordinarias en el hogar: Ana, María, Elisabet, Jocabed, Eunice. Intentos de pensar en una lista comparable de hombres, sin embargo, no son fáciles, porque muchos grandes hombres de la biblia eran fracasos en sus hogares: David, Lot, Eli, Samuel, Jacobo, etc. Un vistazo a las causas de su fracaso puede ayudar a nuestros lectores varones a evitar sus errores. 
\begin{enumerate}
\item\textbf{Inmoralidad.} El adulterio de David con Betsabé y el homicidio posterior de su esposo, Urías, resultó en el ruino del hogar de David\ibible{IISamuel}(11:). La borrachera de Noe contribuyó a problemas dentro de su familia, así estropeando el éxito con que él había criado a sus hijos en una sociedad de maldad sin paralelo\ibible{Genesis}(9:20-27). Hombres hoy en día, si viven en borracheras, adulterio, y otras formas de inmoralidad, no pueden esperar ser éxitos en sus hogares.
\item\textbf{Falta de disciplina.} Destrucción vino sobre Elí y su familia porque «sus hijos trajeron sobre sí una maldición, y él no los reprendió» (\ibibleverse{ISamuel}(3:13)). La biblia dice: «Corrige a tu hijo mientras hay esperanza, pero no desee tu alma causarle la muerte» (\ibibleverse{Proverbs}(19:18)).
\item\textbf{Avaricia.} «Alzó Lot los ojos y vio todo el valle del Jordán, el cual estaba bien regado por todas partes… como el huerto del Señor» (\ibibleverse{Genesis}(13:10)), y, aparentemente motivado por un deseo por abundancia material, mudó su familia a Sodoma. Las consecuencias de este triste error son bien conocidas a nuestros lectores. Muchos cristianos hoy en día están cometiendo el mismo error cuando sacrifican sus hijos en el altar de la avaricia, siendo dispuestos a mudar sus hijos a dondequiera por una promoción o más dinero.
\item\textbf{Parcialidad.} Este error comúnmente cometido llevó problemas al hogar de Isaac, quien era parcial hacia Esaú mientras Rebeca era parcial hacia Jacobo, y al hogar de Jacobo, quien era parcial a Josef. No solo sufrió la familia generalmente en todos estos casos, sino el favorecido sufrió especialmente. Compadece al hijo favorito en cualquier familia. Él sufre mas que cualquier otro en la familia.
\end{enumerate}
¿Qué pueden hacer los hombres para evitar el fracaso en el hogar? Ellos pueden reconocer su posición de liderazgo en el hogar. «Porque el marido es cabeza de la mujer, así como Cristo es cabeza de la iglesia, siendo Él mismo el Salvador del cuerpo» (\ibibleverse{Ephesians}(5:23)). Reconociendo esto, ellos pueden desarrollar una mayor dignidad de este lugar de liderazgo a través de mayor fuerza de carácter y convicción. Es triste ver a una mujer piadosa intentando estar en sujeción a un esposo débil, indeciso y vacilante. Los hombres harían bien con escuchar el consejo de David a Salomón, «Sé hombre» (\ibibleverse{IKings}(2:2)).

Ellos pueden ser mas comprensivos con sus esposas e hijos. Los hombres harían bien con desviar su atención de la televisión, programas de deportes, y periódicos, y pasar tiempo con sus familias. 

Ellos pueden guiar la familia en oración, lectura bíblica, y devoción. Ellos harán bien con considerar las instrucciones dadas a los hijos de Israel: «Y estas palabras que yo te mando hoy, estarán sobre tu corazón; y diligentemente las enseñarás a tus hijos, y hablarás de ellas cuando te sientes en tu casa y cuando andes por el camino, cuando te acuestes y cuando te levantes» (\ibibleverse{Deuteronomy}(6:6-7)).

Muchos niños nunca han escuchado a sus padres orar; nunca han escuchado a ellos ni dar gracias por la comida. Compadece a esos niños.

Los padres pueden orar sin cesar por la ayuda del Señor. La tarea de criar niños en la disciplina e instrucción del Señor es una de las mayores responsabilidades que los hombres deben enfrentar en la vida. Esa responsabilidad debe cumplirse con oración.

Finalmente, ellos pueden desarrollar la humildad. El siguiente poema, «Dos Oraciones», escrito por un autor desconocido, sugiere la humildad que todo padre necesita.
\begin{flushleft}\begin{verse}
Anoche mi niñito\\
Me confesó\\
Algún error infantil;\\
Y arrodillado a mi rodilla\\
Oró con lágrimas;\\
«Querido Dios, hazme un hombre\\
Como Papa – sabio y fuerte;\\
Yo sé que lo puedes».\\
Entonces mientras él dormía\\
Me arrodillé junto a su cama,\\
Confesé mis pecados,\\
Y oré con cabeza inclinada;\\
«O Dios, hazme un niño,\\
Como mi niño aquí – \\
Puro, sin engaño,\\
Confiando en Ti con fe sincera».\\
\end{verse}\end{flushleft}

\section{La Madre Cristiana}
Baxter Walker escribió una vez el siguiente anuncio:

«Hay una posición abierta para un individuo completo, agradable, paciente, limpio y arreglado, bien vestido, sano (este requisito no exigido a veces) y atractivo».

«Horas largas, sueldo bajo, condiciones de trabajo pasables, pocos beneficios marginales, vacaciones ocasionales, no días feriados garantizados. Aplicante debe tener habilidad de gestión, conocimiento dietario, habilidad de conducir, conocimiento de buenos modales, conocimiento de la matemática nueva es útil pero no requerido, habilidad de cocer, conocimiento mecánico, y debe mostrar compromiso a la posición. Ningún examen necesario».

La posición descrita es la de la maternidad, ese rol difícil que Dios ha dado a la mujer.

La descripción bíblica de la mujer virtuosa en \ibibleverse{Proverbs}(31:10-31) ciertamente indica que la maternidad no es un rol fácil. Observa los siguiente de este pasaje:
\begin{enumerate}
\item\textbf{Su carácter.} Ella es una «mujer virtuosa»\ibible{Proverbs}(31:10) [RVR1960]. «En ella confía el corazón de su marido»\ibible{Proverbs}(31:11).
\item\textbf{Su valor.} «Su valor supera en mucho al de las joyas»\ibible{Proverbs}(31:10).
\item\textbf{Su disposición para el trabajo.} «Busca lana y lino, y con agrado trabaja con sus manos»\ibible{Proverbs}(31:13). «También se levanta cuando aún es de noche, y da alimento a los de su casa, y tarea a sus doncellas»\ibible{Proverbs}(31:15).
\item\textbf{Su economía.} «Evalúa un campo y lo compra; con sus ganancias planta una viña»\ibible{Proverbs}(31:16). Porque ella maneja bien las finanzas, su esposo «no carecerá de ganancias»\ibible{Proverbs}(31:11).
\item\textbf{Su benevolencia.} «Extiende su mano al pobre, y alarga sus manos al necesitado»\ibible{Proverbs}(31:20).
\item\textbf{Su desinterés.} «No tiene temor de la nieve por los de su casa, porque todos los de su casa llevan ropa escarlata»\ibible{Proverbs}(31:21).
\item\textbf{Su aliento a su esposo.} «Su marido es conocido en las puertas, cuando se sienta con los ancianos de la tierra»\ibible{Proverbs}(31:23).
\item\textbf{Su sabiduría.} «Abre su boca con sabiduría»\ibible{Proverbs}(31:26).
\item\textbf{Su disposición bondadosa.} «Hay enseñanza de bondad en su lengua»\ibible{Proverbs}(31:26). 
\item\textbf{Su habilidad de disciplinar.} «Ella vigila la marcha de su casa»\ibible{Proverbs}(31:27).
\end{enumerate}
El rol de la madre ciertamente es difícil, pero las bendiciones pesan mucho mas que las dificultades. Con respeto a esta mujer virtuosa de Proverbios 31 las escrituras dicen: « Sus hijos se levantan y la llaman bienaventurada, también su marido, y la alaba diciendo: Muchas mujeres han obrado con nobleza, pero tú las superas a todas» (\ibibleverse{Proverbs}(31:28-29)).

Y, además, Engañosa es la gracia y vana la belleza, pero la mujer que teme al Señor, esa será alabada» (\ibiblechvs{Proverbs}(31:30)). Este escritor no sabe de ninguna mayor necesidad en este mundo que por las mujeres volver a sus roles ordenados por Dios. Alentadas por los movimientos de «derechos iguales» y «liberación de mujeres», mujeres han dejado sus hogares e hijos para entrar en el mundo de los negocios y trabajo secular. Un sentimiento de independencia puede desarrollarse, causando a la mujer, quien ahora tiene sus propios ingresos, a sentirse superior a su esposo y ya no sujeto a él o dependiente de él.

Los hijos se pueden descuidar. Ellos vienen de la escuela a una casa vacía mientras sus amigos tienen a «Madre» para saludarlos y escuchar sus cuentos de los eventos del día. Cuando la madre trabajadora sí llega, ella puede estar tan exhausta y frustrada que ella puede dar poco tiempo, atención, y amor a los hijos. Su vida gradualmente puede volver a ser centrada en su trabajo en vez de en su hogar.

Además, la mujer se puede influenciar por las personas mundanas con quienes ella está asociada en el trabajo. Muchas mujeres que nunca fumaron, maldijeron, o se vistieron inmodestamente antes de aceptar trabajos se influenciado a hacerlo después de comenzar a trabajar.

No es nuestro propósito traer estos cargos contra toda mujer que ha trabajado. Ciertamente, hemos conocido mujeres quienes, a causa de circunstancias desafortunadas, tuvieron que trabajar, y, habiendo reconocido los peligros potenciales, han buscado evitarlos. No tenemos nada sino admiración para tales mujeres. Pero creemos que toda persona honesta debe reconocer que muchos problemas sí existen porque tantas madres trabajan, y que \textbf{idealmente}, el lugar de la mujer está en el hogar, no en trabajo publico. Mujeres cristianas deben ser «hacendosas en el hogar» (\ibibleverse{Titus}(2:4-5)).

\section{Unas Cosas Que Quiero Enseñar A Mis Hijos}
«Y vosotros, padres, no provoquéis a ira a vuestros hijos, sino criadlos \textit{en la disciplina e instrucción del Señor}» (\ibibleverse{Ephesians}(6:4)). 

De acuerdo con las instrucciones de este pasaje, hay ciertas cosas que yo quiero enseñar a mis hijos.
\begin{enumerate}
\item\textbf{Yo quiero enseñar a mis hijos la debida reverencia y respeto en los periodos de adoración de la iglesia.} Cuando nos reunimos para adoración, estamos presentes delante de Dios para adorar a Dios. Cornelio entendió esto cuando él dijo, «Ahora, pues, todos nosotros estamos aquí presentes delante de Dios, para oír todo lo que el Señor te ha mandado» (\ibibleverse{Acts}(10:33)). Muchas personas seguramente han pensado poco en esta cuestión. Mujeres hablan entre si en el cuarto de los bebés. Jóvenes se ríen y hablan en la asamblea. Niños marchan en un desfile innecesario a los baños. Hombres y mujeres sanos quienes pueden ser tan ansiosos y entusiasmados en una subasta de antigüedades o un juego de pelota o un programa de la escuela se arrastran al periodo de adoración, se echan apáticos en el banco, y duermen por todo el sermón. Seguramente estas personas no reconocen que estamos presentes {delante de Dios para adorar a Dios}. Yo espero enseñar a mis hijos que ellos deben reverenciar a Dios, que ellos deben sentarse callados por todo el periodo de adoración, que ellos deben inclinar sus cabezas en oración, que ellos deben participar en los cantos, y que ellos deben evitar crear cualquier distracción innecesaria durante la adoración.
\item\textbf{Yo quiero enseñar a mis hijos a buscar primero el reino de Dios.} Jesús dijo, «Pero buscad primero su reino y su justicia, y todas estas cosas (cosas materiales de esta vida - BH) os serán añadidas» (\ibibleverse{Matthew}(6:33)). Yo quiero que mis hijos sepan que ellos creciendo para ser cristianos fieles me importa mas que cualquier cosa con respeto a ellos. Si ellos sobresalen en los deportes, sacan las mejores notas en la escuela, ganan sus doctorados, ganan competencias de belleza, y viven en lujo el resto de sus días, pero no son cristianos fieles, y por consiguiente van al infierno cuando mueren, yo habré fallado como padre. 

Si mi hijo quiere jugar en la liga juvenil de beisbol, debemos platicar con su entrenador \textbf{antes} de comprometernos, y explicar al entrenador que si un conflicto surge entre las actividades espirituales de mi hijo y sus actividades de beisbol, sus actividades espirituales deben tomar la prioridad. Después debemos actuar conforme a ese acuerdo consistentemente, sean lo que sean las presiones que surgen para hacer lo contrario. El mismo principio se debe aplicar en actividades de la escuela, actividades sociales, Boy Scouts, Guía Scouts, etc. Además, yo espero llevar a mis hijos, mientras crecen a la madurez, a tomar estas decisiones por sí mismos. Demasiados jóvenes se esconden detrás de sus padres con la frase conveniente, «Madre no me lo permite», en vez de tomar una posición por sus convicciones y luchar sin vergüenza por el Señor. Mi meta no es \textbf{hacerlos} buscar primero el reino, sino llevarlos a \textbf{querer} buscar primero el reino para que ellos sean agradables al Señor. 
\item\textbf{Yo quiero enseñar a mis hijos el respeto por autoridad:} por la autoridad de los padres (\ibibleverse{Ephesians}(6:1)), por la autoridad del gobierno (\ibibleverse{Matthew}(22:21)), y \textbf{sobre todo}, por la autoridad divina (\ibibleverse{Acts}(5:29)). Un niño se debe compadecer quien no se enseña el respeto por autoridad cuando está muy pequeño. Él se convierte en un problema en la clase bíblica, en la escuela, y en la comunidad. Mas tarde en la vida él es un problema en el trabajo; se mete en problemas con la ley; y finalmente él se pierde eternalmente, no habiendo respetado la autoridad de Dios.
\end{enumerate}
Hay muchas otras cosas además de estas que espero enseñar a mis hijos. El espacio no permite una discusión de fuerza de carácter, honestidad, justicia, buenos modales, etc., todos de los cuales yo espero enseñar a mis hijos.

Que nadie piense de este artículo como una \textbf{jactancia} de lo que yo haré; es una presentación de \textbf{metas}. Nadie es más consciente de la posibilidad del fracaso que yo. Pero mi esposa y yo oramos regularmente que Dios nos ayude a criar bien a nuestros hijos, y que Él anule nuestros errores. Mientras tanto, intentamos presentar un buen ejemplo delante de ellos; les enseñamos verbalmente, buscando guardar esas instrucciones que Dios dio a Israel: «Y estas palabras que yo te mando hoy, estarán sobre tu corazón; y diligentemente las enseñarás a tus hijos, y hablarás de ellas cuando te sientes en tu casa y cuando andes por el camino, cuando te acuestes y cuando te levantes» (\ibibleverse{Deuteronomy}(6:6-7)); e intentamos hacer esto consistentemente. Y si algún día sí vemos a nuestros hijos crecer para ser fieles al Señor, sabemos que será por la gracia de Dios, y le daremos a Él la gloria.

\section{Padres Permisivos}
Los hijos pueden influenciar a sus padres así como los padres pueden influenciar a sus hijos. Puede ser que la siguiente historia de una pareja \textbf{imaginaria} se haya duplicado en las vidas de muchos de nuestros lectores. 

Jorge y María eran una pareja maravillosa cuando ellos comenzaron su vida juntos. A través de su juventud, ellos habían recibido enseñanza fuerte con respeto a mundanería y su conducta mostró los efectos de esa enseñanza. Se les había enseñado fidelidad en asistencia y nunca faltaron en un servicio para «nada». En carácter y convicción, ellos eran intachables. 

Esta pareja joven falló, sin embargo, en inculcar en los corazones de sus hijos estas mismas convicciones. Por consiguiente, cuando los hijos llegaron a la adolescencia, ellos comenzaron a presionar a sus padres a dejarlos hacer lo que todos los otros jóvenes hacían. Gradualmente la voluntad de los padres se debilitó, y ellos comenzaron a permitir a sus hijos a hacer cosas que nunca soñaron que \textbf{sus} hijos harían.

La racionalización era fácil para Jorge y María. «Después de todo, la biblia no es especifica en estas cuestiones», ellos pensaron. «La biblia dice “ropa modesta”, pero no define modestia». «Y, solo están planeando ir al baile; no planean bailar». «No podemos decir “no” a todo», ellos dijeron. Cuando Junior comenzó a demostrar habilidad atlética inusual, la cuestión de asistir a los servicios se convirtió en un problema. Al principio ellos sacaron a Junior de los juegos y lo traían a los servicios a mediados de la semana, pero después el equipo comenzó a depender más y más de él. Llegó el desempate, y la única esperanza del equipo en el desempate fue que Junior jugara. Jorge y María cedieron. Y una vez que habían cedido, ya no tenían argumento para el futuro. Junior nunca faltó en otro juego para «ir a la iglesia».

Jorge y María a menudo se encontraron a la defensiva en clases bíblicas. Ellos comenzaron a argumentar a favor de la conducta de sus hijos. Y, lo mas que se acostumbraron a las acciones de sus hijos, lo mas inocente sus propias acciones parecían. Eventualmente su propia conducta se afectó. Ellos llegaron al punto donde pensaban nada de faltar la noche del viernes durante una serie en la iglesia para ver a Junior jugar pelota. María incluso adoptó unos de los hábitos de vestirse de su hija, aunque siguió siendo suficientemente «discreta» como para seguir disfrutando del favor de los hermanos. Sí, Jorge y María siguen en buen estado en la iglesia, y su cambio ha sido tan gradual que muchos no se han dado cuenta de que ellos ya no son los cristianos fuertes que eran anteriormente. ¿Qué pasó a Jorge y María? En vez de criar a sus hijos, levantándolos en la disciplina e instrucción del «Señor»\ibible{Ephesians}(6:4), sus hijos los bajaron en la disciplina e instrucción del «diablo».

Nuestros hijos pueden hacer lo malo, pero ¡ellos no deben hacer lo malo \textbf{con nuestro permiso!} No buscamos enojo, sino arrepentimiento. Padres, ¿encajarían sus nombres en el lugar de «Jorge» y «María» en la historia arriba?

\chapter{PUNTOS BREVES}
\section{Jesús Es Único}
Entre todos los que han vivido en la tierra, Jesús es único. Él \textbf{era} lo que ningún otro hombre haya sido. Él era «Immanuel», Dios con nosotros. Él \textbf{hizo} por la humanidad lo que ningún otro hombre haya hecho. Él murió por nuestros pecados. Él \textbf{merece} lo que ningún otro hombre haya merecido. Él merece nuestra adoración, veneración y obediencia completa. La cristiandad busca exaltar a Cristo. Él debe convertirse en el corazón y el centro de nuestras vidas.

\section{Exaltando a Cristo}
¿Cómo exaltamos a Cristo? Exaltamos a Cristo cuando predicamos \textbf{Su} palabra, cuando seguimos \textbf{Su} enseñanza, cuando hacemos solo lo que \textbf{Él} autoriza, cuando llevamos solo \textbf{Su} nombre, cuando hacemos que \textbf{Él} sea el centro de nuestra afección y adoración, cuando le reconocemos a \textbf{Él} como nuestra única Cabeza, Señor, y Rey. Hacer otra cosa es fallar de exaltarlo.

\section{¡Positivamente No Visitas!}
Recientemente, mientras visitábamos pacientes en un hospital, observamos varios letreros diciendo «No Visitas». Después llegamos a un letrero que decía, «Positivamente No Visitas». Supongo que algún día encontraremos uno que leerá, «Absolutamente Positivamente No Visitas». Si solo pudiéramos aprender a respetar autoridad para que «No Visitas» simplemente significaría no visitas.

La actitud de muchos hacia la autoridad del Señor es igual a la que tienen hacia un letrero diciendo «No Visitas». Ellos de alguna forma sienten que ellos son una excepción a la regla o que de algún modo el Señor pasará por alto esta sola instancia de desobedecer Su autoridad. Pero el mismo Dios quien no pasaría por alto el fuego extraño de Nadab y Abiú (\ibibleverse{Leviticus}(10:1-2)), el carro nuevo de David (\ibibleverse{IISamuel}(6:1-7)), o las tácticas engañadoras de Ananías y Safira\ibible{Acts}(6:1-11), no pasará por alto nuestra falta de respeto por Su autoridad.

Las palabras «positivamente» y «absolutamente» no se encuentran en La Biblia de las Américas. Pero sí se infieren en toda declaración y mandamiento encontrado en las escrituras. Uno no puede esperar ir a los cielos mientras nos falta respeto por la autoridad de Cristo.

\section{Vida Verdadera}
La meta del evangelio es preparar personas para los cielos. Su énfasis se centra en la felicidad de una vida venidera en vez de esta vida. «Pues ¿qué provecho obtendrá un hombre», Jesús preguntó una vez, «si gana el mundo entero, pero pierde su alma?»\ibible{Matthew}(16:26). El Señor nos enseñaría a ver por la fe mas allá del sufrimiento y las penas de esta vida a una vida libre de sufrimiento y penas, a guardar tesoros en los cielos en lugar de en la tierra. Él quiere que proclamemos en nuestra predicación un mensaje de salvación por medio de Su sangre, para llevar a la consideración de la gente esas cosas que son eternales. Pero cuando logramos guiar a hombres y mujeres a entregar sus vidas a favor de la vida venidera, los guiamos a ganar, no solo los cielos, sino también la paz y satisfacción que contribuye a la felicidad verdadera en esta tierra. «Él que quiera salvar su vida, la perderá», Jesús dijo, «pero el que pierda su vida por causa de mí, la hallará»\ibible{Matthew}(16:25).

\section{Hombres De Convicción}
Una gran necesidad en la iglesia es por hombres de convicción – hombres quienes saben lo que creen, por qué lo creen, y están viviendo conforme a sus convicciones. Tan a menudo vemos personas zambullirse de cabeza en algún proyecto religioso en su ansia de estar \textbf{haciendo} algo, cuando necesitan estar escudriñando las escrituras para aprender lo que Dios \textbf{quiere} que hagan. La de ellos es acción sin convicción; celo sin conocimiento. Jesús dijo, «No todo el que me dice: “Señor, Señor”, entrará en el reino de los cielos, sino el que hace la voluntad de mi Padre que está en los cielos» (\ibibleverse{Matthew}(7:21)). En el juicio, la pregunta no será, «¿Eras religioso?» sino, «¿Hiciste la voluntad del Padre?»

\section{Estándares Morales}
Los estándares morales de Dios son absolutos e inmutables. Cualquier cosa que era inmoral cuando se escribieron las escrituras es inmoral hoy en día. Filosofías modernas que buscarían justificar la homosexualidad, relaciones prematrimoniales, el divorcio y volver a casarse por cualquier causa, etc., son contrarias a la biblia y son degradantes a nuestra sociedad. El concepto popular de la ética situacional, cuando dejado a la aplicación de cada individuo, fácilmente se convierte en un concepto sin ética en absoluto. Esto también es contrario a las escrituras. «Y no os adaptéis a este mundo, sino transformaos mediante la renovación de vuestra mente» (\ibibleverse{Romans}(12:2)). Cristianos deben subir de alguna forma por encima de los estándares del mundo para vivir según los estándares de Dios. La justicia, sin arrogancia, es nuestra meta.

\section{Popularidad}
Jesús dijo una vez, «¡Ay de vosotros, cuando todos los hombres hablen bien de vosotros!» (\ibibleverse{Luke}(6:26)). Un hombre quien hace la voluntad de Dios no será popular con todos. Pero ay de ese hombre de quien nadie habla bien, porque, «Más vale el buen nombre que las muchas riquezas» (\ibibleverse{Proverbs}(22:1)). No debemos \textbf{buscar} la popularidad; no debemos \textbf{buscar} la persecución. Debemos buscar hacer la voluntad de Dios con fidelidad y amar a nuestros prójimos. Cuando hacemos esto, unos nos odiarán, pero otros nos apreciarán, y, viendo el reflejo de Cristo en nuestras vidas, glorificarán a Dios (\ibibleverse{Matthew}(5:16)).

\section{La Cabeza De La Iglesia}
La iglesia que reconoce a Jesucristo como su cabeza no hace lo que la mayoría de sus miembros quieren hacer, tampoco sigue los dictados de alguna conferencia o asociación o sínodo de la iglesia. La iglesia que reconoce a Jesús como su cabeza hace lo que Él ha autorizado en Su palabra. Nuestro Señor no aceptará el mero servicio de labios. Él dijo, «¿Y por qué me llamáis: “Señor, Señor”, y no hacéis lo que yo digo?» (\ibibleverse{Luke}(6:46)). Otra vez Él dijo, «Si me amáis, guardaréis mis mandamientos» (\ibibleverse{John}(14:15)). Cristo, como la cabeza de la iglesia, ha dado instrucciones con respeto a su adoración, su nombre, su organización, y su misión. Es solo cuando hayamos escrudiñado diligentemente esas instrucciones y las hayamos seguido que tenemos el derecho de referirnos a Jesús como nuestra Cabeza y Señor y Rey.

\section{La Iglesia Local}
El Señor arregló un plan por lo cual cristianos individuos juntarían su dinero (\ibibleverse{ICorinthians}(16:2)), para ser gastado bajo una administración común (ancianos – \ibibleverse{IPeter}(5:1-2), para cumplir una obra con la cual todos estaban igualmente relacionados. El resultado de este arreglo es una iglesia local, una organización bíblica. Él \textbf{no} arregló un plan por lo cual iglesias juntarían su dinero bajo una administración común (una junta directiva) para cumplir una obra con la cual todas estaban igualmente relacionadas. El resultado de este arreglo es una institución humana, misionera o benevolente, una organización no bíblica. La primera se ha autorizado por el Señor, la segunda no. La primera es de Dios; la segunda es de los hombres.

\section{Énfasis Espiritual}
La iglesia del Señor es un cuerpo espiritual con una misión espiritual. No puede competir con restaurantes en el campo de banquetes, con la televisión en el campo de entretenimiento, ni con escuelas en el campo de educación, porque el Señor no la estableció ni la capacitó para funcionar en estas esferas. Pero en enseñar a otros acerca de la voluntad de Dios y prepararlos para los cielos, la iglesia es sin igual, porque fue con este propósito que el Señor la edificó. La iglesia es la columna y sostén de la verdad (\ibibleverse{ITimothy}(3:15)). La iglesia es una «casa espiritual», ofreciendo «sacrificios espirituales aceptables a Dios por medio de Jesucristo» (\ibibleverse{IPeter}(2:5)).  Anhelamos ver el día cuando las iglesias de nuevo reconocerán su misión dada por Dios y volverán a ese énfasis espiritual enseñando en Su palabra.

\section{Crecimiento De La Iglesia}
La iglesia del Señor experimentó su mayor crecimiento en el primer siglo – en un tiempo cuando su única organización era congregacional y la iglesia local era su unidad funcional más grande. Tememos que los lideres religiosos del siglo veinte hayan organizado más que ellos han evangelizado. Fondos que hubieran llegado al evangelista en el primer siglo se están usando para presidentes de sociedades, secretarios, y papeleo en el siglo veinte. Organizaciones grandes nunca pueden sustituir por celo y dedicación personal. La gran necesidad de hoy en día es una vuelta a la organización sencilla de las iglesias del Nuevo Testamento y a ese celo personal motivador que caracterizó a esos primeros discípulos. Cuando estemos ardiendo por el Señor y vayamos a todos lados predicando la palabra, veremos de nuevo una gran afluencia de almas siendo salvadas y agregadas a la iglesia del Señor.

\section{Alcanzando A Los Perdidos}
El evangelio de Jesucristo es el mensaje más precioso que este mundo ha conocido. Es un mensaje que debe guardarse y protegerse a todo costo. El apóstol Pablo declaró, «Porque no me avergüenzo del evangelio, pues es el poder de Dios para la salvación de todo el que cree; del judío primeramente y también del griego» (\ibibleverse{Romans}(1:16)). Los que atraen a la gente a sus servicios ofreciendo viajes gratis a Six Flags o un equipo CB al adorador afortunado solo abaratan el evangelio y llenan las iglesias con personas de mente mundana. El evangelio no necesita promociones mundanas; no se puede mejorar por esquemas mundanas; no tiene atracción a la mente mundana. Pero para esa persona quien tiene hambre y sed de justicia, solo el evangelio puede traer satisfacción. 

\section{Escogiendo Candidatos}
Dios escogió a unos humildes pastores para ser los primeros de oír del nacimiento de Cristo. Jesús escogió a una mujer samaritana inmoral para ser la primera persona a quien Él se revelaría como el Mesías. Fue a María Magdalena, de quien Él había echado fuera a siete demonios, que Él apareció primero después de su resurrección. Dios no hace acepción de personas. Él quiere que el evangelio sea predicado a los buenos y a los malos, a los ricos y a los pobres, a los poderosos y a los humildes, a los religiosos y a los irreligiosos, a los rojos, amarillos, negros y blancos. «Id por todo el mundo y predicad el evangelio a toda criatura» (\ibibleverse{Mark}(16:15)).

Tendemos a ser selectivos. Buscamos por el que en nuestro juicio será el más receptivo, por la persona influyente que será una «ayuda» a la iglesia, por el hombre de la clase media quien encajará socialmente, por los moralmente buenos quienes no serán un problema para nosotros en el futuro. Tendemos a evitar alcohólicos, madres solteras, homosexuales, hippies, y drogadictos. Mientras es probable que hubiéramos visto a Lidia como un buen candidato, ciertamente hubiéramos pasado por alto a Saul de Tarso, o Simón el mago, o el carcelero filipense. Y en nuestra selectividad, podemos estar pasando por alto a nuestros mejores candidatos.

El evangelio brilla al máximo cuando vuelve el alcohólico a la sobriedad, el inmoral a la pureza de vida, el sucio a la limpieza, y el incrédulo a la fe. «Y esto erais algunos de vosotros», Pablo escribió a los corintios, «pero fuisteis lavados, pero fuisteis santificados, pero fuisteis justificados en el nombre del Señor Jesucristo y en el Espíritu de nuestro Dios» (\ibibleverse{ICorinthians}(6:11)). Que no subestimemos el poder del evangelio para cambiar los hombres.

\section{Predicación Entendible}
Una marca de un predicador verdaderamente bueno es la habilidad de explicar los temas difíciles de la biblia en palabras que toda persona de mente espiritual puede comprender. Esdras sirve como un gran ejemplo. Mientras él leía la ley de Dios delante del pueblo, él estaba «dándole el sentido para que entendieran la lectura» (\ibibleverse{Nehemiah}(8:8)). ¿Se gana algo por pararse enfrente de una audiencia de personas por cuarenta minutos, si nada de lo que se dice se entiende? ¿Acaso se están haciendo todas las cosas para edificación (\ibibleverse{ICorinthians}(14:26)) cuando tal es el caso? Alguien lo dijo de esta forma: «Muchos sermones son como la paz de Dios; sobrepasan todo entendimiento». ¡Muy cierto! ¡Muy cierto! Que no nos compliquemos. Los hermanos nos apreciarán más, y, sobre todo, los ayudaremos a ir a los cielos.

\section{Lo Que Damos A Nuestros Hijos}
Hace varios años yo visité a una dama cuya hija de dieciséis años se había desaparecido de repente. Ninguna palabra se había escuchado de ella en más que una semana. La madre no podía entender el porqué su niña había huido de casa. Ella abrió la puerta de un closet. «Mira toda esta ropa hermosa», ella dijo. «Queríamos que nuestra hija fuera tan bien vestida como cualquier otra chica en la escuela». Ella señaló a una televisión en la esquina del cuarto. «Teníamos una televisión en la sala, pero nuestra hija quería su propia televisión privada, entonces la compramos para ella», la madre dijo. Después ella señaló a una máquina de escribir. «Nuestra hija tomaba un curso de mecanografía, y pensamos que ella necesitaba una maquina de escribir en casa para practicar». Esa pareja había provisto toda cosa para su hija menos lo que ella necesitaba más que todo – \textbf{un amor por Dios y Su palabra.} 

Padres, si no están llevando sus hijos a los servicios de la iglesia, si no están enseñándolos la palabra de Dios, si no están inculcando en sus corazones una fe en Dios y sumisión a Su voluntad, están fallando como padres, no importa lo que pueden estar dando a sus hijos materialmente.

\backmatter
\clearpage
\addcontentsline{toc}{chapter}{\'Indice de Escrituras}
\renewcommand*{\BRbooktitlestyle}[1]{\textbf{#1}\nopagebreak} % book names in index will be in boldface
\printindex[scr]
\end{document}
